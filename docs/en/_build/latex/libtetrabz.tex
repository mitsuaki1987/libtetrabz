%% Generated by Sphinx.
\def\sphinxdocclass{report}
\documentclass[letterpaper,10pt,pdftex,openany,english]{sphinxmanual}
\ifdefined\pdfpxdimen
   \let\sphinxpxdimen\pdfpxdimen\else\newdimen\sphinxpxdimen
\fi \sphinxpxdimen=.75bp\relax
\ifdefined\pdfimageresolution
    \pdfimageresolution= \numexpr \dimexpr1in\relax/\sphinxpxdimen\relax
\fi
%% let collapsible pdf bookmarks panel have high depth per default
\PassOptionsToPackage{bookmarksdepth=5}{hyperref}

\PassOptionsToPackage{booktabs}{sphinx}
\PassOptionsToPackage{colorrows}{sphinx}

\PassOptionsToPackage{warn}{textcomp}
\usepackage[utf8]{inputenc}
\ifdefined\DeclareUnicodeCharacter
% support both utf8 and utf8x syntaxes
  \ifdefined\DeclareUnicodeCharacterAsOptional
    \def\sphinxDUC#1{\DeclareUnicodeCharacter{"#1}}
  \else
    \let\sphinxDUC\DeclareUnicodeCharacter
  \fi
  \sphinxDUC{00A0}{\nobreakspace}
  \sphinxDUC{2500}{\sphinxunichar{2500}}
  \sphinxDUC{2502}{\sphinxunichar{2502}}
  \sphinxDUC{2514}{\sphinxunichar{2514}}
  \sphinxDUC{251C}{\sphinxunichar{251C}}
  \sphinxDUC{2572}{\textbackslash}
\fi
\usepackage{cmap}
\usepackage[T1]{fontenc}
\usepackage{amsmath,amssymb,amstext}
\usepackage[english]{babel}



\usepackage{tgtermes}
\usepackage{tgheros}
\renewcommand{\ttdefault}{txtt}



\usepackage[Bjarne]{fncychap}
\usepackage{sphinx}

\fvset{fontsize=auto}
\usepackage{geometry}


% Include hyperref last.
\usepackage{hyperref}
% Fix anchor placement for figures with captions.
\usepackage{hypcap}% it must be loaded after hyperref.
% Set up styles of URL: it should be placed after hyperref.
\urlstyle{same}


\usepackage{sphinxmessages}
\setcounter{tocdepth}{2}



\title{Libtetrabz Documentation}
\date{Jun 17, 2023}
\release{2.0.0}
\author{kawamura}
\newcommand{\sphinxlogo}{\sphinxincludegraphics{libtetrabz.png}\par}
\renewcommand{\releasename}{Release}
\makeindex
\begin{document}

\ifdefined\shorthandoff
  \ifnum\catcode`\=\string=\active\shorthandoff{=}\fi
  \ifnum\catcode`\"=\active\shorthandoff{"}\fi
\fi

\pagestyle{empty}
\sphinxmaketitle
\pagestyle{plain}
\sphinxtableofcontents
\pagestyle{normal}
\phantomsection\label{\detokenize{index::doc}}


\sphinxstepscope


\chapter{Introduction}
\label{\detokenize{overview:introduction}}\label{\detokenize{overview::doc}}
\sphinxAtStartPar
This document explains a tetrahedron method library \sphinxcode{\sphinxupquote{libtetrabz}}.
\sphinxcode{\sphinxupquote{libtetrabz}} is a library to calculate the total energy, the charge
density, partial density of states, response functions, etc. in a solid
by using the optimized tetrahedron method {\hyperref[\detokenize{ref:ref}]{\sphinxcrossref{\DUrole{std,std-ref}{{[}1{]}}}}}.
Subroutines in this library receive the orbital (Kohn\sphinxhyphen{}Sham) energies as an input and
calculate weights \(w_{n n' k}\) for integration such as
\begin{equation*}
\begin{split}\begin{align}
\sum_{n n'}
\int_{\rm BZ} \frac{d^3 k}{V_{\rm BZ}}
F(\varepsilon_{n k}, \varepsilon_{n' k+q})X_{n n' k}
= \sum_{n n'} \sum_{k}^{N_k} w_{n n' k} X_{n n' k}
\end{align}\end{split}
\end{equation*}
\sphinxAtStartPar
\sphinxcode{\sphinxupquote{libtetrabz}} supports following Brillouin\sphinxhyphen{}zone integrations
\begin{equation*}
\begin{split}\begin{align}
\sum_{n}
\int_{\rm BZ} \frac{d^3 k}{V_{\rm BZ}}
\theta(\varepsilon_{\rm F} - \varepsilon_{n k})
X_{n k}
\end{align}\end{split}
\end{equation*}\begin{equation*}
\begin{split}\begin{align}
\sum_{n}
\int_{\rm BZ} \frac{d^3 k}{V_{\rm BZ}}
\delta(\omega - \varepsilon_{n k})
X_{n k}(\omega)
\end{align}\end{split}
\end{equation*}\begin{equation*}
\begin{split}\begin{align}
\sum_{n n'}
\int_{\rm BZ} \frac{d^3 k}{V_{\rm BZ}}
\delta(\varepsilon_{\rm F} - \varepsilon_{n k})
\delta(\varepsilon_{\rm F} - \varepsilon'_{n' k})
X_{n n' k}
\end{align}\end{split}
\end{equation*}\begin{equation*}
\begin{split}\begin{align}
\sum_{n n'}
\int_{\rm BZ} \frac{d^3 k}{V_{\rm BZ}}
\theta(\varepsilon_{\rm F} - \varepsilon_{n k})
\theta(\varepsilon_{n k} - \varepsilon'_{n' k})
X_{n n' k}
\end{align}\end{split}
\end{equation*}\begin{equation*}
\begin{split}\begin{align}
\sum_{n n'}
\int_{\rm BZ} \frac{d^3 k}{V_{\rm BZ}}
\frac{
\theta(\varepsilon_{\rm F} - \varepsilon_{n k})
\theta(\varepsilon'_{n' k} - \varepsilon_{\rm F})}
{\varepsilon'_{n' k} - \varepsilon_{n k}}
X_{n n' k}
\end{align}\end{split}
\end{equation*}\begin{equation*}
\begin{split}\begin{align}
\sum_{n n'}
\int_{\rm BZ} \frac{d^3 k}{V_{\rm BZ}}
\theta(\varepsilon_{\rm F} - \varepsilon_{n k})
\theta(\varepsilon'_{n' k} - \varepsilon_{\rm F})
\delta(\varepsilon'_{n' k} - \varepsilon_{n k} - \omega)
X_{n n' k}(\omega)
\end{align}\end{split}
\end{equation*}\begin{equation*}
\begin{split}\begin{align}
\sum_{n n'}
\int_{\rm BZ} \frac{d^3 k}{V_{\rm BZ}}
\frac{
\theta(\varepsilon_{\rm F} - \varepsilon_{n k})
\theta(\varepsilon'_{n' k} - \varepsilon_{\rm F})}
{\varepsilon'_{n' k} - \varepsilon_{n k} + i \omega}
X_{n n' k}(\omega)
\end{align}\end{split}
\end{equation*}
\sphinxstepscope


\chapter{Installation}
\label{\detokenize{install:installation}}\label{\detokenize{install::doc}}

\section{Important files and directories}
\label{\detokenize{install:important-files-and-directories}}\begin{itemize}
\item {} \begin{description}
\sphinxlineitem{\sphinxcode{\sphinxupquote{doc/}}}{[}Directory for manuals{]}\begin{itemize}
\item {} 
\sphinxAtStartPar
\sphinxcode{\sphinxupquote{doc/index.html}} : Index page

\end{itemize}

\end{description}

\item {} 
\sphinxAtStartPar
\sphinxcode{\sphinxupquote{src/}} : Directory for the sources of the library

\item {} 
\sphinxAtStartPar
\sphinxcode{\sphinxupquote{example/}} : Directory for the sample program

\item {} 
\sphinxAtStartPar
\sphinxcode{\sphinxupquote{test/}} : Directory for tests

\item {} 
\sphinxAtStartPar
\sphinxcode{\sphinxupquote{configure}} : Configuration script for the build

\end{itemize}


\section{Prerequisite}
\label{\detokenize{install:prerequisite}}
\sphinxAtStartPar
\sphinxcode{\sphinxupquote{libtetrabz}} requires the following
\begin{itemize}
\item {} 
\sphinxAtStartPar
fortran and C compiler

\item {} 
\sphinxAtStartPar
MPI library (If you use MPI/Hybrid version)

\end{itemize}


\section{Installation guide}
\label{\detokenize{install:installation-guide}}\begin{enumerate}
\sphinxsetlistlabels{\arabic}{enumi}{enumii}{}{.}%
\item {} 
\sphinxAtStartPar
Download \sphinxcode{\sphinxupquote{.tar.gz}} file from following web page.

\sphinxAtStartPar
\sphinxurl{https://github.com/mitsuaki1987/libtetrabz/releases/}

\item {} 
\sphinxAtStartPar
Uncompress \sphinxcode{\sphinxupquote{.tar.gz}} file and enter the generated directory.

\begin{sphinxVerbatim}[commandchars=\\\{\}]
\PYGZdl{}\PYG{+w}{ }tar\PYG{+w}{ }xzvf\PYG{+w}{ }xzvf\PYG{+w}{ }libtetrabz\PYGZus{}1.0.1.tar.gz
\PYGZdl{}\PYG{+w}{ }\PYG{n+nb}{cd}\PYG{+w}{ }libtetrabz
\end{sphinxVerbatim}

\item {} 
\sphinxAtStartPar
Configure the build environment.

\begin{sphinxVerbatim}[commandchars=\\\{\}]
\PYGZdl{}\PYG{+w}{ }./configure\PYG{+w}{ }\PYGZhy{}\PYGZhy{}prefix\PYG{o}{=}install\PYGZus{}dir
\end{sphinxVerbatim}

\sphinxAtStartPar
Then, this script checks the compiler and the libraries required for the installation,
and creates Makefiles.
\sphinxcode{\sphinxupquote{install\_dir}} indicates the full path of the directory where the library is installed
(you should replace it according to your case).
If none is specified, \sphinxcode{\sphinxupquote{/use/local/}} is chosen for storing libraries
by \sphinxcode{\sphinxupquote{make install}}  (Therefore, if one is not the admin, \sphinxcode{\sphinxupquote{install\_dir}} must be specified to
the different directory).
\sphinxcode{\sphinxupquote{configure}} has many options, and they are used according to the environment etc.
For more details, please see {\hyperref[\detokenize{install:configoption}]{\sphinxcrossref{\DUrole{std,std-ref}{Options for configure}}}}.

\item {} 
\sphinxAtStartPar
After \sphinxcode{\sphinxupquote{configure}} finishes successfully and Makefiles are generated,
please type

\begin{sphinxVerbatim}[commandchars=\\\{\}]
\PYGZdl{}\PYG{+w}{ }make
\end{sphinxVerbatim}

\sphinxAtStartPar
to build libraries. Then please type

\begin{sphinxVerbatim}[commandchars=\\\{\}]
\PYGZdl{}\PYG{+w}{ }make\PYG{+w}{ }install
\end{sphinxVerbatim}

\sphinxAtStartPar
to store libraries and the sample program to \sphinxcode{\sphinxupquote{install\_dir/lib}} and \sphinxcode{\sphinxupquote{install\_dir/bin}}, respectively.
Although one can use libraries and the sample program without \sphinxcode{\sphinxupquote{make install}},
they are a little different to the installed one.

\item {} 
\sphinxAtStartPar
Add the libtetrabz’s library directory (\sphinxcode{\sphinxupquote{install\_dir/lib}}) to the
search path of the dynamically linked program (environment variable \sphinxcode{\sphinxupquote{LD\_LIBRARY\_PATH}}).

\begin{sphinxVerbatim}[commandchars=\\\{\}]
\PYGZdl{}\PYG{+w}{ }\PYG{n+nb}{export}\PYG{+w}{ }\PYG{n+nv}{LD\PYGZus{}LIBRARY\PYGZus{}PATH}\PYG{o}{=}\PYG{l+s+si}{\PYGZdl{}\PYGZob{}}\PYG{n+nv}{LD\PYGZus{}LIBRARY\PYGZus{}PATH}\PYG{l+s+si}{\PYGZcb{}}:install\PYGZus{}dir/lib
\end{sphinxVerbatim}

\item {} 
\sphinxAtStartPar
Sample programs using \sphinxcode{\sphinxupquote{libtetrabz}} are also compiled in \sphinxcode{\sphinxupquote{example/}} .

\sphinxAtStartPar
\sphinxcode{\sphinxupquote{example/dos.x}} : Compute DOS of a tight\sphinxhyphen{}binding model in the cubic
lattice. The source code is \sphinxcode{\sphinxupquote{dos.f90}}
\begin{quote}

\begin{figure}[htbp]
\centering
\capstart

\noindent\sphinxincludegraphics[scale=0.5]{{dos}.png}
\caption{Density of states of the tight\sphinxhyphen{}binding model on the
cubic lattice calculated by using \sphinxcode{\sphinxupquote{dos.x}}.
The solid line indicates the
result converged about the number of \(k\).
“ \(+\) “ and “ \(\times\) “ indicate
results by using the linear tetrahedron method and the optimized
tetrahedron method,
respectively with \(8\times8\times8 k\) grid.}\label{\detokenize{install:id1}}\end{figure}
\end{quote}

\sphinxAtStartPar
\sphinxcode{\sphinxupquote{example/lindhard.x}} : Compute the Lindhard function. The source code
is \sphinxcode{\sphinxupquote{lindhard.f90}}
\begin{quote}

\begin{figure}[htbp]
\centering
\capstart

\noindent\sphinxincludegraphics[scale=0.5]{{lindhard}.png}
\caption{(solid line) The analytical result of the Lindhard
function. “ \(+\) “ and “ \(\times\) “ indicate results by using the linear
tetrahedron method and the optimized tetrahedron method, respectively
with \(8\times8\times8 k\) grid.}\label{\detokenize{install:id2}}\end{figure}
\end{quote}

\end{enumerate}


\section{Options for configure}
\label{\detokenize{install:options-for-configure}}\label{\detokenize{install:configoption}}
\sphinxAtStartPar
\sphinxcode{\sphinxupquote{configure}} has many options and environment variables.
They can be specified at once. E.g.

\begin{sphinxVerbatim}[commandchars=\\\{\}]
\PYGZdl{}\PYG{+w}{ }./configure\PYG{+w}{ }\PYGZhy{}\PYGZhy{}prefix\PYG{o}{=}/home/libtetrabz/\PYG{+w}{ }\PYGZhy{}\PYGZhy{}with\PYGZhy{}mpi\PYG{o}{=}yes\PYG{+w}{ }\PYG{n+nv}{FC}\PYG{o}{=}mpif90
\end{sphinxVerbatim}

\sphinxAtStartPar
All options and variables have default values.
We show a part of them as follows:

\sphinxAtStartPar
\sphinxcode{\sphinxupquote{\sphinxhyphen{}\sphinxhyphen{}\sphinxhyphen{}prefix}}
\begin{quote}

\sphinxAtStartPar
Default: \sphinxcode{\sphinxupquote{\sphinxhyphen{}\sphinxhyphen{}\sphinxhyphen{}prefix=/usr/local/}}.
Specify the directory where the library etc. are installed.
\end{quote}

\sphinxAtStartPar
\sphinxcode{\sphinxupquote{\sphinxhyphen{}\sphinxhyphen{}with\sphinxhyphen{}mpi}}
\begin{quote}

\sphinxAtStartPar
Default: \sphinxcode{\sphinxupquote{\sphinxhyphen{}\sphinxhyphen{}with\sphinxhyphen{}mpi=no}} (without MPI).
Whether use MPI (\sphinxcode{\sphinxupquote{\sphinxhyphen{}\sphinxhyphen{}with\sphinxhyphen{}mpi=yes}}), or not.
\end{quote}

\sphinxAtStartPar
\sphinxcode{\sphinxupquote{\sphinxhyphen{}\sphinxhyphen{}with\sphinxhyphen{}openmp}}
\begin{quote}

\sphinxAtStartPar
Default: \sphinxcode{\sphinxupquote{\sphinxhyphen{}\sphinxhyphen{}with\sphinxhyphen{}openmp=yes}} (with OpenMP).
Whether use OpenMP or not (\sphinxcode{\sphinxupquote{\sphinxhyphen{}\sphinxhyphen{}with\sphinxhyphen{}openmp=no}}).
\end{quote}

\sphinxAtStartPar
\sphinxcode{\sphinxupquote{\sphinxhyphen{}\sphinxhyphen{}enable\sphinxhyphen{}shared}}
\begin{quote}

\sphinxAtStartPar
Default: \sphinxcode{\sphinxupquote{\sphinxhyphen{}\sphinxhyphen{}enable\sphinxhyphen{}shared}}.
Whether generate shared library.
\end{quote}

\sphinxAtStartPar
\sphinxcode{\sphinxupquote{\sphinxhyphen{}\sphinxhyphen{}enable\sphinxhyphen{}static}}
\begin{quote}

\sphinxAtStartPar
Default: \sphinxcode{\sphinxupquote{\sphinxhyphen{}\sphinxhyphen{}enable\sphinxhyphen{}static}}.
Whether generate static library.
\end{quote}

\sphinxAtStartPar
\sphinxcode{\sphinxupquote{FC}}, \sphinxcode{\sphinxupquote{C}}
\begin{quote}

\sphinxAtStartPar
Default: The fortran/C compiler chosen automatically from those in the system.
When \sphinxcode{\sphinxupquote{\sphinxhyphen{}\sphinxhyphen{}with\sphinxhyphen{}mpi}} is specified, the corresponding MPI compiler
(such as \sphinxcode{\sphinxupquote{mpif90}}) is searched.
If \sphinxcode{\sphinxupquote{FC}} printed the end of the standard\sphinxhyphen{}output of \sphinxcode{\sphinxupquote{configure}} is not
what you want, please set it manually as \sphinxcode{\sphinxupquote{./configure FC=gfortran}}.
\end{quote}

\sphinxAtStartPar
\sphinxcode{\sphinxupquote{\sphinxhyphen{}\sphinxhyphen{}help}}
\begin{quote}

\sphinxAtStartPar
Display all options including above, and stop without configuration.
\end{quote}

\sphinxstepscope


\chapter{Linking libtetrabz}
\label{\detokenize{link:linking-libtetrabz}}\label{\detokenize{link::doc}}
\sphinxAtStartPar
\sphinxstylestrong{e. g. / For intel fortran}
\begin{quote}

\begin{sphinxVerbatim}[commandchars=\\\{\}]
\PYGZdl{}\PYG{+w}{ }ifort\PYG{+w}{ }program.f90\PYG{+w}{ }\PYGZhy{}L\PYG{+w}{ }install\PYGZus{}dir/lib\PYG{+w}{ }\PYGZhy{}I\PYG{+w}{ }install\PYGZus{}dir/include\PYG{+w}{ }\PYGZhy{}ltetrabz\PYG{+w}{ }\PYGZhy{}fopenmp
\PYGZdl{}\PYG{+w}{ }mpiifort\PYG{+w}{ }program.f90\PYG{+w}{ }\PYGZhy{}L\PYG{+w}{ }install\PYGZus{}dir/lib\PYG{+w}{ }\PYGZhy{}I\PYG{+w}{ }install\PYGZus{}dir/include\PYG{+w}{ }\PYGZhy{}ltetrabz\PYG{+w}{ }\PYGZhy{}fopenmp
\end{sphinxVerbatim}
\end{quote}

\sphinxAtStartPar
\sphinxstylestrong{e. g. / For intel C}
\begin{quote}

\begin{sphinxVerbatim}[commandchars=\\\{\}]
\PYGZdl{}\PYG{+w}{ }icc\PYG{+w}{ }\PYGZhy{}lifcore\PYG{+w}{ }program.f90\PYG{+w}{ }\PYGZhy{}L\PYG{+w}{ }install\PYGZus{}dir/lib\PYG{+w}{ }\PYGZhy{}I\PYG{+w}{ }install\PYGZus{}dir/include\PYG{+w}{ }\PYGZhy{}ltetrabz\PYG{+w}{ }\PYGZhy{}fopenmp
\PYGZdl{}\PYG{+w}{ }mpiicc\PYG{+w}{ }\PYGZhy{}lifcore\PYG{+w}{ }program.f90\PYG{+w}{ }\PYGZhy{}L\PYG{+w}{ }install\PYGZus{}dir/lib\PYG{+w}{ }\PYGZhy{}I\PYG{+w}{ }install\PYGZus{}dir/include\PYG{+w}{ }\PYGZhy{}ltetrabz\PYGZus{}mpi\PYG{+w}{ }\PYGZhy{}fopenmp
\end{sphinxVerbatim}
\end{quote}

\sphinxstepscope


\chapter{Subroutines}
\label{\detokenize{routine:subroutines}}\label{\detokenize{routine::doc}}
\sphinxAtStartPar
You can call a subroutine in this library as follows:

\begin{sphinxVerbatim}[commandchars=\\\{\}]
\PYG{k}{USE }\PYG{n}{libtetrabz}\PYG{p}{,}\PYG{+w}{ }\PYG{k}{ONLY}\PYG{+w}{ }\PYG{p}{:}\PYG{+w}{ }\PYG{n}{libtetrabz\PYGZus{}occ}

\PYG{k}{CALL }\PYG{n}{libtetrabz\PYGZus{}occ}\PYG{p}{(}\PYG{n}{ltetra}\PYG{p}{,}\PYG{n}{bvec}\PYG{p}{,}\PYG{n}{nb}\PYG{p}{,}\PYG{n}{nge}\PYG{p}{,}\PYG{n}{eig}\PYG{p}{,}\PYG{n}{ngw}\PYG{p}{,}\PYG{n}{wght}\PYG{p}{)}
\end{sphinxVerbatim}

\sphinxAtStartPar
Every subroutine has a name starts from \sphinxcode{\sphinxupquote{libtetrabz\_}}.

\sphinxAtStartPar
For the C program, it can be used as follows:

\begin{sphinxVerbatim}[commandchars=\\\{\}]
\PYG{c+cp}{\PYGZsh{}}\PYG{c+cp}{include}\PYG{+w}{ }\PYG{c+cpf}{\PYGZdq{}libtetrabz.h\PYGZdq{}}

\PYG{n}{libtetrabz\PYGZus{}occ}\PYG{p}{(}\PYG{o}{\PYGZam{}}\PYG{n}{ltetra}\PYG{p}{,}\PYG{n}{bvec}\PYG{p}{,}\PYG{o}{\PYGZam{}}\PYG{n}{nb}\PYG{p}{,}\PYG{n}{nge}\PYG{p}{,}\PYG{n}{eig}\PYG{p}{,}\PYG{n}{ngw}\PYG{p}{,}\PYG{n}{wght}\PYG{p}{)}
\end{sphinxVerbatim}

\sphinxAtStartPar
Variables should be passed as pointers.
Arrays should be declared as one dimensional arrays.
Also, the communicator argument for the routine should be
transformed from the C/C++’s one to the fortran’s one as follows.

\begin{sphinxVerbatim}[commandchars=\\\{\}]
\PYG{n}{comm\PYGZus{}f}\PYG{+w}{ }\PYG{o}{=}\PYG{+w}{ }\PYG{n}{MPI\PYGZus{}Comm\PYGZus{}c2f}\PYG{p}{(}\PYG{n}{comm\PYGZus{}c}\PYG{p}{)}\PYG{p}{;}
\end{sphinxVerbatim}


\section{Total energy, charge density, occupations}
\label{\detokenize{routine:total-energy-charge-density-occupations}}\begin{equation*}
\begin{split}\begin{align}
\sum_{n}
\int_{\rm BZ} \frac{d^3 k}{V_{\rm BZ}}
\theta(\varepsilon_{\rm F} -
\varepsilon_{n k}) X_{n k}
\end{align}\end{split}
\end{equation*}
\begin{sphinxVerbatim}[commandchars=\\\{\}]
\PYG{k}{CALL }\PYG{n}{libtetrabz\PYGZus{}occ}\PYG{p}{(}\PYG{n}{ltetra}\PYG{p}{,}\PYG{n}{bvec}\PYG{p}{,}\PYG{n}{nb}\PYG{p}{,}\PYG{n}{nge}\PYG{p}{,}\PYG{n}{eig}\PYG{p}{,}\PYG{n}{ngw}\PYG{p}{,}\PYG{n}{wght}\PYG{p}{,}\PYG{n}{comm}\PYG{p}{)}
\end{sphinxVerbatim}

\sphinxAtStartPar
Parameters
\begin{quote}

\begin{sphinxVerbatim}[commandchars=\\\{\}]
\PYG{k+kt}{INTEGER}\PYG{p}{,}\PYG{k}{INTENT}\PYG{p}{(}\PYG{n}{IN}\PYG{p}{)}\PYG{+w}{ }\PYG{k+kd}{::}\PYG{+w}{ }\PYG{n}{ltetra}
\end{sphinxVerbatim}
\begin{quote}

\sphinxAtStartPar
Specify the type of the tetrahedron method.
1 \(\cdots\) the linear tetrahedron method.
2 \(\cdots\) the optimized tetrahedron method {\hyperref[\detokenize{ref:ref}]{\sphinxcrossref{\DUrole{std,std-ref}{{[}1{]}}}}}.
\end{quote}

\begin{sphinxVerbatim}[commandchars=\\\{\}]
\PYG{k+kt}{REAL}\PYG{p}{(}\PYG{l+m+mi}{8}\PYG{p}{)}\PYG{p}{,}\PYG{k}{INTENT}\PYG{p}{(}\PYG{n}{IN}\PYG{p}{)}\PYG{+w}{ }\PYG{k+kd}{::}\PYG{+w}{ }\PYG{n}{bvec}\PYG{p}{(}\PYG{l+m+mi}{3}\PYG{p}{,}\PYG{l+m+mi}{3}\PYG{p}{)}
\end{sphinxVerbatim}
\begin{quote}

\sphinxAtStartPar
Reciprocal lattice vectors (arbitrary unit).
Because they are used to choose the direction of tetrahedra,
only their ratio is used.
\end{quote}

\begin{sphinxVerbatim}[commandchars=\\\{\}]
\PYG{k+kt}{INTEGER}\PYG{p}{,}\PYG{k}{INTENT}\PYG{p}{(}\PYG{n}{IN}\PYG{p}{)}\PYG{+w}{ }\PYG{k+kd}{::}\PYG{+w}{ }\PYG{n}{nb}
\end{sphinxVerbatim}
\begin{quote}

\sphinxAtStartPar
The number of bands.
\end{quote}

\begin{sphinxVerbatim}[commandchars=\\\{\}]
\PYG{k+kt}{INTEGER}\PYG{p}{,}\PYG{k}{INTENT}\PYG{p}{(}\PYG{n}{IN}\PYG{p}{)}\PYG{+w}{ }\PYG{k+kd}{::}\PYG{+w}{ }\PYG{n}{nge}\PYG{p}{(}\PYG{l+m+mi}{3}\PYG{p}{)}
\end{sphinxVerbatim}
\begin{quote}

\sphinxAtStartPar
Specify the \(k\)\sphinxhyphen{}grid
for input orbital energy.
\end{quote}

\begin{sphinxVerbatim}[commandchars=\\\{\}]
\PYG{k+kt}{REAL}\PYG{p}{(}\PYG{l+m+mi}{8}\PYG{p}{)}\PYG{p}{,}\PYG{k}{INTENT}\PYG{p}{(}\PYG{n}{IN}\PYG{p}{)}\PYG{+w}{ }\PYG{k+kd}{::}\PYG{+w}{ }\PYG{n}{eig}\PYG{p}{(}\PYG{n}{nb}\PYG{p}{,}\PYG{n}{nge}\PYG{p}{(}\PYG{l+m+mi}{1}\PYG{p}{)}\PYG{p}{,}\PYG{n}{nge}\PYG{p}{(}\PYG{l+m+mi}{2}\PYG{p}{)}\PYG{p}{,}\PYG{n}{nge}\PYG{p}{(}\PYG{l+m+mi}{3}\PYG{p}{)}\PYG{p}{)}
\end{sphinxVerbatim}
\begin{quote}

\sphinxAtStartPar
The orbital energy measured from the Fermi energy
( \(\varepsilon_{\rm F} = 0\) ).
\end{quote}

\begin{sphinxVerbatim}[commandchars=\\\{\}]
\PYG{k+kt}{INTEGER}\PYG{p}{,}\PYG{k}{INTENT}\PYG{p}{(}\PYG{n}{IN}\PYG{p}{)}\PYG{+w}{ }\PYG{k+kd}{::}\PYG{+w}{ }\PYG{n}{ngw}\PYG{p}{(}\PYG{l+m+mi}{3}\PYG{p}{)}
\end{sphinxVerbatim}
\begin{quote}

\sphinxAtStartPar
Specify the \(k\)\sphinxhyphen{}grid for output integration weights.
You can make \sphinxcode{\sphinxupquote{ngw}} \(\neq\) \sphinxcode{\sphinxupquote{nge}} (See {\hyperref[\detokenize{app:app}]{\sphinxcrossref{\DUrole{std,std-ref}{Appendix}}}}).
\end{quote}

\begin{sphinxVerbatim}[commandchars=\\\{\}]
\PYG{k+kt}{REAL}\PYG{p}{(}\PYG{l+m+mi}{8}\PYG{p}{)}\PYG{p}{,}\PYG{k}{INTENT}\PYG{p}{(}\PYG{n}{OUT}\PYG{p}{)}\PYG{+w}{ }\PYG{k+kd}{::}\PYG{+w}{ }\PYG{n}{wght}\PYG{p}{(}\PYG{n}{nb}\PYG{p}{,}\PYG{n}{ngw}\PYG{p}{(}\PYG{l+m+mi}{1}\PYG{p}{)}\PYG{p}{,}\PYG{n}{ngw}\PYG{p}{(}\PYG{l+m+mi}{2}\PYG{p}{)}\PYG{p}{,}\PYG{n}{ngw}\PYG{p}{(}\PYG{l+m+mi}{3}\PYG{p}{)}\PYG{p}{)}
\end{sphinxVerbatim}
\begin{quote}

\sphinxAtStartPar
The integration weights.
\end{quote}

\begin{sphinxVerbatim}[commandchars=\\\{\}]
\PYG{k+kt}{INTEGER}\PYG{p}{,}\PYG{k}{INTENT}\PYG{p}{(}\PYG{n}{IN}\PYG{p}{)}\PYG{p}{,}\PYG{k}{OPTIONAL}\PYG{+w}{ }\PYG{k+kd}{::}\PYG{+w}{ }\PYG{n}{comm}
\end{sphinxVerbatim}
\begin{quote}

\sphinxAtStartPar
Optional argument. Communicators for MPI such as \sphinxcode{\sphinxupquote{MPI\_COMM\_WORLD}}.
Only for MPI / Hybrid parallelization.
For C compiler without MPI, just pass \sphinxcode{\sphinxupquote{NULL}} to omit this argment.
\end{quote}
\end{quote}


\section{Fermi energy and occupations}
\label{\detokenize{routine:fermi-energy-and-occupations}}\begin{equation*}
\begin{split}\begin{align}
\sum_{n}
\int_{\rm BZ} \frac{d^3 k}{V_{\rm BZ}}
\theta(\varepsilon_{\rm F} -
\varepsilon_{n k}) X_{n k}
\end{align}\end{split}
\end{equation*}
\begin{sphinxVerbatim}[commandchars=\\\{\}]
\PYG{k}{CALL }\PYG{n}{libtetrabz\PYGZus{}fermieng}\PYG{p}{(}\PYG{n}{ltetra}\PYG{p}{,}\PYG{n}{bvec}\PYG{p}{,}\PYG{n}{nb}\PYG{p}{,}\PYG{n}{nge}\PYG{p}{,}\PYG{n}{eig}\PYG{p}{,}\PYG{n}{ngw}\PYG{p}{,}\PYG{n}{wght}\PYG{p}{,}\PYG{n}{ef}\PYG{p}{,}\PYG{n}{nelec}\PYG{p}{,}\PYG{n}{comm}\PYG{p}{)}
\end{sphinxVerbatim}

\sphinxAtStartPar
Parameters
\begin{quote}

\begin{sphinxVerbatim}[commandchars=\\\{\}]
\PYG{k+kt}{INTEGER}\PYG{p}{,}\PYG{k}{INTENT}\PYG{p}{(}\PYG{n}{IN}\PYG{p}{)}\PYG{+w}{ }\PYG{k+kd}{::}\PYG{+w}{ }\PYG{n}{ltetra}
\end{sphinxVerbatim}
\begin{quote}

\sphinxAtStartPar
Specify the type of the tetrahedron method.
1 \(\cdots\) the linear tetrahedron method.
2 \(\cdots\) the optimized tetrahedron method {\hyperref[\detokenize{ref:ref}]{\sphinxcrossref{\DUrole{std,std-ref}{{[}1{]}}}}}.
\end{quote}

\begin{sphinxVerbatim}[commandchars=\\\{\}]
\PYG{k+kt}{REAL}\PYG{p}{(}\PYG{l+m+mi}{8}\PYG{p}{)}\PYG{p}{,}\PYG{k}{INTENT}\PYG{p}{(}\PYG{n}{IN}\PYG{p}{)}\PYG{+w}{ }\PYG{k+kd}{::}\PYG{+w}{ }\PYG{n}{bvec}\PYG{p}{(}\PYG{l+m+mi}{3}\PYG{p}{,}\PYG{l+m+mi}{3}\PYG{p}{)}
\end{sphinxVerbatim}
\begin{quote}

\sphinxAtStartPar
Reciprocal lattice vectors (arbitrary unit).
Because they are used to choose the direction of tetrahedra,
only their ratio is used.
\end{quote}

\begin{sphinxVerbatim}[commandchars=\\\{\}]
\PYG{k+kt}{INTEGER}\PYG{p}{,}\PYG{k}{INTENT}\PYG{p}{(}\PYG{n}{IN}\PYG{p}{)}\PYG{+w}{ }\PYG{k+kd}{::}\PYG{+w}{ }\PYG{n}{nb}
\end{sphinxVerbatim}
\begin{quote}

\sphinxAtStartPar
The number of bands.
\end{quote}

\begin{sphinxVerbatim}[commandchars=\\\{\}]
\PYG{k+kt}{INTEGER}\PYG{p}{,}\PYG{k}{INTENT}\PYG{p}{(}\PYG{n}{IN}\PYG{p}{)}\PYG{+w}{ }\PYG{k+kd}{::}\PYG{+w}{ }\PYG{n}{nge}\PYG{p}{(}\PYG{l+m+mi}{3}\PYG{p}{)}
\end{sphinxVerbatim}
\begin{quote}

\sphinxAtStartPar
Specify the \(k\)\sphinxhyphen{}grid
for input orbital energy.
\end{quote}

\begin{sphinxVerbatim}[commandchars=\\\{\}]
\PYG{k+kt}{REAL}\PYG{p}{(}\PYG{l+m+mi}{8}\PYG{p}{)}\PYG{p}{,}\PYG{k}{INTENT}\PYG{p}{(}\PYG{n}{IN}\PYG{p}{)}\PYG{+w}{ }\PYG{k+kd}{::}\PYG{+w}{ }\PYG{n}{eig}\PYG{p}{(}\PYG{n}{nb}\PYG{p}{,}\PYG{n}{nge}\PYG{p}{(}\PYG{l+m+mi}{1}\PYG{p}{)}\PYG{p}{,}\PYG{n}{nge}\PYG{p}{(}\PYG{l+m+mi}{2}\PYG{p}{)}\PYG{p}{,}\PYG{n}{nge}\PYG{p}{(}\PYG{l+m+mi}{3}\PYG{p}{)}\PYG{p}{)}
\end{sphinxVerbatim}
\begin{quote}

\sphinxAtStartPar
The orbital energy measured from the Fermi energy
( \(\varepsilon_{\rm F} = 0\) ).
\end{quote}

\begin{sphinxVerbatim}[commandchars=\\\{\}]
\PYG{k+kt}{INTEGER}\PYG{p}{,}\PYG{k}{INTENT}\PYG{p}{(}\PYG{n}{IN}\PYG{p}{)}\PYG{+w}{ }\PYG{k+kd}{::}\PYG{+w}{ }\PYG{n}{ngw}\PYG{p}{(}\PYG{l+m+mi}{3}\PYG{p}{)}
\end{sphinxVerbatim}
\begin{quote}

\sphinxAtStartPar
Specify the \(k\)\sphinxhyphen{}grid for output integration weights.
You can make \sphinxcode{\sphinxupquote{ngw}} \(\neq\) \sphinxcode{\sphinxupquote{nge}} (See {\hyperref[\detokenize{app:app}]{\sphinxcrossref{\DUrole{std,std-ref}{Appendix}}}}).
\end{quote}

\begin{sphinxVerbatim}[commandchars=\\\{\}]
\PYG{k+kt}{REAL}\PYG{p}{(}\PYG{l+m+mi}{8}\PYG{p}{)}\PYG{p}{,}\PYG{k}{INTENT}\PYG{p}{(}\PYG{n}{OUT}\PYG{p}{)}\PYG{+w}{ }\PYG{k+kd}{::}\PYG{+w}{ }\PYG{n}{wght}\PYG{p}{(}\PYG{n}{nb}\PYG{p}{,}\PYG{n}{ngw}\PYG{p}{(}\PYG{l+m+mi}{1}\PYG{p}{)}\PYG{p}{,}\PYG{n}{ngw}\PYG{p}{(}\PYG{l+m+mi}{2}\PYG{p}{)}\PYG{p}{,}\PYG{n}{ngw}\PYG{p}{(}\PYG{l+m+mi}{3}\PYG{p}{)}\PYG{p}{)}
\end{sphinxVerbatim}
\begin{quote}

\sphinxAtStartPar
The integration weights.
\end{quote}

\begin{sphinxVerbatim}[commandchars=\\\{\}]
\PYG{k+kt}{REAL}\PYG{p}{(}\PYG{l+m+mi}{8}\PYG{p}{)}\PYG{p}{,}\PYG{k}{INTENT}\PYG{p}{(}\PYG{n}{OUT}\PYG{p}{)}\PYG{+w}{ }\PYG{k+kd}{::}\PYG{+w}{ }\PYG{n}{ef}
\end{sphinxVerbatim}
\begin{quote}

\sphinxAtStartPar
The Fermi energy.
\end{quote}

\begin{sphinxVerbatim}[commandchars=\\\{\}]
\PYG{k+kt}{REAL}\PYG{p}{(}\PYG{l+m+mi}{8}\PYG{p}{)}\PYG{p}{,}\PYG{k}{INTENT}\PYG{p}{(}\PYG{n}{IN}\PYG{p}{)}\PYG{+w}{ }\PYG{k+kd}{::}\PYG{+w}{ }\PYG{n}{nelec}
\end{sphinxVerbatim}
\begin{quote}

\sphinxAtStartPar
The number of (valence) electrons per spin.
\end{quote}

\begin{sphinxVerbatim}[commandchars=\\\{\}]
\PYG{k+kt}{INTEGER}\PYG{p}{,}\PYG{k}{INTENT}\PYG{p}{(}\PYG{n}{IN}\PYG{p}{)}\PYG{p}{,}\PYG{k}{OPTIONAL}\PYG{+w}{ }\PYG{k+kd}{::}\PYG{+w}{ }\PYG{n}{comm}
\end{sphinxVerbatim}
\begin{quote}

\sphinxAtStartPar
Optional argument. Communicators for MPI such as \sphinxcode{\sphinxupquote{MPI\_COMM\_WORLD}}.
Only for MPI / Hybrid parallelization.
For C compiler without MPI, just pass \sphinxcode{\sphinxupquote{NULL}} to omit this argment.
\end{quote}
\end{quote}


\section{Partial density of states}
\label{\detokenize{routine:partial-density-of-states}}\begin{equation*}
\begin{split}\begin{align}
\sum_{n}
\int_{\rm BZ} \frac{d^3 k}{V_{\rm BZ}}
\delta(\omega - \varepsilon_{n k})
X_{n k}(\omega)
\end{align}\end{split}
\end{equation*}
\begin{sphinxVerbatim}[commandchars=\\\{\}]
\PYG{k}{CALL }\PYG{n}{libtetrabz\PYGZus{}dos}\PYG{p}{(}\PYG{n}{ltetra}\PYG{p}{,}\PYG{n}{bvec}\PYG{p}{,}\PYG{n}{nb}\PYG{p}{,}\PYG{n}{nge}\PYG{p}{,}\PYG{n}{eig}\PYG{p}{,}\PYG{n}{ngw}\PYG{p}{,}\PYG{n}{wght}\PYG{p}{,}\PYG{n}{ne}\PYG{p}{,}\PYG{n}{e0}\PYG{p}{,}\PYG{n}{comm}\PYG{p}{)}
\end{sphinxVerbatim}

\sphinxAtStartPar
Parameters
\begin{quote}

\begin{sphinxVerbatim}[commandchars=\\\{\}]
\PYG{k+kt}{INTEGER}\PYG{p}{,}\PYG{k}{INTENT}\PYG{p}{(}\PYG{n}{IN}\PYG{p}{)}\PYG{+w}{ }\PYG{k+kd}{::}\PYG{+w}{ }\PYG{n}{ltetra}
\end{sphinxVerbatim}
\begin{quote}

\sphinxAtStartPar
Specify the type of the tetrahedron method.
1 \(\cdots\) the linear tetrahedron method.
2 \(\cdots\) the optimized tetrahedron method {\hyperref[\detokenize{ref:ref}]{\sphinxcrossref{\DUrole{std,std-ref}{{[}1{]}}}}}.
\end{quote}

\begin{sphinxVerbatim}[commandchars=\\\{\}]
\PYG{k+kt}{REAL}\PYG{p}{(}\PYG{l+m+mi}{8}\PYG{p}{)}\PYG{p}{,}\PYG{k}{INTENT}\PYG{p}{(}\PYG{n}{IN}\PYG{p}{)}\PYG{+w}{ }\PYG{k+kd}{::}\PYG{+w}{ }\PYG{n}{bvec}\PYG{p}{(}\PYG{l+m+mi}{3}\PYG{p}{,}\PYG{l+m+mi}{3}\PYG{p}{)}
\end{sphinxVerbatim}
\begin{quote}

\sphinxAtStartPar
Reciprocal lattice vectors (arbitrary unit).
Because they are used to choose the direction of tetrahedra,
only their ratio is used.
\end{quote}

\begin{sphinxVerbatim}[commandchars=\\\{\}]
\PYG{k+kt}{INTEGER}\PYG{p}{,}\PYG{k}{INTENT}\PYG{p}{(}\PYG{n}{IN}\PYG{p}{)}\PYG{+w}{ }\PYG{k+kd}{::}\PYG{+w}{ }\PYG{n}{nb}
\end{sphinxVerbatim}
\begin{quote}

\sphinxAtStartPar
The number of bands.
\end{quote}

\begin{sphinxVerbatim}[commandchars=\\\{\}]
\PYG{k+kt}{INTEGER}\PYG{p}{,}\PYG{k}{INTENT}\PYG{p}{(}\PYG{n}{IN}\PYG{p}{)}\PYG{+w}{ }\PYG{k+kd}{::}\PYG{+w}{ }\PYG{n}{nge}\PYG{p}{(}\PYG{l+m+mi}{3}\PYG{p}{)}
\end{sphinxVerbatim}
\begin{quote}

\sphinxAtStartPar
Specify the \(k\)\sphinxhyphen{}grid
for input orbital energy.
\end{quote}

\begin{sphinxVerbatim}[commandchars=\\\{\}]
\PYG{k+kt}{REAL}\PYG{p}{(}\PYG{l+m+mi}{8}\PYG{p}{)}\PYG{p}{,}\PYG{k}{INTENT}\PYG{p}{(}\PYG{n}{IN}\PYG{p}{)}\PYG{+w}{ }\PYG{k+kd}{::}\PYG{+w}{ }\PYG{n}{eig}\PYG{p}{(}\PYG{n}{nb}\PYG{p}{,}\PYG{n}{nge}\PYG{p}{(}\PYG{l+m+mi}{1}\PYG{p}{)}\PYG{p}{,}\PYG{n}{nge}\PYG{p}{(}\PYG{l+m+mi}{2}\PYG{p}{)}\PYG{p}{,}\PYG{n}{nge}\PYG{p}{(}\PYG{l+m+mi}{3}\PYG{p}{)}\PYG{p}{)}
\end{sphinxVerbatim}
\begin{quote}

\sphinxAtStartPar
The orbital energy measured from the Fermi energy
( \(\varepsilon_{\rm F} = 0\) ).
\end{quote}

\begin{sphinxVerbatim}[commandchars=\\\{\}]
\PYG{k+kt}{INTEGER}\PYG{p}{,}\PYG{k}{INTENT}\PYG{p}{(}\PYG{n}{IN}\PYG{p}{)}\PYG{+w}{ }\PYG{k+kd}{::}\PYG{+w}{ }\PYG{n}{ngw}\PYG{p}{(}\PYG{l+m+mi}{3}\PYG{p}{)}
\end{sphinxVerbatim}
\begin{quote}

\sphinxAtStartPar
Specify the \(k\)\sphinxhyphen{}grid for output integration weights.
You can make \sphinxcode{\sphinxupquote{ngw}} \(\neq\) \sphinxcode{\sphinxupquote{nge}} (See {\hyperref[\detokenize{app:app}]{\sphinxcrossref{\DUrole{std,std-ref}{Appendix}}}}).
\end{quote}

\begin{sphinxVerbatim}[commandchars=\\\{\}]
\PYG{k+kt}{REAL}\PYG{p}{(}\PYG{l+m+mi}{8}\PYG{p}{)}\PYG{p}{,}\PYG{k}{INTENT}\PYG{p}{(}\PYG{n}{OUT}\PYG{p}{)}\PYG{+w}{ }\PYG{k+kd}{::}\PYG{+w}{ }\PYG{n}{wght}\PYG{p}{(}\PYG{n}{ne}\PYG{p}{,}\PYG{n}{nb}\PYG{p}{,}\PYG{n}{ngw}\PYG{p}{(}\PYG{l+m+mi}{1}\PYG{p}{)}\PYG{p}{,}\PYG{n}{ngw}\PYG{p}{(}\PYG{l+m+mi}{2}\PYG{p}{)}\PYG{p}{,}\PYG{n}{ngw}\PYG{p}{(}\PYG{l+m+mi}{3}\PYG{p}{)}\PYG{p}{)}
\end{sphinxVerbatim}
\begin{quote}

\sphinxAtStartPar
The integration weights.
\end{quote}

\begin{sphinxVerbatim}[commandchars=\\\{\}]
\PYG{k+kt}{INTEGER}\PYG{p}{,}\PYG{k}{INTENT}\PYG{p}{(}\PYG{n}{IN}\PYG{p}{)}\PYG{+w}{ }\PYG{k+kd}{::}\PYG{+w}{ }\PYG{n}{ne}
\end{sphinxVerbatim}
\begin{quote}

\sphinxAtStartPar
The number of energy where DOS is calculated.
\end{quote}

\begin{sphinxVerbatim}[commandchars=\\\{\}]
\PYG{k+kt}{REAL}\PYG{p}{(}\PYG{l+m+mi}{8}\PYG{p}{)}\PYG{p}{,}\PYG{k}{INTENT}\PYG{p}{(}\PYG{n}{IN}\PYG{p}{)}\PYG{+w}{ }\PYG{k+kd}{::}\PYG{+w}{ }\PYG{n}{e0}\PYG{p}{(}\PYG{n}{ne}\PYG{p}{)}
\end{sphinxVerbatim}
\begin{quote}

\sphinxAtStartPar
Energies where DOS is calculated.
\end{quote}

\begin{sphinxVerbatim}[commandchars=\\\{\}]
\PYG{k+kt}{INTEGER}\PYG{p}{,}\PYG{k}{INTENT}\PYG{p}{(}\PYG{n}{IN}\PYG{p}{)}\PYG{p}{,}\PYG{k}{OPTIONAL}\PYG{+w}{ }\PYG{k+kd}{::}\PYG{+w}{ }\PYG{n}{comm}
\end{sphinxVerbatim}
\begin{quote}

\sphinxAtStartPar
Optional argument. Communicators for MPI such as \sphinxcode{\sphinxupquote{MPI\_COMM\_WORLD}}.
Only for MPI / Hybrid parallelization.
For C compiler without MPI, just pass \sphinxcode{\sphinxupquote{NULL}} to omit this argment.
\end{quote}
\end{quote}


\section{Integrated density of states}
\label{\detokenize{routine:integrated-density-of-states}}\begin{equation*}
\begin{split}\begin{align}
\sum_{n}
\int_{\rm BZ} \frac{d^3 k}{V_{\rm BZ}}
\theta(\omega - \varepsilon_{n k})
X_{n k}(\omega)
\end{align}\end{split}
\end{equation*}
\begin{sphinxVerbatim}[commandchars=\\\{\}]
\PYG{k}{CALL }\PYG{n}{libtetrabz\PYGZus{}intdos}\PYG{p}{(}\PYG{n}{ltetra}\PYG{p}{,}\PYG{n}{bvec}\PYG{p}{,}\PYG{n}{nb}\PYG{p}{,}\PYG{n}{nge}\PYG{p}{,}\PYG{n}{eig}\PYG{p}{,}\PYG{n}{ngw}\PYG{p}{,}\PYG{n}{wght}\PYG{p}{,}\PYG{n}{ne}\PYG{p}{,}\PYG{n}{e0}\PYG{p}{,}\PYG{n}{comm}\PYG{p}{)}
\end{sphinxVerbatim}

\sphinxAtStartPar
Parameters
\begin{quote}

\begin{sphinxVerbatim}[commandchars=\\\{\}]
\PYG{k+kt}{INTEGER}\PYG{p}{,}\PYG{k}{INTENT}\PYG{p}{(}\PYG{n}{IN}\PYG{p}{)}\PYG{+w}{ }\PYG{k+kd}{::}\PYG{+w}{ }\PYG{n}{ltetra}
\end{sphinxVerbatim}
\begin{quote}

\sphinxAtStartPar
Specify the type of the tetrahedron method.
1 \(\cdots\) the linear tetrahedron method.
2 \(\cdots\) the optimized tetrahedron method {\hyperref[\detokenize{ref:ref}]{\sphinxcrossref{\DUrole{std,std-ref}{{[}1{]}}}}}.
\end{quote}

\begin{sphinxVerbatim}[commandchars=\\\{\}]
\PYG{k+kt}{REAL}\PYG{p}{(}\PYG{l+m+mi}{8}\PYG{p}{)}\PYG{p}{,}\PYG{k}{INTENT}\PYG{p}{(}\PYG{n}{IN}\PYG{p}{)}\PYG{+w}{ }\PYG{k+kd}{::}\PYG{+w}{ }\PYG{n}{bvec}\PYG{p}{(}\PYG{l+m+mi}{3}\PYG{p}{,}\PYG{l+m+mi}{3}\PYG{p}{)}
\end{sphinxVerbatim}
\begin{quote}

\sphinxAtStartPar
Reciprocal lattice vectors (arbitrary unit).
Because they are used to choose the direction of tetrahedra,
only their ratio is used.
\end{quote}

\begin{sphinxVerbatim}[commandchars=\\\{\}]
\PYG{k+kt}{INTEGER}\PYG{p}{,}\PYG{k}{INTENT}\PYG{p}{(}\PYG{n}{IN}\PYG{p}{)}\PYG{+w}{ }\PYG{k+kd}{::}\PYG{+w}{ }\PYG{n}{nb}
\end{sphinxVerbatim}
\begin{quote}

\sphinxAtStartPar
The number of bands.
\end{quote}

\begin{sphinxVerbatim}[commandchars=\\\{\}]
\PYG{k+kt}{INTEGER}\PYG{p}{,}\PYG{k}{INTENT}\PYG{p}{(}\PYG{n}{IN}\PYG{p}{)}\PYG{+w}{ }\PYG{k+kd}{::}\PYG{+w}{ }\PYG{n}{nge}\PYG{p}{(}\PYG{l+m+mi}{3}\PYG{p}{)}
\end{sphinxVerbatim}
\begin{quote}

\sphinxAtStartPar
Specify the \(k\)\sphinxhyphen{}grid
for input orbital energy.
\end{quote}

\begin{sphinxVerbatim}[commandchars=\\\{\}]
\PYG{k+kt}{REAL}\PYG{p}{(}\PYG{l+m+mi}{8}\PYG{p}{)}\PYG{p}{,}\PYG{k}{INTENT}\PYG{p}{(}\PYG{n}{IN}\PYG{p}{)}\PYG{+w}{ }\PYG{k+kd}{::}\PYG{+w}{ }\PYG{n}{eig}\PYG{p}{(}\PYG{n}{nb}\PYG{p}{,}\PYG{n}{nge}\PYG{p}{(}\PYG{l+m+mi}{1}\PYG{p}{)}\PYG{p}{,}\PYG{n}{nge}\PYG{p}{(}\PYG{l+m+mi}{2}\PYG{p}{)}\PYG{p}{,}\PYG{n}{nge}\PYG{p}{(}\PYG{l+m+mi}{3}\PYG{p}{)}\PYG{p}{)}
\end{sphinxVerbatim}
\begin{quote}

\sphinxAtStartPar
The orbital energy measured from the Fermi energy
( \(\varepsilon_{\rm F} = 0\) ).
\end{quote}

\begin{sphinxVerbatim}[commandchars=\\\{\}]
\PYG{k+kt}{INTEGER}\PYG{p}{,}\PYG{k}{INTENT}\PYG{p}{(}\PYG{n}{IN}\PYG{p}{)}\PYG{+w}{ }\PYG{k+kd}{::}\PYG{+w}{ }\PYG{n}{ngw}\PYG{p}{(}\PYG{l+m+mi}{3}\PYG{p}{)}
\end{sphinxVerbatim}
\begin{quote}

\sphinxAtStartPar
Specify the \(k\)\sphinxhyphen{}grid for output integration weights.
You can make \sphinxcode{\sphinxupquote{ngw}} \(\neq\) \sphinxcode{\sphinxupquote{nge}} (See {\hyperref[\detokenize{app:app}]{\sphinxcrossref{\DUrole{std,std-ref}{Appendix}}}}).
\end{quote}

\begin{sphinxVerbatim}[commandchars=\\\{\}]
\PYG{k+kt}{REAL}\PYG{p}{(}\PYG{l+m+mi}{8}\PYG{p}{)}\PYG{p}{,}\PYG{k}{INTENT}\PYG{p}{(}\PYG{n}{OUT}\PYG{p}{)}\PYG{+w}{ }\PYG{k+kd}{::}\PYG{+w}{ }\PYG{n}{wght}\PYG{p}{(}\PYG{n}{ne}\PYG{p}{,}\PYG{n}{nb}\PYG{p}{,}\PYG{n}{ngw}\PYG{p}{(}\PYG{l+m+mi}{1}\PYG{p}{)}\PYG{p}{,}\PYG{n}{ngw}\PYG{p}{(}\PYG{l+m+mi}{2}\PYG{p}{)}\PYG{p}{,}\PYG{n}{ngw}\PYG{p}{(}\PYG{l+m+mi}{3}\PYG{p}{)}\PYG{p}{)}
\end{sphinxVerbatim}
\begin{quote}

\sphinxAtStartPar
The integration weights.
\end{quote}

\begin{sphinxVerbatim}[commandchars=\\\{\}]
\PYG{k+kt}{INTEGER}\PYG{p}{,}\PYG{k}{INTENT}\PYG{p}{(}\PYG{n}{IN}\PYG{p}{)}\PYG{+w}{ }\PYG{k+kd}{::}\PYG{+w}{ }\PYG{n}{ne}
\end{sphinxVerbatim}
\begin{quote}

\sphinxAtStartPar
The number of energy where DOS is calculated.
\end{quote}

\begin{sphinxVerbatim}[commandchars=\\\{\}]
\PYG{k+kt}{REAL}\PYG{p}{(}\PYG{l+m+mi}{8}\PYG{p}{)}\PYG{p}{,}\PYG{k}{INTENT}\PYG{p}{(}\PYG{n}{IN}\PYG{p}{)}\PYG{+w}{ }\PYG{k+kd}{::}\PYG{+w}{ }\PYG{n}{e0}\PYG{p}{(}\PYG{n}{ne}\PYG{p}{)}
\end{sphinxVerbatim}
\begin{quote}

\sphinxAtStartPar
Energies where DOS is calculated.
\end{quote}

\begin{sphinxVerbatim}[commandchars=\\\{\}]
\PYG{k+kt}{INTEGER}\PYG{p}{,}\PYG{k}{INTENT}\PYG{p}{(}\PYG{n}{IN}\PYG{p}{)}\PYG{p}{,}\PYG{k}{OPTIONAL}\PYG{+w}{ }\PYG{k+kd}{::}\PYG{+w}{ }\PYG{n}{comm}
\end{sphinxVerbatim}
\begin{quote}

\sphinxAtStartPar
Optional argument. Communicators for MPI such as \sphinxcode{\sphinxupquote{MPI\_COMM\_WORLD}}.
Only for MPI / Hybrid parallelization.
For C compiler without MPI, just pass \sphinxcode{\sphinxupquote{NULL}} to omit this argment.
\end{quote}
\end{quote}


\section{Nesting function and Fr\&oumlhlich parameter}
\label{\detokenize{routine:nesting-function-and-fr-oumlhlich-parameter}}\begin{equation*}
\begin{split}\begin{align}
\sum_{n n'}
\int_{\rm BZ} \frac{d^3 k}{V_{\rm BZ}}
\delta(\varepsilon_{\rm F} -
\varepsilon_{n k}) \delta(\varepsilon_{\rm F} - \varepsilon'_{n' k})
X_{n n' k}
\end{align}\end{split}
\end{equation*}
\begin{sphinxVerbatim}[commandchars=\\\{\}]
\PYG{k}{CALL }\PYG{n}{libtetrabz\PYGZus{}dbldelta}\PYG{p}{(}\PYG{n}{ltetra}\PYG{p}{,}\PYG{n}{bvec}\PYG{p}{,}\PYG{n}{nb}\PYG{p}{,}\PYG{n}{nge}\PYG{p}{,}\PYG{n}{eig1}\PYG{p}{,}\PYG{n}{eig2}\PYG{p}{,}\PYG{n}{ngw}\PYG{p}{,}\PYG{n}{wght}\PYG{p}{,}\PYG{n}{comm}\PYG{p}{)}
\end{sphinxVerbatim}

\sphinxAtStartPar
Parameters
\begin{quote}

\begin{sphinxVerbatim}[commandchars=\\\{\}]
\PYG{k+kt}{INTEGER}\PYG{p}{,}\PYG{k}{INTENT}\PYG{p}{(}\PYG{n}{IN}\PYG{p}{)}\PYG{+w}{ }\PYG{k+kd}{::}\PYG{+w}{ }\PYG{n}{ltetra}
\end{sphinxVerbatim}
\begin{quote}

\sphinxAtStartPar
Specify the type of the tetrahedron method.
1 \(\cdots\) the linear tetrahedron method.
2 \(\cdots\) the optimized tetrahedron method {\hyperref[\detokenize{ref:ref}]{\sphinxcrossref{\DUrole{std,std-ref}{{[}1{]}}}}}.
\end{quote}

\begin{sphinxVerbatim}[commandchars=\\\{\}]
\PYG{k+kt}{REAL}\PYG{p}{(}\PYG{l+m+mi}{8}\PYG{p}{)}\PYG{p}{,}\PYG{k}{INTENT}\PYG{p}{(}\PYG{n}{IN}\PYG{p}{)}\PYG{+w}{ }\PYG{k+kd}{::}\PYG{+w}{ }\PYG{n}{bvec}\PYG{p}{(}\PYG{l+m+mi}{3}\PYG{p}{,}\PYG{l+m+mi}{3}\PYG{p}{)}
\end{sphinxVerbatim}
\begin{quote}

\sphinxAtStartPar
Reciprocal lattice vectors (arbitrary unit).
Because they are used to choose the direction of tetrahedra,
only their ratio is used.
\end{quote}

\begin{sphinxVerbatim}[commandchars=\\\{\}]
\PYG{k+kt}{INTEGER}\PYG{p}{,}\PYG{k}{INTENT}\PYG{p}{(}\PYG{n}{IN}\PYG{p}{)}\PYG{+w}{ }\PYG{k+kd}{::}\PYG{+w}{ }\PYG{n}{nb}
\end{sphinxVerbatim}
\begin{quote}

\sphinxAtStartPar
The number of bands.
\end{quote}

\begin{sphinxVerbatim}[commandchars=\\\{\}]
\PYG{k+kt}{INTEGER}\PYG{p}{,}\PYG{k}{INTENT}\PYG{p}{(}\PYG{n}{IN}\PYG{p}{)}\PYG{+w}{ }\PYG{k+kd}{::}\PYG{+w}{ }\PYG{n}{nge}\PYG{p}{(}\PYG{l+m+mi}{3}\PYG{p}{)}
\end{sphinxVerbatim}
\begin{quote}

\sphinxAtStartPar
Specify the \(k\)\sphinxhyphen{}grid
for input orbital energy.
\end{quote}

\begin{sphinxVerbatim}[commandchars=\\\{\}]
\PYG{k+kt}{REAL}\PYG{p}{(}\PYG{l+m+mi}{8}\PYG{p}{)}\PYG{p}{,}\PYG{k}{INTENT}\PYG{p}{(}\PYG{n}{IN}\PYG{p}{)}\PYG{+w}{ }\PYG{k+kd}{::}\PYG{+w}{ }\PYG{n}{eig1}\PYG{p}{(}\PYG{n}{nb}\PYG{p}{,}\PYG{n}{nge}\PYG{p}{(}\PYG{l+m+mi}{1}\PYG{p}{)}\PYG{p}{,}\PYG{n}{nge}\PYG{p}{(}\PYG{l+m+mi}{2}\PYG{p}{)}\PYG{p}{,}\PYG{n}{nge}\PYG{p}{(}\PYG{l+m+mi}{3}\PYG{p}{)}\PYG{p}{)}
\end{sphinxVerbatim}
\begin{quote}

\sphinxAtStartPar
The orbital energy measured from the Fermi energy
( \(\varepsilon_{\rm F} = 0\) ).
Do the same with \sphinxcode{\sphinxupquote{eig2}}.
\end{quote}

\begin{sphinxVerbatim}[commandchars=\\\{\}]
\PYG{k+kt}{REAL}\PYG{p}{(}\PYG{l+m+mi}{8}\PYG{p}{)}\PYG{p}{,}\PYG{k}{INTENT}\PYG{p}{(}\PYG{n}{IN}\PYG{p}{)}\PYG{+w}{ }\PYG{k+kd}{::}\PYG{+w}{ }\PYG{n}{eig2}\PYG{p}{(}\PYG{n}{nb}\PYG{p}{,}\PYG{n}{nge}\PYG{p}{(}\PYG{l+m+mi}{1}\PYG{p}{)}\PYG{p}{,}\PYG{n}{nge}\PYG{p}{(}\PYG{l+m+mi}{2}\PYG{p}{)}\PYG{p}{,}\PYG{n}{nge}\PYG{p}{(}\PYG{l+m+mi}{3}\PYG{p}{)}\PYG{p}{)}
\end{sphinxVerbatim}
\begin{quote}

\sphinxAtStartPar
Another orbital energy.
E.g. \(\varepsilon_{k + q}\) on a shifted grid.
\end{quote}

\begin{sphinxVerbatim}[commandchars=\\\{\}]
\PYG{k+kt}{INTEGER}\PYG{p}{,}\PYG{k}{INTENT}\PYG{p}{(}\PYG{n}{IN}\PYG{p}{)}\PYG{+w}{ }\PYG{k+kd}{::}\PYG{+w}{ }\PYG{n}{ngw}\PYG{p}{(}\PYG{l+m+mi}{3}\PYG{p}{)}
\end{sphinxVerbatim}
\begin{quote}

\sphinxAtStartPar
Specify the \(k\)\sphinxhyphen{}grid for output integration weights.
You can make \sphinxcode{\sphinxupquote{ngw}} \(\neq\) \sphinxcode{\sphinxupquote{nge}} (See {\hyperref[\detokenize{app:app}]{\sphinxcrossref{\DUrole{std,std-ref}{Appendix}}}}).
\end{quote}

\begin{sphinxVerbatim}[commandchars=\\\{\}]
\PYG{k+kt}{REAL}\PYG{p}{(}\PYG{l+m+mi}{8}\PYG{p}{)}\PYG{p}{,}\PYG{k}{INTENT}\PYG{p}{(}\PYG{n}{OUT}\PYG{p}{)}\PYG{+w}{ }\PYG{k+kd}{::}\PYG{+w}{ }\PYG{n}{wght}\PYG{p}{(}\PYG{n}{nb}\PYG{p}{,}\PYG{n}{nb}\PYG{p}{,}\PYG{n}{ngw}\PYG{p}{(}\PYG{l+m+mi}{1}\PYG{p}{)}\PYG{p}{,}\PYG{n}{ngw}\PYG{p}{(}\PYG{l+m+mi}{2}\PYG{p}{)}\PYG{p}{,}\PYG{n}{ngw}\PYG{p}{(}\PYG{l+m+mi}{3}\PYG{p}{)}\PYG{p}{)}
\end{sphinxVerbatim}
\begin{quote}

\sphinxAtStartPar
The integration weights.
\end{quote}

\begin{sphinxVerbatim}[commandchars=\\\{\}]
\PYG{k+kt}{INTEGER}\PYG{p}{,}\PYG{k}{INTENT}\PYG{p}{(}\PYG{n}{IN}\PYG{p}{)}\PYG{p}{,}\PYG{k}{OPTIONAL}\PYG{+w}{ }\PYG{k+kd}{::}\PYG{+w}{ }\PYG{n}{comm}
\end{sphinxVerbatim}
\begin{quote}

\sphinxAtStartPar
Optional argument. Communicators for MPI such as \sphinxcode{\sphinxupquote{MPI\_COMM\_WORLD}}.
Only for MPI / Hybrid parallelization.
For C compiler without MPI, just pass \sphinxcode{\sphinxupquote{NULL}} to omit this argment.
\end{quote}
\end{quote}


\section{A part of DFPT calculation}
\label{\detokenize{routine:a-part-of-dfpt-calculation}}\begin{equation*}
\begin{split}\begin{align}
\sum_{n n'}
\int_{\rm BZ} \frac{d^3 k}{V_{\rm BZ}}
\theta(\varepsilon_{\rm F} -
\varepsilon_{n k}) \theta(\varepsilon_{n k} - \varepsilon'_{n' k})
X_{n n' k}
\end{align}\end{split}
\end{equation*}
\begin{sphinxVerbatim}[commandchars=\\\{\}]
\PYG{k}{CALL }\PYG{n}{libtetrabz\PYGZus{}dblstep}\PYG{p}{(}\PYG{n}{ltetra}\PYG{p}{,}\PYG{n}{bvec}\PYG{p}{,}\PYG{n}{nb}\PYG{p}{,}\PYG{n}{nge}\PYG{p}{,}\PYG{n}{eig1}\PYG{p}{,}\PYG{n}{eig2}\PYG{p}{,}\PYG{n}{ngw}\PYG{p}{,}\PYG{n}{wght}\PYG{p}{,}\PYG{n}{comm}\PYG{p}{)}
\end{sphinxVerbatim}

\sphinxAtStartPar
Parameters
\begin{quote}

\begin{sphinxVerbatim}[commandchars=\\\{\}]
\PYG{k+kt}{INTEGER}\PYG{p}{,}\PYG{k}{INTENT}\PYG{p}{(}\PYG{n}{IN}\PYG{p}{)}\PYG{+w}{ }\PYG{k+kd}{::}\PYG{+w}{ }\PYG{n}{ltetra}
\end{sphinxVerbatim}
\begin{quote}

\sphinxAtStartPar
Specify the type of the tetrahedron method.
1 \(\cdots\) the linear tetrahedron method.
2 \(\cdots\) the optimized tetrahedron method {\hyperref[\detokenize{ref:ref}]{\sphinxcrossref{\DUrole{std,std-ref}{{[}1{]}}}}}.
\end{quote}

\begin{sphinxVerbatim}[commandchars=\\\{\}]
\PYG{k+kt}{REAL}\PYG{p}{(}\PYG{l+m+mi}{8}\PYG{p}{)}\PYG{p}{,}\PYG{k}{INTENT}\PYG{p}{(}\PYG{n}{IN}\PYG{p}{)}\PYG{+w}{ }\PYG{k+kd}{::}\PYG{+w}{ }\PYG{n}{bvec}\PYG{p}{(}\PYG{l+m+mi}{3}\PYG{p}{,}\PYG{l+m+mi}{3}\PYG{p}{)}
\end{sphinxVerbatim}
\begin{quote}

\sphinxAtStartPar
Reciprocal lattice vectors (arbitrary unit).
Because they are used to choose the direction of tetrahedra,
only their ratio is used.
\end{quote}

\begin{sphinxVerbatim}[commandchars=\\\{\}]
\PYG{k+kt}{INTEGER}\PYG{p}{,}\PYG{k}{INTENT}\PYG{p}{(}\PYG{n}{IN}\PYG{p}{)}\PYG{+w}{ }\PYG{k+kd}{::}\PYG{+w}{ }\PYG{n}{nb}
\end{sphinxVerbatim}
\begin{quote}

\sphinxAtStartPar
The number of bands.
\end{quote}

\begin{sphinxVerbatim}[commandchars=\\\{\}]
\PYG{k+kt}{INTEGER}\PYG{p}{,}\PYG{k}{INTENT}\PYG{p}{(}\PYG{n}{IN}\PYG{p}{)}\PYG{+w}{ }\PYG{k+kd}{::}\PYG{+w}{ }\PYG{n}{nge}\PYG{p}{(}\PYG{l+m+mi}{3}\PYG{p}{)}
\end{sphinxVerbatim}
\begin{quote}

\sphinxAtStartPar
Specify the \(k\)\sphinxhyphen{}grid
for input orbital energy.
\end{quote}

\begin{sphinxVerbatim}[commandchars=\\\{\}]
\PYG{k+kt}{REAL}\PYG{p}{(}\PYG{l+m+mi}{8}\PYG{p}{)}\PYG{p}{,}\PYG{k}{INTENT}\PYG{p}{(}\PYG{n}{IN}\PYG{p}{)}\PYG{+w}{ }\PYG{k+kd}{::}\PYG{+w}{ }\PYG{n}{eig1}\PYG{p}{(}\PYG{n}{nb}\PYG{p}{,}\PYG{n}{nge}\PYG{p}{(}\PYG{l+m+mi}{1}\PYG{p}{)}\PYG{p}{,}\PYG{n}{nge}\PYG{p}{(}\PYG{l+m+mi}{2}\PYG{p}{)}\PYG{p}{,}\PYG{n}{nge}\PYG{p}{(}\PYG{l+m+mi}{3}\PYG{p}{)}\PYG{p}{)}
\end{sphinxVerbatim}
\begin{quote}

\sphinxAtStartPar
The orbital energy measured from the Fermi energy
( \(\varepsilon_{\rm F} = 0\) ).
Do the same with \sphinxcode{\sphinxupquote{eig2}}.
\end{quote}

\begin{sphinxVerbatim}[commandchars=\\\{\}]
\PYG{k+kt}{REAL}\PYG{p}{(}\PYG{l+m+mi}{8}\PYG{p}{)}\PYG{p}{,}\PYG{k}{INTENT}\PYG{p}{(}\PYG{n}{IN}\PYG{p}{)}\PYG{+w}{ }\PYG{k+kd}{::}\PYG{+w}{ }\PYG{n}{eig2}\PYG{p}{(}\PYG{n}{nb}\PYG{p}{,}\PYG{n}{nge}\PYG{p}{(}\PYG{l+m+mi}{1}\PYG{p}{)}\PYG{p}{,}\PYG{n}{nge}\PYG{p}{(}\PYG{l+m+mi}{2}\PYG{p}{)}\PYG{p}{,}\PYG{n}{nge}\PYG{p}{(}\PYG{l+m+mi}{3}\PYG{p}{)}\PYG{p}{)}
\end{sphinxVerbatim}
\begin{quote}

\sphinxAtStartPar
Another orbital energy.
E.g. \(\varepsilon_{k + q}\) on a shifted grid.
\end{quote}

\begin{sphinxVerbatim}[commandchars=\\\{\}]
\PYG{k+kt}{INTEGER}\PYG{p}{,}\PYG{k}{INTENT}\PYG{p}{(}\PYG{n}{IN}\PYG{p}{)}\PYG{+w}{ }\PYG{k+kd}{::}\PYG{+w}{ }\PYG{n}{ngw}\PYG{p}{(}\PYG{l+m+mi}{3}\PYG{p}{)}
\end{sphinxVerbatim}
\begin{quote}

\sphinxAtStartPar
Specify the \(k\)\sphinxhyphen{}grid for output integration weights.
You can make \sphinxcode{\sphinxupquote{ngw}} \(\neq\) \sphinxcode{\sphinxupquote{nge}} (See {\hyperref[\detokenize{app:app}]{\sphinxcrossref{\DUrole{std,std-ref}{Appendix}}}}).
\end{quote}

\begin{sphinxVerbatim}[commandchars=\\\{\}]
\PYG{k+kt}{REAL}\PYG{p}{(}\PYG{l+m+mi}{8}\PYG{p}{)}\PYG{p}{,}\PYG{k}{INTENT}\PYG{p}{(}\PYG{n}{OUT}\PYG{p}{)}\PYG{+w}{ }\PYG{k+kd}{::}\PYG{+w}{ }\PYG{n}{wght}\PYG{p}{(}\PYG{n}{nb}\PYG{p}{,}\PYG{n}{nb}\PYG{p}{,}\PYG{n}{ngw}\PYG{p}{(}\PYG{l+m+mi}{1}\PYG{p}{)}\PYG{p}{,}\PYG{n}{ngw}\PYG{p}{(}\PYG{l+m+mi}{2}\PYG{p}{)}\PYG{p}{,}\PYG{n}{ngw}\PYG{p}{(}\PYG{l+m+mi}{3}\PYG{p}{)}\PYG{p}{)}
\end{sphinxVerbatim}
\begin{quote}

\sphinxAtStartPar
The integration weights.
\end{quote}

\begin{sphinxVerbatim}[commandchars=\\\{\}]
\PYG{k+kt}{INTEGER}\PYG{p}{,}\PYG{k}{INTENT}\PYG{p}{(}\PYG{n}{IN}\PYG{p}{)}\PYG{p}{,}\PYG{k}{OPTIONAL}\PYG{+w}{ }\PYG{k+kd}{::}\PYG{+w}{ }\PYG{n}{comm}
\end{sphinxVerbatim}
\begin{quote}

\sphinxAtStartPar
Optional argument. Communicators for MPI such as \sphinxcode{\sphinxupquote{MPI\_COMM\_WORLD}}.
Only for MPI / Hybrid parallelization.
For C compiler without MPI, just pass \sphinxcode{\sphinxupquote{NULL}} to omit this argment.
\end{quote}
\end{quote}


\section{Static polarization function}
\label{\detokenize{routine:static-polarization-function}}\begin{equation*}
\begin{split}\begin{align}
\sum_{n n'}
\int_{\rm BZ} \frac{d^3 k}{V_{\rm BZ}}
\frac{\theta(\varepsilon_{\rm F} - \varepsilon_{n k})
\theta(\varepsilon'_{n' k} - \varepsilon_{\rm F})}
{\varepsilon'_{n' k} - \varepsilon_{n k}}
X_{n n' k}
\end{align}\end{split}
\end{equation*}
\begin{sphinxVerbatim}[commandchars=\\\{\}]
\PYG{k}{CALL }\PYG{n}{libtetrabz\PYGZus{}polstat}\PYG{p}{(}\PYG{n}{ltetra}\PYG{p}{,}\PYG{n}{bvec}\PYG{p}{,}\PYG{n}{nb}\PYG{p}{,}\PYG{n}{nge}\PYG{p}{,}\PYG{n}{eig1}\PYG{p}{,}\PYG{n}{eig2}\PYG{p}{,}\PYG{n}{ngw}\PYG{p}{,}\PYG{n}{wght}\PYG{p}{,}\PYG{n}{comm}\PYG{p}{)}
\end{sphinxVerbatim}

\sphinxAtStartPar
Parameters
\begin{quote}

\begin{sphinxVerbatim}[commandchars=\\\{\}]
\PYG{k+kt}{INTEGER}\PYG{p}{,}\PYG{k}{INTENT}\PYG{p}{(}\PYG{n}{IN}\PYG{p}{)}\PYG{+w}{ }\PYG{k+kd}{::}\PYG{+w}{ }\PYG{n}{ltetra}
\end{sphinxVerbatim}
\begin{quote}

\sphinxAtStartPar
Specify the type of the tetrahedron method.
1 \(\cdots\) the linear tetrahedron method.
2 \(\cdots\) the optimized tetrahedron method {\hyperref[\detokenize{ref:ref}]{\sphinxcrossref{\DUrole{std,std-ref}{{[}1{]}}}}}.
\end{quote}

\begin{sphinxVerbatim}[commandchars=\\\{\}]
\PYG{k+kt}{REAL}\PYG{p}{(}\PYG{l+m+mi}{8}\PYG{p}{)}\PYG{p}{,}\PYG{k}{INTENT}\PYG{p}{(}\PYG{n}{IN}\PYG{p}{)}\PYG{+w}{ }\PYG{k+kd}{::}\PYG{+w}{ }\PYG{n}{bvec}\PYG{p}{(}\PYG{l+m+mi}{3}\PYG{p}{,}\PYG{l+m+mi}{3}\PYG{p}{)}
\end{sphinxVerbatim}
\begin{quote}

\sphinxAtStartPar
Reciprocal lattice vectors (arbitrary unit).
Because they are used to choose the direction of tetrahedra,
only their ratio is used.
\end{quote}

\begin{sphinxVerbatim}[commandchars=\\\{\}]
\PYG{k+kt}{INTEGER}\PYG{p}{,}\PYG{k}{INTENT}\PYG{p}{(}\PYG{n}{IN}\PYG{p}{)}\PYG{+w}{ }\PYG{k+kd}{::}\PYG{+w}{ }\PYG{n}{nb}
\end{sphinxVerbatim}
\begin{quote}

\sphinxAtStartPar
The number of bands.
\end{quote}

\begin{sphinxVerbatim}[commandchars=\\\{\}]
\PYG{k+kt}{INTEGER}\PYG{p}{,}\PYG{k}{INTENT}\PYG{p}{(}\PYG{n}{IN}\PYG{p}{)}\PYG{+w}{ }\PYG{k+kd}{::}\PYG{+w}{ }\PYG{n}{nge}\PYG{p}{(}\PYG{l+m+mi}{3}\PYG{p}{)}
\end{sphinxVerbatim}
\begin{quote}

\sphinxAtStartPar
Specify the \(k\)\sphinxhyphen{}grid
for input orbital energy.
\end{quote}

\begin{sphinxVerbatim}[commandchars=\\\{\}]
\PYG{k+kt}{REAL}\PYG{p}{(}\PYG{l+m+mi}{8}\PYG{p}{)}\PYG{p}{,}\PYG{k}{INTENT}\PYG{p}{(}\PYG{n}{IN}\PYG{p}{)}\PYG{+w}{ }\PYG{k+kd}{::}\PYG{+w}{ }\PYG{n}{eig1}\PYG{p}{(}\PYG{n}{nb}\PYG{p}{,}\PYG{n}{nge}\PYG{p}{(}\PYG{l+m+mi}{1}\PYG{p}{)}\PYG{p}{,}\PYG{n}{nge}\PYG{p}{(}\PYG{l+m+mi}{2}\PYG{p}{)}\PYG{p}{,}\PYG{n}{nge}\PYG{p}{(}\PYG{l+m+mi}{3}\PYG{p}{)}\PYG{p}{)}
\end{sphinxVerbatim}
\begin{quote}

\sphinxAtStartPar
The orbital energy measured from the Fermi energy
( \(\varepsilon_{\rm F} = 0\) ).
Do the same with \sphinxcode{\sphinxupquote{eig2}}.
\end{quote}

\begin{sphinxVerbatim}[commandchars=\\\{\}]
\PYG{k+kt}{REAL}\PYG{p}{(}\PYG{l+m+mi}{8}\PYG{p}{)}\PYG{p}{,}\PYG{k}{INTENT}\PYG{p}{(}\PYG{n}{IN}\PYG{p}{)}\PYG{+w}{ }\PYG{k+kd}{::}\PYG{+w}{ }\PYG{n}{eig2}\PYG{p}{(}\PYG{n}{nb}\PYG{p}{,}\PYG{n}{nge}\PYG{p}{(}\PYG{l+m+mi}{1}\PYG{p}{)}\PYG{p}{,}\PYG{n}{nge}\PYG{p}{(}\PYG{l+m+mi}{2}\PYG{p}{)}\PYG{p}{,}\PYG{n}{nge}\PYG{p}{(}\PYG{l+m+mi}{3}\PYG{p}{)}\PYG{p}{)}
\end{sphinxVerbatim}
\begin{quote}

\sphinxAtStartPar
Another orbital energy.
E.g. \(\varepsilon_{k + q}\) on a shifted grid.
\end{quote}

\begin{sphinxVerbatim}[commandchars=\\\{\}]
\PYG{k+kt}{INTEGER}\PYG{p}{,}\PYG{k}{INTENT}\PYG{p}{(}\PYG{n}{IN}\PYG{p}{)}\PYG{+w}{ }\PYG{k+kd}{::}\PYG{+w}{ }\PYG{n}{ngw}\PYG{p}{(}\PYG{l+m+mi}{3}\PYG{p}{)}
\end{sphinxVerbatim}
\begin{quote}

\sphinxAtStartPar
Specify the \(k\)\sphinxhyphen{}grid for output integration weights.
You can make \sphinxcode{\sphinxupquote{ngw}} \(\neq\) \sphinxcode{\sphinxupquote{nge}} (See {\hyperref[\detokenize{app:app}]{\sphinxcrossref{\DUrole{std,std-ref}{Appendix}}}}).
\end{quote}

\begin{sphinxVerbatim}[commandchars=\\\{\}]
\PYG{k+kt}{REAL}\PYG{p}{(}\PYG{l+m+mi}{8}\PYG{p}{)}\PYG{p}{,}\PYG{k}{INTENT}\PYG{p}{(}\PYG{n}{OUT}\PYG{p}{)}\PYG{+w}{ }\PYG{k+kd}{::}\PYG{+w}{ }\PYG{n}{wght}\PYG{p}{(}\PYG{n}{nb}\PYG{p}{,}\PYG{n}{nb}\PYG{p}{,}\PYG{n}{ngw}\PYG{p}{(}\PYG{l+m+mi}{1}\PYG{p}{)}\PYG{p}{,}\PYG{n}{ngw}\PYG{p}{(}\PYG{l+m+mi}{2}\PYG{p}{)}\PYG{p}{,}\PYG{n}{ngw}\PYG{p}{(}\PYG{l+m+mi}{3}\PYG{p}{)}\PYG{p}{)}
\end{sphinxVerbatim}
\begin{quote}

\sphinxAtStartPar
The integration weights.
\end{quote}

\begin{sphinxVerbatim}[commandchars=\\\{\}]
\PYG{k+kt}{INTEGER}\PYG{p}{,}\PYG{k}{INTENT}\PYG{p}{(}\PYG{n}{IN}\PYG{p}{)}\PYG{p}{,}\PYG{k}{OPTIONAL}\PYG{+w}{ }\PYG{k+kd}{::}\PYG{+w}{ }\PYG{n}{comm}
\end{sphinxVerbatim}
\begin{quote}

\sphinxAtStartPar
Optional argument. Communicators for MPI such as \sphinxcode{\sphinxupquote{MPI\_COMM\_WORLD}}.
Only for MPI / Hybrid parallelization.
For C compiler without MPI, just pass \sphinxcode{\sphinxupquote{NULL}} to omit this argment.
\end{quote}
\end{quote}


\section{Phonon linewidth}
\label{\detokenize{routine:phonon-linewidth}}\begin{equation*}
\begin{split}\begin{align}
\sum_{n n'}
\int_{\rm BZ} \frac{d^3 k}{V_{\rm BZ}}
\theta(\varepsilon_{\rm F} -
\varepsilon_{n k}) \theta(\varepsilon'_{n' k} - \varepsilon_{\rm F})
\delta(\varepsilon'_{n' k} - \varepsilon_{n k} - \omega)
X_{n n' k}(\omega)
\end{align}\end{split}
\end{equation*}
\begin{sphinxVerbatim}[commandchars=\\\{\}]
\PYG{k}{CALL }\PYG{n}{libtetrabz\PYGZus{}fermigr}\PYG{p}{(}\PYG{n}{ltetra}\PYG{p}{,}\PYG{n}{bvec}\PYG{p}{,}\PYG{n}{nb}\PYG{p}{,}\PYG{n}{nge}\PYG{p}{,}\PYG{n}{eig1}\PYG{p}{,}\PYG{n}{eig2}\PYG{p}{,}\PYG{n}{ngw}\PYG{p}{,}\PYG{n}{wght}\PYG{p}{,}\PYG{n}{ne}\PYG{p}{,}\PYG{n}{e0}\PYG{p}{,}\PYG{n}{comm}\PYG{p}{)}
\end{sphinxVerbatim}

\sphinxAtStartPar
Parameters
\begin{quote}

\begin{sphinxVerbatim}[commandchars=\\\{\}]
\PYG{k+kt}{INTEGER}\PYG{p}{,}\PYG{k}{INTENT}\PYG{p}{(}\PYG{n}{IN}\PYG{p}{)}\PYG{+w}{ }\PYG{k+kd}{::}\PYG{+w}{ }\PYG{n}{ltetra}
\end{sphinxVerbatim}
\begin{quote}

\sphinxAtStartPar
Specify the type of the tetrahedron method.
1 \(\cdots\) the linear tetrahedron method.
2 \(\cdots\) the optimized tetrahedron method {\hyperref[\detokenize{ref:ref}]{\sphinxcrossref{\DUrole{std,std-ref}{{[}1{]}}}}}.
\end{quote}

\begin{sphinxVerbatim}[commandchars=\\\{\}]
\PYG{k+kt}{REAL}\PYG{p}{(}\PYG{l+m+mi}{8}\PYG{p}{)}\PYG{p}{,}\PYG{k}{INTENT}\PYG{p}{(}\PYG{n}{IN}\PYG{p}{)}\PYG{+w}{ }\PYG{k+kd}{::}\PYG{+w}{ }\PYG{n}{bvec}\PYG{p}{(}\PYG{l+m+mi}{3}\PYG{p}{,}\PYG{l+m+mi}{3}\PYG{p}{)}
\end{sphinxVerbatim}
\begin{quote}

\sphinxAtStartPar
Reciprocal lattice vectors (arbitrary unit).
Because they are used to choose the direction of tetrahedra,
only their ratio is used.
\end{quote}

\begin{sphinxVerbatim}[commandchars=\\\{\}]
\PYG{k+kt}{INTEGER}\PYG{p}{,}\PYG{k}{INTENT}\PYG{p}{(}\PYG{n}{IN}\PYG{p}{)}\PYG{+w}{ }\PYG{k+kd}{::}\PYG{+w}{ }\PYG{n}{nb}
\end{sphinxVerbatim}
\begin{quote}

\sphinxAtStartPar
The number of bands.
\end{quote}

\begin{sphinxVerbatim}[commandchars=\\\{\}]
\PYG{k+kt}{INTEGER}\PYG{p}{,}\PYG{k}{INTENT}\PYG{p}{(}\PYG{n}{IN}\PYG{p}{)}\PYG{+w}{ }\PYG{k+kd}{::}\PYG{+w}{ }\PYG{n}{nge}\PYG{p}{(}\PYG{l+m+mi}{3}\PYG{p}{)}
\end{sphinxVerbatim}
\begin{quote}

\sphinxAtStartPar
Specify the \(k\)\sphinxhyphen{}grid
for input orbital energy.
\end{quote}

\begin{sphinxVerbatim}[commandchars=\\\{\}]
\PYG{k+kt}{REAL}\PYG{p}{(}\PYG{l+m+mi}{8}\PYG{p}{)}\PYG{p}{,}\PYG{k}{INTENT}\PYG{p}{(}\PYG{n}{IN}\PYG{p}{)}\PYG{+w}{ }\PYG{k+kd}{::}\PYG{+w}{ }\PYG{n}{eig1}\PYG{p}{(}\PYG{n}{nb}\PYG{p}{,}\PYG{n}{nge}\PYG{p}{(}\PYG{l+m+mi}{1}\PYG{p}{)}\PYG{p}{,}\PYG{n}{nge}\PYG{p}{(}\PYG{l+m+mi}{2}\PYG{p}{)}\PYG{p}{,}\PYG{n}{nge}\PYG{p}{(}\PYG{l+m+mi}{3}\PYG{p}{)}\PYG{p}{)}
\end{sphinxVerbatim}
\begin{quote}

\sphinxAtStartPar
The orbital energy measured from the Fermi energy
( \(\varepsilon_{\rm F} = 0\) ).
Do the same with \sphinxcode{\sphinxupquote{eig2}}.
\end{quote}

\begin{sphinxVerbatim}[commandchars=\\\{\}]
\PYG{k+kt}{REAL}\PYG{p}{(}\PYG{l+m+mi}{8}\PYG{p}{)}\PYG{p}{,}\PYG{k}{INTENT}\PYG{p}{(}\PYG{n}{IN}\PYG{p}{)}\PYG{+w}{ }\PYG{k+kd}{::}\PYG{+w}{ }\PYG{n}{eig2}\PYG{p}{(}\PYG{n}{nb}\PYG{p}{,}\PYG{n}{nge}\PYG{p}{(}\PYG{l+m+mi}{1}\PYG{p}{)}\PYG{p}{,}\PYG{n}{nge}\PYG{p}{(}\PYG{l+m+mi}{2}\PYG{p}{)}\PYG{p}{,}\PYG{n}{nge}\PYG{p}{(}\PYG{l+m+mi}{3}\PYG{p}{)}\PYG{p}{)}
\end{sphinxVerbatim}
\begin{quote}

\sphinxAtStartPar
Another orbital energy.
E.g. \(\varepsilon_{k + q}\) on a shifted grid.
\end{quote}

\begin{sphinxVerbatim}[commandchars=\\\{\}]
\PYG{k+kt}{INTEGER}\PYG{p}{,}\PYG{k}{INTENT}\PYG{p}{(}\PYG{n}{IN}\PYG{p}{)}\PYG{+w}{ }\PYG{k+kd}{::}\PYG{+w}{ }\PYG{n}{ngw}\PYG{p}{(}\PYG{l+m+mi}{3}\PYG{p}{)}
\end{sphinxVerbatim}
\begin{quote}

\sphinxAtStartPar
Specify the \(k\)\sphinxhyphen{}grid for output integration weights.
You can make \sphinxcode{\sphinxupquote{ngw}} \(\neq\) \sphinxcode{\sphinxupquote{nge}} (See {\hyperref[\detokenize{app:app}]{\sphinxcrossref{\DUrole{std,std-ref}{Appendix}}}}).
\end{quote}

\begin{sphinxVerbatim}[commandchars=\\\{\}]
\PYG{k+kt}{REAL}\PYG{p}{(}\PYG{l+m+mi}{8}\PYG{p}{)}\PYG{p}{,}\PYG{k}{INTENT}\PYG{p}{(}\PYG{n}{OUT}\PYG{p}{)}\PYG{+w}{ }\PYG{k+kd}{::}\PYG{+w}{ }\PYG{n}{wght}\PYG{p}{(}\PYG{n}{ne}\PYG{p}{,}\PYG{n}{nb}\PYG{p}{,}\PYG{n}{nb}\PYG{p}{,}\PYG{n}{ngw}\PYG{p}{(}\PYG{l+m+mi}{1}\PYG{p}{)}\PYG{p}{,}\PYG{n}{ngw}\PYG{p}{(}\PYG{l+m+mi}{2}\PYG{p}{)}\PYG{p}{,}\PYG{n}{ngw}\PYG{p}{(}\PYG{l+m+mi}{3}\PYG{p}{)}\PYG{p}{)}
\end{sphinxVerbatim}
\begin{quote}

\sphinxAtStartPar
The integration weights.
\end{quote}

\begin{sphinxVerbatim}[commandchars=\\\{\}]
\PYG{k+kt}{INTEGER}\PYG{p}{,}\PYG{k}{INTENT}\PYG{p}{(}\PYG{n}{IN}\PYG{p}{)}\PYG{+w}{ }\PYG{k+kd}{::}\PYG{+w}{ }\PYG{n}{ne}
\end{sphinxVerbatim}
\begin{quote}

\sphinxAtStartPar
The number of branches of the phonon.
\end{quote}

\begin{sphinxVerbatim}[commandchars=\\\{\}]
\PYG{k+kt}{REAL}\PYG{p}{(}\PYG{l+m+mi}{8}\PYG{p}{)}\PYG{p}{,}\PYG{k}{INTENT}\PYG{p}{(}\PYG{n}{IN}\PYG{p}{)}\PYG{+w}{ }\PYG{k+kd}{::}\PYG{+w}{ }\PYG{n}{e0}\PYG{p}{(}\PYG{n}{ne}\PYG{p}{)}
\end{sphinxVerbatim}
\begin{quote}

\sphinxAtStartPar
Phonon frequencies.
\end{quote}

\begin{sphinxVerbatim}[commandchars=\\\{\}]
\PYG{k+kt}{INTEGER}\PYG{p}{,}\PYG{k}{INTENT}\PYG{p}{(}\PYG{n}{IN}\PYG{p}{)}\PYG{p}{,}\PYG{k}{OPTIONAL}\PYG{+w}{ }\PYG{k+kd}{::}\PYG{+w}{ }\PYG{n}{comm}
\end{sphinxVerbatim}
\begin{quote}

\sphinxAtStartPar
Optional argument. Communicators for MPI such as \sphinxcode{\sphinxupquote{MPI\_COMM\_WORLD}}.
Only for MPI / Hybrid parallelization.
For C compiler without MPI, just pass \sphinxcode{\sphinxupquote{NULL}} to omit this argment.
\end{quote}
\end{quote}


\section{Polarization function (complex frequency)}
\label{\detokenize{routine:polarization-function-complex-frequency}}\begin{equation*}
\begin{split}\begin{align}
\sum_{n n'}
\int_{\rm BZ} \frac{d^3 k}{V_{\rm BZ}}
\frac{\theta(\varepsilon_{\rm F} - \varepsilon_{n k})
\theta(\varepsilon'_{n' k} - \varepsilon_{\rm F})}
{\varepsilon'_{n' k} - \varepsilon_{n k} + i \omega}
X_{n n' k}(\omega)
\end{align}\end{split}
\end{equation*}
\begin{sphinxVerbatim}[commandchars=\\\{\}]
\PYG{k}{CALL }\PYG{n}{libtetrabz\PYGZus{}polcmplx}\PYG{p}{(}\PYG{n}{ltetra}\PYG{p}{,}\PYG{n}{bvec}\PYG{p}{,}\PYG{n}{nb}\PYG{p}{,}\PYG{n}{nge}\PYG{p}{,}\PYG{n}{eig1}\PYG{p}{,}\PYG{n}{eig2}\PYG{p}{,}\PYG{n}{ngw}\PYG{p}{,}\PYG{n}{wght}\PYG{p}{,}\PYG{n}{ne}\PYG{p}{,}\PYG{n}{e0}\PYG{p}{,}\PYG{n}{comm}\PYG{p}{)}
\end{sphinxVerbatim}

\sphinxAtStartPar
Parameters
\begin{quote}

\begin{sphinxVerbatim}[commandchars=\\\{\}]
\PYG{k+kt}{INTEGER}\PYG{p}{,}\PYG{k}{INTENT}\PYG{p}{(}\PYG{n}{IN}\PYG{p}{)}\PYG{+w}{ }\PYG{k+kd}{::}\PYG{+w}{ }\PYG{n}{ltetra}
\end{sphinxVerbatim}
\begin{quote}

\sphinxAtStartPar
Specify the type of the tetrahedron method.
1 \(\cdots\) the linear tetrahedron method.
2 \(\cdots\) the optimized tetrahedron method {\hyperref[\detokenize{ref:ref}]{\sphinxcrossref{\DUrole{std,std-ref}{{[}1{]}}}}}.
\end{quote}

\begin{sphinxVerbatim}[commandchars=\\\{\}]
\PYG{k+kt}{REAL}\PYG{p}{(}\PYG{l+m+mi}{8}\PYG{p}{)}\PYG{p}{,}\PYG{k}{INTENT}\PYG{p}{(}\PYG{n}{IN}\PYG{p}{)}\PYG{+w}{ }\PYG{k+kd}{::}\PYG{+w}{ }\PYG{n}{bvec}\PYG{p}{(}\PYG{l+m+mi}{3}\PYG{p}{,}\PYG{l+m+mi}{3}\PYG{p}{)}
\end{sphinxVerbatim}
\begin{quote}

\sphinxAtStartPar
Reciprocal lattice vectors (arbitrary unit).
Because they are used to choose the direction of tetrahedra,
only their ratio is used.
\end{quote}

\begin{sphinxVerbatim}[commandchars=\\\{\}]
\PYG{k+kt}{INTEGER}\PYG{p}{,}\PYG{k}{INTENT}\PYG{p}{(}\PYG{n}{IN}\PYG{p}{)}\PYG{+w}{ }\PYG{k+kd}{::}\PYG{+w}{ }\PYG{n}{nb}
\end{sphinxVerbatim}
\begin{quote}

\sphinxAtStartPar
The number of bands.
\end{quote}

\begin{sphinxVerbatim}[commandchars=\\\{\}]
\PYG{k+kt}{INTEGER}\PYG{p}{,}\PYG{k}{INTENT}\PYG{p}{(}\PYG{n}{IN}\PYG{p}{)}\PYG{+w}{ }\PYG{k+kd}{::}\PYG{+w}{ }\PYG{n}{nge}\PYG{p}{(}\PYG{l+m+mi}{3}\PYG{p}{)}
\end{sphinxVerbatim}
\begin{quote}

\sphinxAtStartPar
Specify the \(k\)\sphinxhyphen{}grid
for input orbital energy.
\end{quote}

\begin{sphinxVerbatim}[commandchars=\\\{\}]
\PYG{k+kt}{REAL}\PYG{p}{(}\PYG{l+m+mi}{8}\PYG{p}{)}\PYG{p}{,}\PYG{k}{INTENT}\PYG{p}{(}\PYG{n}{IN}\PYG{p}{)}\PYG{+w}{ }\PYG{k+kd}{::}\PYG{+w}{ }\PYG{n}{eig1}\PYG{p}{(}\PYG{n}{nb}\PYG{p}{,}\PYG{n}{nge}\PYG{p}{(}\PYG{l+m+mi}{1}\PYG{p}{)}\PYG{p}{,}\PYG{n}{nge}\PYG{p}{(}\PYG{l+m+mi}{2}\PYG{p}{)}\PYG{p}{,}\PYG{n}{nge}\PYG{p}{(}\PYG{l+m+mi}{3}\PYG{p}{)}\PYG{p}{)}
\end{sphinxVerbatim}
\begin{quote}

\sphinxAtStartPar
The orbital energy measured from the Fermi energy
( \(\varepsilon_{\rm F} = 0\) ).
Do the same with \sphinxcode{\sphinxupquote{eig2}}.
\end{quote}

\begin{sphinxVerbatim}[commandchars=\\\{\}]
\PYG{k+kt}{REAL}\PYG{p}{(}\PYG{l+m+mi}{8}\PYG{p}{)}\PYG{p}{,}\PYG{k}{INTENT}\PYG{p}{(}\PYG{n}{IN}\PYG{p}{)}\PYG{+w}{ }\PYG{k+kd}{::}\PYG{+w}{ }\PYG{n}{eig2}\PYG{p}{(}\PYG{n}{nb}\PYG{p}{,}\PYG{n}{nge}\PYG{p}{(}\PYG{l+m+mi}{1}\PYG{p}{)}\PYG{p}{,}\PYG{n}{nge}\PYG{p}{(}\PYG{l+m+mi}{2}\PYG{p}{)}\PYG{p}{,}\PYG{n}{nge}\PYG{p}{(}\PYG{l+m+mi}{3}\PYG{p}{)}\PYG{p}{)}
\end{sphinxVerbatim}
\begin{quote}

\sphinxAtStartPar
Another orbital energy.
E.g. \(\varepsilon_{k + q}\) on a shifted grid.
\end{quote}

\begin{sphinxVerbatim}[commandchars=\\\{\}]
\PYG{k+kt}{INTEGER}\PYG{p}{,}\PYG{k}{INTENT}\PYG{p}{(}\PYG{n}{IN}\PYG{p}{)}\PYG{+w}{ }\PYG{k+kd}{::}\PYG{+w}{ }\PYG{n}{ngw}\PYG{p}{(}\PYG{l+m+mi}{3}\PYG{p}{)}
\end{sphinxVerbatim}
\begin{quote}

\sphinxAtStartPar
Specify the \(k\)\sphinxhyphen{}grid for output integration weights.
You can make \sphinxcode{\sphinxupquote{ngw}} \(\neq\) \sphinxcode{\sphinxupquote{nge}} (See {\hyperref[\detokenize{app:app}]{\sphinxcrossref{\DUrole{std,std-ref}{Appendix}}}}).
\end{quote}

\begin{sphinxVerbatim}[commandchars=\\\{\}]
\PYG{k+kt}{COMPLEX}\PYG{p}{(}\PYG{l+m+mi}{8}\PYG{p}{)}\PYG{p}{,}\PYG{k}{INTENT}\PYG{p}{(}\PYG{n}{OUT}\PYG{p}{)}\PYG{+w}{ }\PYG{k+kd}{::}\PYG{+w}{ }\PYG{n}{wght}\PYG{p}{(}\PYG{n}{ne}\PYG{p}{,}\PYG{n}{nb}\PYG{p}{,}\PYG{n}{nb}\PYG{p}{,}\PYG{n}{ngw}\PYG{p}{(}\PYG{l+m+mi}{1}\PYG{p}{)}\PYG{p}{,}\PYG{n}{ngw}\PYG{p}{(}\PYG{l+m+mi}{2}\PYG{p}{)}\PYG{p}{,}\PYG{n}{ngw}\PYG{p}{(}\PYG{l+m+mi}{3}\PYG{p}{)}\PYG{p}{)}
\end{sphinxVerbatim}
\begin{quote}

\sphinxAtStartPar
The integration weights.
\end{quote}

\begin{sphinxVerbatim}[commandchars=\\\{\}]
\PYG{k+kt}{INTEGER}\PYG{p}{,}\PYG{k}{INTENT}\PYG{p}{(}\PYG{n}{IN}\PYG{p}{)}\PYG{+w}{ }\PYG{k+kd}{::}\PYG{+w}{ }\PYG{n}{ne}
\end{sphinxVerbatim}
\begin{quote}

\sphinxAtStartPar
The number of imaginary frequencies where
polarization functions are calculated.
\end{quote}

\begin{sphinxVerbatim}[commandchars=\\\{\}]
\PYG{k+kt}{COMPLEX}\PYG{p}{(}\PYG{l+m+mi}{8}\PYG{p}{)}\PYG{p}{,}\PYG{k}{INTENT}\PYG{p}{(}\PYG{n}{IN}\PYG{p}{)}\PYG{+w}{ }\PYG{k+kd}{::}\PYG{+w}{ }\PYG{n}{e0}\PYG{p}{(}\PYG{n}{ne}\PYG{p}{)}
\end{sphinxVerbatim}
\begin{quote}

\sphinxAtStartPar
Complex frequencies where
polarization functions are calculated.
\end{quote}

\begin{sphinxVerbatim}[commandchars=\\\{\}]
\PYG{k+kt}{INTEGER}\PYG{p}{,}\PYG{k}{INTENT}\PYG{p}{(}\PYG{n}{IN}\PYG{p}{)}\PYG{p}{,}\PYG{k}{OPTIONAL}\PYG{+w}{ }\PYG{k+kd}{::}\PYG{+w}{ }\PYG{n}{comm}
\end{sphinxVerbatim}
\begin{quote}

\sphinxAtStartPar
Optional argument. Communicators for MPI such as \sphinxcode{\sphinxupquote{MPI\_COMM\_WORLD}}.
Only for MPI / Hybrid parallelization.
For C compiler without MPI, just pass \sphinxcode{\sphinxupquote{NULL}} to omit this argment.
\end{quote}
\end{quote}

\sphinxstepscope


\chapter{Piece of sample code}
\label{\detokenize{sample:piece-of-sample-code}}\label{\detokenize{sample::doc}}
\sphinxAtStartPar
This sample shows the calculation of the charge density.
\begin{equation*}
\begin{split}\begin{align}
\rho(r) = 2 \sum_{n k} \theta(\varepsilon_{\rm F} - \varepsilon_{n k})
|\varphi_{n k}(r)|^2
\end{align}\end{split}
\end{equation*}
\begin{sphinxVerbatim}[commandchars=\\\{\}]
\PYG{k}{SUBROUTINE }\PYG{n}{calc\PYGZus{}rho}\PYG{p}{(}\PYG{n}{nr}\PYG{p}{,}\PYG{n}{nb}\PYG{p}{,}\PYG{n}{ng}\PYG{p}{,}\PYG{n}{nelec}\PYG{p}{,}\PYG{n}{bvec}\PYG{p}{,}\PYG{n}{eig}\PYG{p}{,}\PYG{n}{ef}\PYG{p}{,}\PYG{n}{phi}\PYG{p}{,}\PYG{n}{rho}\PYG{p}{)}
\PYG{+w}{  }\PYG{c}{!}
\PYG{+w}{  }\PYG{k}{USE }\PYG{n}{libtetrabz}\PYG{p}{,}\PYG{+w}{ }\PYG{k}{ONLY}\PYG{+w}{ }\PYG{p}{:}\PYG{+w}{ }\PYG{n}{libtetrabz\PYGZus{}fermieng}
\PYG{+w}{  }\PYG{k}{IMPLICIT }\PYG{k}{NONE}
\PYG{+w}{  }\PYG{c}{!}
\PYG{+w}{  }\PYG{k+kt}{INTEGER}\PYG{p}{,}\PYG{k}{INTENT}\PYG{p}{(}\PYG{n}{IN}\PYG{p}{)}\PYG{+w}{ }\PYG{k+kd}{::}\PYG{+w}{ }\PYG{n}{nr}\PYG{+w}{ }\PYG{c}{! number of r}
\PYG{+w}{  }\PYG{k+kt}{INTEGER}\PYG{p}{,}\PYG{k}{INTENT}\PYG{p}{(}\PYG{n}{IN}\PYG{p}{)}\PYG{+w}{ }\PYG{k+kd}{::}\PYG{+w}{ }\PYG{n}{nb}\PYG{+w}{ }\PYG{c}{! number of bands}
\PYG{+w}{  }\PYG{k+kt}{INTEGER}\PYG{p}{,}\PYG{k}{INTENT}\PYG{p}{(}\PYG{n}{IN}\PYG{p}{)}\PYG{+w}{ }\PYG{k+kd}{::}\PYG{+w}{ }\PYG{n}{ng}\PYG{p}{(}\PYG{l+m+mi}{3}\PYG{p}{)}
\PYG{+w}{  }\PYG{c}{! k\PYGZhy{}point mesh}
\PYG{+w}{  }\PYG{k+kt}{REAL}\PYG{p}{(}\PYG{l+m+mi}{8}\PYG{p}{)}\PYG{p}{,}\PYG{k}{INTENT}\PYG{p}{(}\PYG{n}{IN}\PYG{p}{)}\PYG{+w}{ }\PYG{k+kd}{::}\PYG{+w}{ }\PYG{n}{nelec}\PYG{+w}{ }\PYG{c}{! number of electrons per spin}
\PYG{+w}{  }\PYG{k+kt}{REAL}\PYG{p}{(}\PYG{l+m+mi}{8}\PYG{p}{)}\PYG{p}{,}\PYG{k}{INTENT}\PYG{p}{(}\PYG{n}{IN}\PYG{p}{)}\PYG{+w}{ }\PYG{k+kd}{::}\PYG{+w}{ }\PYG{n}{bvec}\PYG{p}{(}\PYG{l+m+mi}{3}\PYG{p}{,}\PYG{l+m+mi}{3}\PYG{p}{)}\PYG{+w}{ }\PYG{c}{! reciplocal lattice vector}
\PYG{+w}{  }\PYG{k+kt}{REAL}\PYG{p}{(}\PYG{l+m+mi}{8}\PYG{p}{)}\PYG{p}{,}\PYG{k}{INTENT}\PYG{p}{(}\PYG{n}{IN}\PYG{p}{)}\PYG{+w}{ }\PYG{k+kd}{::}\PYG{+w}{ }\PYG{n}{eig}\PYG{p}{(}\PYG{n}{nb}\PYG{p}{,}\PYG{n}{ng}\PYG{p}{(}\PYG{l+m+mi}{1}\PYG{p}{)}\PYG{p}{,}\PYG{n}{ng}\PYG{p}{(}\PYG{l+m+mi}{2}\PYG{p}{)}\PYG{p}{,}\PYG{n}{ng}\PYG{p}{(}\PYG{l+m+mi}{3}\PYG{p}{)}\PYG{p}{)}\PYG{+w}{ }\PYG{c}{! Kohn\PYGZhy{}Sham eigenvalues}
\PYG{+w}{  }\PYG{k+kt}{REAL}\PYG{p}{(}\PYG{l+m+mi}{8}\PYG{p}{)}\PYG{p}{,}\PYG{k}{INTENT}\PYG{p}{(}\PYG{n}{OUT}\PYG{p}{)}\PYG{+w}{ }\PYG{k+kd}{::}\PYG{+w}{ }\PYG{n}{ef}\PYG{+w}{ }\PYG{c}{! Fermi energy}
\PYG{+w}{  }\PYG{k+kt}{COMPLEX}\PYG{p}{(}\PYG{l+m+mi}{8}\PYG{p}{)}\PYG{p}{,}\PYG{k}{INTENT}\PYG{p}{(}\PYG{n}{IN}\PYG{p}{)}\PYG{+w}{ }\PYG{k+kd}{::}\PYG{+w}{ }\PYG{n}{phi}\PYG{p}{(}\PYG{n}{nr}\PYG{p}{,}\PYG{n}{nb}\PYG{p}{,}\PYG{n}{ng}\PYG{p}{(}\PYG{l+m+mi}{1}\PYG{p}{)}\PYG{p}{,}\PYG{n}{ng}\PYG{p}{(}\PYG{l+m+mi}{2}\PYG{p}{)}\PYG{p}{,}\PYG{n}{ng}\PYG{p}{(}\PYG{l+m+mi}{3}\PYG{p}{)}\PYG{p}{)}\PYG{+w}{ }\PYG{c}{! Kohn\PYGZhy{}Sham orbitals}
\PYG{+w}{  }\PYG{k+kt}{REAL}\PYG{p}{(}\PYG{l+m+mi}{8}\PYG{p}{)}\PYG{p}{,}\PYG{k}{INTENT}\PYG{p}{(}\PYG{n}{OUT}\PYG{p}{)}\PYG{+w}{ }\PYG{k+kd}{::}\PYG{+w}{ }\PYG{n}{rho}\PYG{p}{(}\PYG{n}{nr}\PYG{p}{)}\PYG{+w}{ }\PYG{c}{! Charge density}
\PYG{+w}{  }\PYG{c}{!}
\PYG{+w}{  }\PYG{k+kt}{INTEGER}\PYG{+w}{ }\PYG{k+kd}{::}\PYG{+w}{ }\PYG{n}{ib}\PYG{p}{,}\PYG{+w}{ }\PYG{n}{i1}\PYG{p}{,}\PYG{+w}{ }\PYG{n}{i2}\PYG{p}{,}\PYG{+w}{ }\PYG{n}{i3}\PYG{p}{,}\PYG{+w}{ }\PYG{n}{ltetra}
\PYG{+w}{  }\PYG{k+kt}{REAL}\PYG{p}{(}\PYG{l+m+mi}{8}\PYG{p}{)}\PYG{+w}{ }\PYG{k+kd}{::}\PYG{+w}{ }\PYG{n}{wght}\PYG{p}{(}\PYG{n}{nb}\PYG{p}{,}\PYG{n}{ng}\PYG{p}{(}\PYG{l+m+mi}{1}\PYG{p}{)}\PYG{p}{,}\PYG{n}{ng}\PYG{p}{(}\PYG{l+m+mi}{2}\PYG{p}{)}\PYG{p}{,}\PYG{n}{ng}\PYG{p}{(}\PYG{l+m+mi}{3}\PYG{p}{)}\PYG{p}{)}
\PYG{+w}{  }\PYG{c}{!}
\PYG{+w}{  }\PYG{n}{ltetra}\PYG{+w}{ }\PYG{o}{=}\PYG{+w}{ }\PYG{l+m+mi}{2}
\PYG{+w}{  }\PYG{c}{!}
\PYG{+w}{  }\PYG{k}{CALL }\PYG{n}{libtetrabz\PYGZus{}fermieng}\PYG{p}{(}\PYG{n}{ltetra}\PYG{p}{,}\PYG{n}{bvec}\PYG{p}{,}\PYG{n}{nb}\PYG{p}{,}\PYG{n}{ng}\PYG{p}{,}\PYG{n}{eig}\PYG{p}{,}\PYG{n}{ng}\PYG{p}{,}\PYG{n}{wght}\PYG{p}{,}\PYG{n}{ef}\PYG{p}{,}\PYG{n}{nelec}\PYG{p}{)}
\PYG{+w}{  }\PYG{c}{!}
\PYG{+w}{  }\PYG{n}{rho}\PYG{p}{(}\PYG{l+m+mi}{1}\PYG{p}{:}\PYG{n}{nr}\PYG{p}{)}\PYG{+w}{ }\PYG{o}{=}\PYG{+w}{ }\PYG{l+m+mi}{0}\PYG{n}{d0}
\PYG{+w}{  }\PYG{k}{DO }\PYG{n}{i1}\PYG{+w}{ }\PYG{o}{=}\PYG{+w}{ }\PYG{l+m+mi}{1}\PYG{p}{,}\PYG{+w}{ }\PYG{n}{ng}\PYG{p}{(}\PYG{l+m+mi}{3}\PYG{p}{)}
\PYG{+w}{     }\PYG{k}{DO }\PYG{n}{i2}\PYG{+w}{ }\PYG{o}{=}\PYG{+w}{ }\PYG{l+m+mi}{1}\PYG{p}{,}\PYG{+w}{ }\PYG{n}{ng}\PYG{p}{(}\PYG{l+m+mi}{2}\PYG{p}{)}
\PYG{+w}{        }\PYG{k}{DO }\PYG{n}{i1}\PYG{+w}{ }\PYG{o}{=}\PYG{+w}{ }\PYG{l+m+mi}{1}\PYG{p}{,}\PYG{+w}{ }\PYG{n}{ng}\PYG{p}{(}\PYG{l+m+mi}{1}\PYG{p}{)}
\PYG{+w}{           }\PYG{k}{DO }\PYG{n}{ib}\PYG{+w}{ }\PYG{o}{=}\PYG{+w}{ }\PYG{l+m+mi}{1}\PYG{p}{,}\PYG{+w}{ }\PYG{n}{nb}
\PYG{+w}{              }\PYG{n}{rho}\PYG{p}{(}\PYG{l+m+mi}{1}\PYG{p}{:}\PYG{n}{nr}\PYG{p}{)}\PYG{+w}{ }\PYG{o}{=}\PYG{+w}{ }\PYG{n}{rho}\PYG{p}{(}\PYG{l+m+mi}{1}\PYG{p}{:}\PYG{n}{nr}\PYG{p}{)}\PYG{+w}{ }\PYG{o}{+}\PYG{+w}{ }\PYG{l+m+mi}{2}\PYG{n}{d0}\PYG{+w}{ }\PYG{o}{*}\PYG{+w}{ }\PYG{n}{wght}\PYG{p}{(}\PYG{n}{ib}\PYG{p}{,}\PYG{n}{i1}\PYG{p}{,}\PYG{n}{i2}\PYG{p}{,}\PYG{n}{i3}\PYG{p}{)}\PYG{+w}{ }\PYG{p}{\PYGZam{}}
\PYG{+w}{              }\PYG{p}{\PYGZam{}}\PYG{+w}{     }\PYG{o}{*}\PYG{+w}{ }\PYG{n+nb}{DBLE}\PYG{p}{(}\PYG{n+nb}{CONJG}\PYG{p}{(}\PYG{n}{phi}\PYG{p}{(}\PYG{l+m+mi}{1}\PYG{p}{:}\PYG{n}{nr}\PYG{p}{,}\PYG{n}{ib}\PYG{p}{,}\PYG{n}{i1}\PYG{p}{,}\PYG{n}{i2}\PYG{p}{,}\PYG{n}{i3}\PYG{p}{)}\PYG{p}{)}\PYG{+w}{ }\PYG{o}{*}\PYG{+w}{ }\PYG{n}{phi}\PYG{p}{(}\PYG{l+m+mi}{1}\PYG{p}{:}\PYG{n}{nr}\PYG{p}{,}\PYG{n}{ib}\PYG{p}{,}\PYG{n}{i1}\PYG{p}{,}\PYG{n}{i2}\PYG{p}{,}\PYG{n}{i3}\PYG{p}{)}\PYG{p}{)}
\PYG{+w}{           }\PYG{k}{END }\PYG{k}{DO}
\PYG{k}{        }\PYG{k}{END }\PYG{k}{DO}
\PYG{k}{     }\PYG{k}{END }\PYG{k}{DO}
\PYG{k}{  }\PYG{k}{END }\PYG{k}{DO}
\PYG{+w}{  }\PYG{c}{!}
\PYG{k}{END }\PYG{k}{SUBROUTINE }\PYG{n}{calc\PYGZus{}rho}
\end{sphinxVerbatim}

\sphinxstepscope


\chapter{Re\sphinxhyphen{}distribution of this program}
\label{\detokenize{copy:re-distribution-of-this-program}}\label{\detokenize{copy::doc}}

\section{Contain libtetrabz in your program}
\label{\detokenize{copy:contain-libtetrabz-in-your-program}}
\sphinxAtStartPar
libtetrabz is distributed with the {\hyperref[\detokenize{copy:mitlicense}]{\sphinxcrossref{\DUrole{std,std-ref}{MIT License}}}}.
To summarize this, you can freely modify, copy and paste libtetrabz to any program
such as a private program (in the research group, co\sphinxhyphen{}workers, etc.),
open\sphinxhyphen{}source, free, and commercial software.
Also, you can freely choose the license to distribute your program.


\section{Build libtetrabz without Autoconf}
\label{\detokenize{copy:build-libtetrabz-without-autoconf}}
\sphinxAtStartPar
In this package, libtetrabz is built with Autotools (Autoconf, Automake, Libtool).
If you do not want to use Autotools for your distributed program with libtetrabz’s source,
you can use the following simple Makefile (please care about TAB).

\begin{sphinxVerbatim}[commandchars=\\\{\}]
\PYG{n+nv}{F90}\PYG{+w}{ }\PYG{o}{=}\PYG{+w}{ }gfortran
\PYG{n+nv}{FFLAGS}\PYG{+w}{ }\PYG{o}{=}\PYG{+w}{ }\PYGZhy{}fopenmp\PYG{+w}{ }\PYGZhy{}O2\PYG{+w}{ }\PYGZhy{}g

\PYG{n+nv}{OBJS}\PYG{+w}{ }\PYG{o}{=}\PYG{+w}{ }\PYG{l+s+se}{\PYGZbs{}}
libtetrabz.o\PYG{+w}{ }\PYG{l+s+se}{\PYGZbs{}}
libtetrabz\PYGZus{}dbldelta\PYGZus{}mod.o\PYG{+w}{ }\PYG{l+s+se}{\PYGZbs{}}
libtetrabz\PYGZus{}dblstep\PYGZus{}mod.o\PYG{+w}{ }\PYG{l+s+se}{\PYGZbs{}}
libtetrabz\PYGZus{}dos\PYGZus{}mod.o\PYG{+w}{ }\PYG{l+s+se}{\PYGZbs{}}
libtetrabz\PYGZus{}fermigr\PYGZus{}mod.o\PYG{+w}{ }\PYG{l+s+se}{\PYGZbs{}}
libtetrabz\PYGZus{}occ\PYGZus{}mod.o\PYG{+w}{ }\PYG{l+s+se}{\PYGZbs{}}
libtetrabz\PYGZus{}polcmplx\PYGZus{}mod.o\PYG{+w}{ }\PYG{l+s+se}{\PYGZbs{}}
libtetrabz\PYGZus{}polstat\PYGZus{}mod.o\PYG{+w}{ }\PYG{l+s+se}{\PYGZbs{}}
libtetrabz\PYGZus{}common.o\PYG{+w}{ }\PYG{l+s+se}{\PYGZbs{}}

\PYG{n+nf}{.SUFFIXES }\PYG{o}{:}
\PYG{n+nf}{.SUFFIXES }\PYG{o}{:}\PYG{+w}{ }.\PYG{n}{o} .\PYG{n}{F}90

\PYG{n+nf}{libtetrabz.a}\PYG{o}{:}\PYG{k}{\PYGZdl{}(}\PYG{n+nv}{OBJS}\PYG{k}{)}
\PYG{+w}{     }ar\PYG{+w}{ }cr\PYG{+w}{ }\PYG{n+nv}{\PYGZdl{}@}\PYG{+w}{ }\PYG{k}{\PYGZdl{}(}OBJS\PYG{k}{)}

\PYG{n+nf}{.F90.o}\PYG{o}{:}
\PYG{+w}{      }\PYG{k}{\PYGZdl{}(}F90\PYG{k}{)}\PYG{+w}{ }\PYG{k}{\PYGZdl{}(}FFLAGS\PYG{k}{)}\PYG{+w}{ }\PYGZhy{}c\PYG{+w}{ }\PYGZdl{}\PYGZlt{}

\PYG{n+nf}{clean}\PYG{o}{:}
\PYG{+w}{      }rm\PYG{+w}{ }\PYGZhy{}f\PYG{+w}{ }*.a\PYG{+w}{ }*.o\PYG{+w}{ }*.mod

\PYG{n+nf}{libtetrabz.o}\PYG{o}{:}\PYG{n}{libtetrabz\PYGZus{}polcmplx\PYGZus{}mod}.\PYG{n}{o}
\PYG{n+nf}{libtetrabz.o}\PYG{o}{:}\PYG{n}{libtetrabz\PYGZus{}fermigr\PYGZus{}mod}.\PYG{n}{o}
\PYG{n+nf}{libtetrabz.o}\PYG{o}{:}\PYG{n}{libtetrabz\PYGZus{}polstat\PYGZus{}mod}.\PYG{n}{o}
\PYG{n+nf}{libtetrabz.o}\PYG{o}{:}\PYG{n}{libtetrabz\PYGZus{}dbldelta\PYGZus{}mod}.\PYG{n}{o}
\PYG{n+nf}{libtetrabz.o}\PYG{o}{:}\PYG{n}{libtetrabz\PYGZus{}dblstep\PYGZus{}mod}.\PYG{n}{o}
\PYG{n+nf}{libtetrabz.o}\PYG{o}{:}\PYG{n}{libtetrabz\PYGZus{}dos\PYGZus{}mod}.\PYG{n}{o}
\PYG{n+nf}{libtetrabz.o}\PYG{o}{:}\PYG{n}{libtetrabz\PYGZus{}occ\PYGZus{}mod}.\PYG{n}{o}
\PYG{n+nf}{libtetrabz\PYGZus{}dbldelta\PYGZus{}mod.o}\PYG{o}{:}\PYG{n}{libtetrabz\PYGZus{}common}.\PYG{n}{o}
\PYG{n+nf}{libtetrabz\PYGZus{}dblstep\PYGZus{}mod.o}\PYG{o}{:}\PYG{n}{libtetrabz\PYGZus{}common}.\PYG{n}{o}
\PYG{n+nf}{libtetrabz\PYGZus{}dos\PYGZus{}mod.o}\PYG{o}{:}\PYG{n}{libtetrabz\PYGZus{}common}.\PYG{n}{o}
\PYG{n+nf}{libtetrabz\PYGZus{}fermigr\PYGZus{}mod.o}\PYG{o}{:}\PYG{n}{libtetrabz\PYGZus{}common}.\PYG{n}{o}
\PYG{n+nf}{libtetrabz\PYGZus{}occ\PYGZus{}mod.o}\PYG{o}{:}\PYG{n}{libtetrabz\PYGZus{}common}.\PYG{n}{o}
\PYG{n+nf}{libtetrabz\PYGZus{}polcmplx\PYGZus{}mod.o}\PYG{o}{:}\PYG{n}{libtetrabz\PYGZus{}common}.\PYG{n}{o}
\PYG{n+nf}{libtetrabz\PYGZus{}polstat\PYGZus{}mod.o}\PYG{o}{:}\PYG{n}{libtetrabz\PYGZus{}common}.\PYG{n}{o}
\end{sphinxVerbatim}


\section{MIT License}
\label{\detokenize{copy:mit-license}}\label{\detokenize{copy:mitlicense}}
\begin{DUlineblock}{0em}
\item[] Copyright (c) 2014 Mitsuaki Kawamura
\item[] 
\item[] Permission is hereby granted, free of charge, to any person obtaining a
\item[] copy of this software and associated documentation files (the
\item[] “Software”), to deal in the Software without restriction, including
\item[] without limitation the rights to use, copy, modify, merge, publish,
\item[] distribute, sublicense, and/or sell copies of the Software, and to
\item[] permit persons to whom the Software is furnished to do so, subject to
\item[] the following conditions:
\item[] 
\item[] The above copyright notice and this permission notice shall be included
\item[] in all copies or substantial portions of the Software.
\item[] 
\item[] THE SOFTWARE IS PROVIDED “AS IS”, WITHOUT WARRANTY OF ANY KIND, EXPRESS
\item[] OR IMPLIED, INCLUDING BUT NOT LIMITED TO THE WARRANTIES OF
\item[] MERCHANTABILITY, FITNESS FOR A PARTICULAR PURPOSE AND NONINFRINGEMENT.
\item[] IN NO EVENT SHALL THE AUTHORS OR COPYRIGHT HOLDERS BE LIABLE FOR ANY
\item[] CLAIM, DAMAGES OR OTHER LIABILITY, WHETHER IN AN ACTION OF CONTRACT,
\item[] TORT OR OTHERWISE, ARISING FROM, OUT OF OR IN CONNECTION WITH THE
\item[] SOFTWARE OR THE USE OR OTHER DEALINGS IN THE SOFTWARE.
\end{DUlineblock}

\sphinxstepscope


\chapter{Contacts}
\label{\detokenize{contact:contacts}}\label{\detokenize{contact::doc}}
\sphinxAtStartPar
Please post bug reports and questions to the forum
\begin{quote}

\sphinxAtStartPar
\sphinxurl{http://sourceforge.jp/projects/fermisurfer/forums/}
\end{quote}

\sphinxAtStartPar
When you want to join us, please contact me as follows.

\sphinxAtStartPar
The Institute of Solid State Physics

\sphinxAtStartPar
Mitsuaki Kawamura

\sphinxAtStartPar
\sphinxcode{\sphinxupquote{mkawamura\_\_at\_\_issp.u\sphinxhyphen{}tokyo.ac.jp}}

\sphinxstepscope


\chapter{Appendix}
\label{\detokenize{app:appendix}}\label{\detokenize{app:app}}\label{\detokenize{app::doc}}

\section{Inverse interpolation}
\label{\detokenize{app:inverse-interpolation}}
\sphinxAtStartPar
We consider an integration as follows:
\begin{equation*}
\begin{split}\begin{align}
\langle X \rangle = \sum_{k} X_k w(\varepsilon_k)
\end{align}\end{split}
\end{equation*}
\sphinxAtStartPar
If this integration has conditions that
\begin{itemize}
\item {} 
\sphinxAtStartPar
\(w(\varepsilon_k)\) is sensitive to \(\varepsilon_k\) (e. g. the
stepfunction, the delta function, etc.) and requires
\(\varepsilon_k\) on a dense \(k\) grid, and

\item {} 
\sphinxAtStartPar
the numerical cost to obtain \(X_k\) is much larger than the cost for
\(\varepsilon_k\) (e. g. the polarization function),

\end{itemize}

\sphinxAtStartPar
it is efficient to interpolate \(X_k\) into a denser \(k\) grid and
evaluate that integration in a dense \(k\) grid. This method is performed
as follows:
\begin{enumerate}
\sphinxsetlistlabels{\arabic}{enumi}{enumii}{}{.}%
\item {} 
\sphinxAtStartPar
Calculate \(\varepsilon_k\) on a dense \(k\) grid.

\item {} 
\sphinxAtStartPar
Calculate \(X_k\) on a coarse \(k\) grid and obtain that on a dense \(k\)
grid by using the linear interpolation, the polynomial interpolation,
the spline interpolation, etc.

\end{enumerate}
\begin{equation*}
\begin{split}\begin{align}
X_k^{\rm dense} = \sum_{k'}^{\rm coarse}
F_{k k'} X_{k'}^{\rm coarse}
\end{align}\end{split}
\end{equation*}\begin{enumerate}
\sphinxsetlistlabels{\arabic}{enumi}{enumii}{}{.}%
\item {} 
\sphinxAtStartPar
Evaluate that integration in the dense \(k\) grid.

\end{enumerate}
\begin{equation*}
\begin{split}\begin{align}
\langle X \rangle = \sum_{k}^{\rm dense}
X_k^{\rm dense} w_k^{\rm dense}
\end{align}\end{split}
\end{equation*}
\sphinxAtStartPar
\sphinxstylestrong{The inverse interpolation method}  (Appendix of {\hyperref[\detokenize{ref:ref}]{\sphinxcrossref{\DUrole{std,std-ref}{{[}2{]}}}}})
arrows as to obtain the same result
to above without interpolating \(X_k\) into a dense \(k\) grid. In this
method, we map the integration weight on a dense \(k\) grid into that on a
coarse \(k\) grid (inverse interpolation). Therefore, if we require
\begin{equation*}
\begin{split}\begin{align}
\sum_k^{\rm dense} X_k^{\rm dense} w_k^{\rm dense}
= \sum_k^{\rm coarse} X_k^{\rm coarse} w_k^{\rm coarse}
\end{align}\end{split}
\end{equation*}
\sphinxAtStartPar
we obtain
\begin{equation*}
\begin{split}\begin{align}
w_k^{\rm coarse} = \sum_k^{\rm dense} F_{k' k}
w_{k'}^{\rm dense}
\end{align}\end{split}
\end{equation*}
\sphinxAtStartPar
The numerical procedure for this method is as follows:
\begin{enumerate}
\sphinxsetlistlabels{\arabic}{enumi}{enumii}{}{.}%
\item {} 
\sphinxAtStartPar
Calculate the integration weight on a dense \(k\) grid
\(w_k^{\rm dense}\) from \(\varepsilon_k\) on a dense \(k\) grid.

\item {} 
\sphinxAtStartPar
Obtain the integration weight on a coarse \(k\) grid \(w_k^{\rm
coarse}\) by using the inverse interpolation method.

\item {} 
\sphinxAtStartPar
Evaluate that integration in a coarse \(k\) grid where \(X_k\) was
calculated.

\end{enumerate}

\sphinxAtStartPar
All routines in \sphinxcode{\sphinxupquote{libtetrabz}} can perform the inverse interpolation
method; if we make \(k\) grids for the orbital energy (\sphinxcode{\sphinxupquote{nge}}) and the
integration weight (\sphinxcode{\sphinxupquote{ngw}}) different, we obtain \(w_k^{\rm coarse}\)
calculated by using the inverse interpolation method.


\section{Double delta integration}
\label{\detokenize{app:double-delta-integration}}
\sphinxAtStartPar
For the integration
\begin{equation*}
\begin{split}\begin{align}
\sum_{n n' k} \delta(\varepsilon_{\rm F} -
\varepsilon_{n k}) \delta(\varepsilon_{\rm F} - \varepsilon'_{n' k})
X_{n n' k}
\end{align}\end{split}
\end{equation*}
\sphinxAtStartPar
first, we cut out one or two triangles where
\(\varepsilon_{n k} = \varepsilon_{\rm F}\) from a tetrahedron
and evaluate \(\varepsilon_{n' k+q}\) at the corners of each triangles as
\begin{equation*}
\begin{split}\begin{align}
\varepsilon'^{k+q}_{i} = \sum_{j=1}^4 F_{i j}(
\varepsilon_1^{k}, \cdots, \varepsilon_{4}^{k}, \varepsilon_{\rm F})
\epsilon_{j}^{k+q}.
\end{align}\end{split}
\end{equation*}
\sphinxAtStartPar
Then we calculate \(\delta(\varepsilon_{n' k+q} - \varepsilon{\rm F})\)
in each triangles and obtain weights of corners.
This weights of corners are mapped into those of corners of the original tetrahedron as
\begin{equation*}
\begin{split}\begin{align}
W_{i} = \sum_{j=1}^3 \frac{S}{\nabla_k \varepsilon_k}F_{j i}(
\varepsilon_{1}^k, \cdots, \varepsilon_{4}^k, \varepsilon_{\rm F})
W'_{j}.
\end{align}\end{split}
\end{equation*}
\sphinxAtStartPar
\(F_{i j}\) and \(\frac{S}{\nabla_k \varepsilon_k}\) are calculated as follows
(\(a_{i j} \equiv (\varepsilon_i - \varepsilon_j)/(\varepsilon_{\rm F} - \varepsilon_j)\)):

\begin{figure}[htbp]
\centering
\capstart

\noindent\sphinxincludegraphics[scale=1.0]{{dbldelta}.png}
\caption{How to divide a tetrahedron
in the case of \(\epsilon_1 \leq \varepsilon_{\rm F} \leq \varepsilon_2\) (a),
\(\varepsilon_2 \leq \varepsilon_{\rm F} \leq \varepsilon_3\) (b), and
\(\varepsilon_3 \leq \varepsilon_{\rm F} \leq \varepsilon_4\) (c).}\label{\detokenize{app:id1}}\label{\detokenize{app:dbldeltapng}}\end{figure}
\begin{itemize}
\item {} 
\sphinxAtStartPar
When \(\varepsilon_1 \leq \varepsilon_{\rm F} \leq \varepsilon_2 \leq \varepsilon_3 \leq\varepsilon_4\)
{[}Fig. \ref{app:dbldeltapng} (a){]},
\begin{quote}
\begin{equation*}
\begin{split}\begin{align}
F &=
\begin{pmatrix}
a_{1 2} & a_{2 1} &       0 & 0 \\
a_{1 3} &       0 & a_{3 1} & 0 \\
a_{1 4} &       0 &       0 & a_{4 1}
\end{pmatrix},
\qquad
\frac{S}{\nabla_k \varepsilon_k} = \frac{3 a_{2 1} a_{3 1} a_{4 1}}{\varepsilon_{\rm F} - \varepsilon_1}
\end{align}\end{split}
\end{equation*}\end{quote}

\item {} 
\sphinxAtStartPar
When \(\varepsilon_1 \leq \varepsilon_2 \leq \varepsilon_{\rm F} \leq \varepsilon_3 \leq\varepsilon_4\)
{[}Fig. \ref{app:dbldeltapng} (b){]},
\begin{quote}
\begin{equation*}
\begin{split}\begin{align}
F &=
\begin{pmatrix}
a_{1 3} &       0 & a_{3 1} & 0 \\
a_{1 4} &       0 &       0 & a_{4 1} \\
0 & a_{2 4} &       0 & a_{4 2}
\end{pmatrix},
\qquad
\frac{S}{\nabla_k \varepsilon_k} = \frac{3 a_{3 1} a_{4 1} a_{2 4}}{\varepsilon_{\rm F} - \varepsilon_1}
\end{align}\end{split}
\end{equation*}\begin{equation*}
\begin{split}\begin{align}
F &=
\begin{pmatrix}
a_{1 3} &       0 & a_{3 1} & 0 \\
0 & a_{2 3} & a_{3 2} & 0 \\
0 & a_{2 4} &       0 & a_{4 2}
\end{pmatrix},
\qquad
\frac{S}{\nabla_k \varepsilon_k} = \frac{3 a_{2 3} a_{3 1} a_{4 2}}{\varepsilon_{\rm F} - \varepsilon_1}
\end{align}\end{split}
\end{equation*}\end{quote}

\item {} 
\sphinxAtStartPar
When \(\varepsilon_1 \leq \varepsilon_2 \leq \varepsilon_3 \leq \varepsilon_{\rm F} \leq \varepsilon_4\)
{[}Fig. \ref{app:dbldeltapng} (c){]},
\begin{quote}
\begin{equation*}
\begin{split}\begin{align}
F &=
\begin{pmatrix}
a_{1 4} &       0 &       0 & a_{4 1} \\
a_{1 3} & a_{2 4} &       0 & a_{4 2} \\
a_{1 2} &       0 & a_{3 4} & a_{4 3}
\end{pmatrix},
\qquad
\frac{S}{\nabla_k \varepsilon_k} = \frac{3 a_{1 4} a_{2 4} a_{3 4}}{\varepsilon_1 - \varepsilon_{\rm F}}
\end{align}\end{split}
\end{equation*}\end{quote}

\end{itemize}

\sphinxAtStartPar
Weights on each corners of the triangle are computed as follows
{[}(\(a'_{i j} \equiv (\varepsilon'_i - \varepsilon'_j)/(\varepsilon_{\rm F} - \varepsilon'_j)\)){]}:
\begin{itemize}
\item {} 
\sphinxAtStartPar
When \(\varepsilon'_1 \leq \varepsilon_{\rm F} \leq \varepsilon'_2 \leq \varepsilon'_3\) {[}Fig. \ref{app:dbldeltapng} (d){]},
\begin{quote}
\begin{equation*}
\begin{split}\begin{align}
W'_1 = L (a'_{1 2} + a'_{1 3}), \qquad
W'_2 = L a'_{2 1}, \qquad
W'_3 = L a'_{3 1}, \qquad
L \equiv \frac{a'_{2 1} a'_{3 1}}{\varepsilon_{\rm F} - \varepsilon'_{1}}
\end{align}\end{split}
\end{equation*}\end{quote}

\item {} 
\sphinxAtStartPar
When \(\varepsilon'_1 \leq \varepsilon'_2 \leq \varepsilon_{\rm F} \leq \varepsilon'_3\) {[}Fig. \ref{app:dbldeltapng} (e){]},
\begin{quote}
\begin{equation*}
\begin{split}\begin{align}
W'_1 = L a'_{1 3}, \qquad
W'_2 = L a'_{2 3}, \qquad
W'_3 = L (a'_{3 1} + a'_{3 2}), \qquad
L \equiv \frac{a'_{1 3} a'_{2 3}}{\varepsilon'_{3} - \varepsilon_{\rm F}}
\end{align}\end{split}
\end{equation*}\end{quote}

\end{itemize}

\sphinxstepscope


\chapter{Reference}
\label{\detokenize{ref:reference}}\label{\detokenize{ref:ref}}\label{\detokenize{ref::doc}}
\sphinxAtStartPar
{[}1{]} \sphinxhref{https://journals.aps.org/prb/abstract/10.1103/PhysRevB.89.094515}{M. Kawamura, Y. Gohda, and S. Tsuneyuki, Phys. Rev. B 89, 094515 (2014).}

\sphinxAtStartPar
{[}2{]} \sphinxhref{https://journals.aps.org/prb/abstract/10.1103/PhysRevB.95.054506}{M. Kawamura, R. Akashi, and S. Tsuneyuki, Phys. Rev. B 95, 054506 (2017).}



\renewcommand{\indexname}{Index}
\printindex
\end{document}