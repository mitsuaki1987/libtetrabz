%% Generated by Sphinx.
\def\sphinxdocclass{ujbook}
\documentclass[letterpaper,10pt,dvipdfmx,openany]{sphinxmanual}
\ifdefined\pdfpxdimen
   \let\sphinxpxdimen\pdfpxdimen\else\newdimen\sphinxpxdimen
\fi \sphinxpxdimen=.75bp\relax
\ifdefined\pdfimageresolution
    \pdfimageresolution= \numexpr \dimexpr1in\relax/\sphinxpxdimen\relax
\fi
%% let collapsible pdf bookmarks panel have high depth per default
\PassOptionsToPackage{bookmarksdepth=5}{hyperref}

\PassOptionsToPackage{booktabs}{sphinx}
\PassOptionsToPackage{colorrows}{sphinx}

\PassOptionsToPackage{warn}{textcomp}


\usepackage{cmap}
\usepackage[T1]{fontenc}
\usepackage{amsmath,amssymb,amstext}




\usepackage{tgtermes}
\usepackage{tgheros}
\renewcommand{\ttdefault}{txtt}




\usepackage{sphinx}

\fvset{fontsize=auto}
\usepackage[dvipdfm]{geometry}


% Include hyperref last.
\usepackage{hyperref}
% Fix anchor placement for figures with captions.
\usepackage{hypcap}% it must be loaded after hyperref.
% Set up styles of URL: it should be placed after hyperref.
\urlstyle{same}
\usepackage{pxjahyper}


\usepackage{sphinxmessages}
\setcounter{tocdepth}{2}

\usepackage{pxjahyper}

\title{Libtetrabz Documentation}
\date{2023年05月29日}
\release{2.0.0}
\author{kawamura}
\newcommand{\sphinxlogo}{\sphinxincludegraphics{libtetrabz.png}\par}
\renewcommand{\releasename}{リリース}
\makeindex
\begin{document}

\pagestyle{empty}
\sphinxmaketitle
\pagestyle{plain}
\sphinxtableofcontents
\pagestyle{normal}
\phantomsection\label{\detokenize{index::doc}}


\sphinxstepscope


\chapter{はじめに}
\label{\detokenize{overview:id1}}\label{\detokenize{overview::doc}}
\sphinxAtStartPar
この文書ではテトラへドロン法ライブラリ \sphinxcode{\sphinxupquote{libtetrabz}} についての解説を行っている.
\sphinxcode{\sphinxupquote{libtetrabz}} は線形テトラへドロン法もしくは最適化線形テトラへドロン法 {\hyperref[\detokenize{ref:ref}]{\sphinxcrossref{\DUrole{std,std-ref}{{[}1{]}}}}}
を用いて全エネルギーや電荷密度, 部分状態密度,
分極関数等を計算するためのライブラリ群である.
このライブラリには, 軌道エネルギーをインプットとして,
\begin{equation*}
\begin{split}\begin{align}
\sum_{n n'} \int_{\rm BZ} \frac{d^3 k}{V_{\rm BZ}} F(\varepsilon_{n k}, \varepsilon_{n' k+q})X_{n n' k}
= \sum_{n n'} \sum_{k}^{N_k} w_{n n' k} X_{n n' k}
\end{align}\end{split}
\end{equation*}
\sphinxAtStartPar
のような積分における, 積分重み \(w_{n n' k}\) を出力するサブルーチンを,
各種計算について取り揃えている. 具体的には以下の計算に対応している.
\begin{equation*}
\begin{split}\begin{align}
\sum_{n}
\int_{\rm BZ} \frac{d^3 k}{V_{\rm BZ}}
\theta(\varepsilon_{\rm F} - \varepsilon_{n k})
X_{n k}
\end{align}\end{split}
\end{equation*}\begin{equation*}
\begin{split}\begin{align}
\sum_{n}
\int_{\rm BZ} \frac{d^3 k}{V_{\rm BZ}}
\delta(\omega - \varepsilon_{n k})
X_{n k}(\omega)
\end{align}\end{split}
\end{equation*}\begin{equation*}
\begin{split}\begin{align}
\sum_{n n'}
\int_{\rm BZ} \frac{d^3 k}{V_{\rm BZ}}
\delta(\varepsilon_{\rm F} - \varepsilon_{n k})
\delta(\varepsilon_{\rm F} - \varepsilon'_{n' k})
X_{n n' k}
\end{align}\end{split}
\end{equation*}\begin{equation*}
\begin{split}\begin{align}
\sum_{n n'}
\int_{\rm BZ} \frac{d^3 k}{V_{\rm BZ}}
\theta(\varepsilon_{\rm F} - \varepsilon_{n k})
\theta(\varepsilon_{n k} - \varepsilon'_{n' k})
X_{n n' k}
\end{align}\end{split}
\end{equation*}\begin{equation*}
\begin{split}\begin{align}
\sum_{n n'}
\int_{\rm BZ} \frac{d^3 k}{V_{\rm BZ}}
\frac{
\theta(\varepsilon_{\rm F} - \varepsilon_{n k})
\theta(\varepsilon'_{n' k} - \varepsilon_{\rm F})}
{\varepsilon'_{n' k} - \varepsilon_{n k}}
X_{n n' k}
\end{align}\end{split}
\end{equation*}\begin{equation*}
\begin{split}\begin{align}
\sum_{n n'}
\int_{\rm BZ} \frac{d^3 k}{V_{\rm BZ}}
\theta(\varepsilon_{\rm F} - \varepsilon_{n k})
\theta(\varepsilon'_{n' k} - \varepsilon_{\rm F})
\delta(\varepsilon'_{n' k} - \varepsilon_{n k} - \omega)
X_{n n' k}(\omega)
\end{align}\end{split}
\end{equation*}\begin{equation*}
\begin{split}\begin{align}
\int_{\rm BZ} \frac{d^3 k}{V_{\rm BZ}}
\sum_{n n'}
\frac{
\theta(\varepsilon_{\rm F} - \varepsilon_{n k})
\theta(\varepsilon'_{n' k} - \varepsilon_{\rm F})}
{\varepsilon'_{n' k} - \varepsilon_{n k} + i \omega}
X_{n n' k}(\omega)
\end{align}\end{split}
\end{equation*}
\sphinxstepscope


\chapter{インストール方法}
\label{\detokenize{install:id1}}\label{\detokenize{install::doc}}

\section{主なファイルとディレクトリ}
\label{\detokenize{install:id2}}\begin{itemize}
\item {} \begin{description}
\sphinxlineitem{\sphinxcode{\sphinxupquote{doc/}}}{[}マニュアルのディレクトリ{]}\begin{itemize}
\item {} 
\sphinxAtStartPar
\sphinxcode{\sphinxupquote{doc/index.html}} : 目録ページ

\end{itemize}

\end{description}

\item {} 
\sphinxAtStartPar
\sphinxcode{\sphinxupquote{src/}} : ライブラリのソースコードのディレクトリ

\item {} 
\sphinxAtStartPar
\sphinxcode{\sphinxupquote{example/}} : ライブラリ使用例のディレクトリ

\item {} 
\sphinxAtStartPar
\sphinxcode{\sphinxupquote{test/}} : ライブラリのビルドのテスト用ディレクトリ

\item {} 
\sphinxAtStartPar
\sphinxcode{\sphinxupquote{configure}} : ビルド環境設定用スクリプト

\end{itemize}


\section{要件}
\label{\detokenize{install:id3}}
\sphinxAtStartPar
以下のものが必要となる.
\begin{itemize}
\item {} 
\sphinxAtStartPar
fortran および C コンパイラ

\item {} 
\sphinxAtStartPar
MPI ライブラリ (MPI/ハイブリッド並列版を利用する場合)

\end{itemize}


\section{インストール手順}
\label{\detokenize{install:id4}}\begin{enumerate}
\sphinxsetlistlabels{\arabic}{enumi}{enumii}{}{.}%
\item {} 
\sphinxAtStartPar
以下の場所から \sphinxcode{\sphinxupquote{.tar.gz}} ファイルをダウンロードする.

\sphinxAtStartPar
\sphinxurl{http://osdn.jp/projects/libtetrabz/releases/}

\item {} 
\sphinxAtStartPar
ダウンロードした \sphinxcode{\sphinxupquote{.tar.gz}} ファイルを展開し,
出来たディレクトリに入る.

\begin{sphinxVerbatim}[commandchars=\\\{\}]
\PYGZdl{}\PYG{+w}{ }tar\PYG{+w}{ }xzvf\PYG{+w}{ }libtetrabz\PYGZhy{}version.tar.gz
\PYGZdl{}\PYG{+w}{ }\PYG{n+nb}{cd}\PYG{+w}{ }libtetrabz\PYGZhy{}version
\end{sphinxVerbatim}

\item {} 
\sphinxAtStartPar
ビルド環境を設定する.

\begin{sphinxVerbatim}[commandchars=\\\{\}]
\PYGZdl{}\PYG{+w}{ }./configure\PYG{+w}{ }\PYGZhy{}\PYGZhy{}prefix\PYG{o}{=}install\PYGZus{}dir
\end{sphinxVerbatim}

\sphinxAtStartPar
これにより, ビルドに必要なコンパイラやライブラリ等の環境のチェックが行われ,
Makefile等が作成される.
ただし \sphinxcode{\sphinxupquote{install\_dir}} はインストール先のディレクトリの絶対パスとする (以後各自のディレクトリ名で読み替えること).
なにも指定しないと \sphinxcode{\sphinxupquote{/use/local/}} が設定され, 後述の \sphinxcode{\sphinxupquote{make install}} で
\sphinxcode{\sphinxupquote{/usr/local/lib}} 内にライブラリが置かれる (したがって, 管理者権限がない場合には \sphinxcode{\sphinxupquote{install\_dir}} を
別の場所に指定しなければならない).
\sphinxcode{\sphinxupquote{configure}} にはこの他にも様々なオプションがあり, 必要に応じて用途や環境に合わせてそれらを使用する.
詳しくは {\hyperref[\detokenize{install:configoption}]{\sphinxcrossref{\DUrole{std,std-ref}{configureのオプション}}}} を参照.

\item {} 
\sphinxAtStartPar
\sphinxcode{\sphinxupquote{configure}} の実行が正常に行われ, \sphinxcode{\sphinxupquote{Makefile}} が生成された後は

\begin{sphinxVerbatim}[commandchars=\\\{\}]
\PYGZdl{}\PYG{+w}{ }make
\end{sphinxVerbatim}

\sphinxAtStartPar
とタイプしてライブラリ等のビルドを行う.これが成功したのちに

\begin{sphinxVerbatim}[commandchars=\\\{\}]
\PYGZdl{}\PYG{+w}{ }make\PYG{+w}{ }install
\end{sphinxVerbatim}

\sphinxAtStartPar
とすると, ライブラリが \sphinxcode{\sphinxupquote{install\_dir/lib}} に置かれる.
\sphinxcode{\sphinxupquote{make install}} をしなくても, ビルドをしたディレクトリ内にあるライブラリやミニアプリを使うことは可能であるが, 使い勝手がやや異なる.

\item {} 
\sphinxAtStartPar
共有リンクを行ったプログラムの実行時にライブラリを探しにいけるよう,
環境変数 \sphinxcode{\sphinxupquote{LD\_LIBRARY\_PATH}} にlibtetrabzをインストールしたディレクトリを追加する.

\begin{sphinxVerbatim}[commandchars=\\\{\}]
\PYGZdl{}\PYG{+w}{ }\PYG{n+nb}{export}\PYG{+w}{ }\PYG{n+nv}{LD\PYGZus{}LIBRARY\PYGZus{}PATH}\PYG{o}{=}\PYG{l+s+si}{\PYGZdl{}\PYGZob{}}\PYG{n+nv}{LD\PYGZus{}LIBRARY\PYGZus{}PATH}\PYG{l+s+si}{\PYGZcb{}}:install\PYGZus{}dir/lib
\end{sphinxVerbatim}

\item {} 
\sphinxAtStartPar
また,
\sphinxcode{\sphinxupquote{example/}} 以下のライブラリ使用例のプログラムもコンパイルされる.

\sphinxAtStartPar
\sphinxcode{\sphinxupquote{example/dos.x}} :
立法格子シングルバンドタイトバインディングモデルのDOSを 計算する.
ソースコードは \sphinxcode{\sphinxupquote{dos.f90}}
\begin{quote}

\begin{figure}[htbp]
\centering
\capstart

\noindent\sphinxincludegraphics[scale=0.5]{{dos}.png}
\caption{dos.xを使って計算した,
立法格子タイトバインディング模型の状態密度.
実線は十分多くの \(k\) 点を利用した時の結果.
" \(+\) ",
" \(\times\) "はそれぞれ
\(8\times8\times8 k\) グリッドでの
線形テトラへドロン法および最適化テトラへドロン法の結果.}\label{\detokenize{install:id5}}\end{figure}
\end{quote}

\sphinxAtStartPar
\sphinxcode{\sphinxupquote{example/lindhard.x}} : リントハルト関数を計算する.
ソースコードは \sphinxcode{\sphinxupquote{lindhard.f90}}
\begin{quote}

\begin{figure}[htbp]
\centering
\capstart

\noindent\sphinxincludegraphics[scale=0.5]{{lindhard}.png}
\caption{lindhard.xを使って計算したLindhard関数.
実線は解析的な結果.
" \(+\) ", " \(\times\) "はそれぞれ
\(8\times8\times8 k\)
グリッドでの線形テトラへドロン法および最適化テトラへドロン法の結果.}\label{\detokenize{install:id6}}\end{figure}
\end{quote}

\end{enumerate}


\section{configureのオプション}
\label{\detokenize{install:configure}}\label{\detokenize{install:configoption}}
\sphinxAtStartPar
configureには多数のオプションと変数があり, それらを組み合わせて指定する.
指定しない場合にはデフォルト値が使われる.

\begin{sphinxVerbatim}[commandchars=\\\{\}]
\PYGZdl{}\PYG{+w}{ }./configure\PYG{+w}{ }\PYGZhy{}\PYGZhy{}prefix\PYG{o}{=}/home/libtetrabz/\PYG{+w}{ }\PYGZhy{}\PYGZhy{}with\PYGZhy{}mpi\PYG{o}{=}yes\PYG{+w}{ }\PYG{n+nv}{FC}\PYG{o}{=}mpif90
\end{sphinxVerbatim}

\sphinxAtStartPar
おもなものを次に挙げる.

\sphinxAtStartPar
\sphinxcode{\sphinxupquote{\sphinxhyphen{}\sphinxhyphen{}\sphinxhyphen{}prefix}}
\begin{quote}

\sphinxAtStartPar
デフォルト: \sphinxcode{\sphinxupquote{\sphinxhyphen{}\sphinxhyphen{}\sphinxhyphen{}prefix=/usr/local/}}.
ライブラリ等のインストールを行うディレクトリツリーを指定する.
\end{quote}

\sphinxAtStartPar
\sphinxcode{\sphinxupquote{\sphinxhyphen{}\sphinxhyphen{}with\sphinxhyphen{}mpi}}
\begin{quote}

\sphinxAtStartPar
デフォルト: \sphinxcode{\sphinxupquote{\sphinxhyphen{}\sphinxhyphen{}with\sphinxhyphen{}mpi=no}} (MPIを用いない).
MPIを用いるか (\sphinxcode{\sphinxupquote{\sphinxhyphen{}\sphinxhyphen{}with\sphinxhyphen{}mpi=yes}}), 否かを指定する.
\end{quote}

\sphinxAtStartPar
\sphinxcode{\sphinxupquote{\sphinxhyphen{}\sphinxhyphen{}with\sphinxhyphen{}openmp}}
\begin{quote}

\sphinxAtStartPar
デフォルト: \sphinxcode{\sphinxupquote{\sphinxhyphen{}\sphinxhyphen{}with\sphinxhyphen{}openmp=yes}} (OpenMPを用いる).
OpenMPを用いるか否か (\sphinxcode{\sphinxupquote{\sphinxhyphen{}\sphinxhyphen{}with\sphinxhyphen{}openmp=no}}) を指定する.
\end{quote}

\sphinxAtStartPar
\sphinxcode{\sphinxupquote{\sphinxhyphen{}\sphinxhyphen{}enable\sphinxhyphen{}shared}}
\begin{quote}

\sphinxAtStartPar
デフォルト: \sphinxcode{\sphinxupquote{\sphinxhyphen{}\sphinxhyphen{}enable\sphinxhyphen{}shared}}.
共有ライブラリを作成するか否か
\end{quote}

\sphinxAtStartPar
\sphinxcode{\sphinxupquote{\sphinxhyphen{}\sphinxhyphen{}enable\sphinxhyphen{}static}}
\begin{quote}

\sphinxAtStartPar
デフォルト: \sphinxcode{\sphinxupquote{\sphinxhyphen{}\sphinxhyphen{}enable\sphinxhyphen{}static}}.
静的ライブラリを作成するか否か.
\end{quote}

\sphinxAtStartPar
\sphinxcode{\sphinxupquote{FC}}, \sphinxcode{\sphinxupquote{CC}}
\begin{quote}

\sphinxAtStartPar
デフォルト: システムにインストールされているfortran/Cコンパイラをスキャンして,
自動的に設定する. \sphinxcode{\sphinxupquote{\sphinxhyphen{}\sphinxhyphen{}with\sphinxhyphen{}mpi}} を指定した時にはそれに応じたコマンド
(\sphinxcode{\sphinxupquote{mpif90}} 等)を自動で探し出して設定する.
\sphinxcode{\sphinxupquote{configure}} の最後に出力される \sphinxcode{\sphinxupquote{FC}} が望んだものでは無かった場合には
\sphinxcode{\sphinxupquote{./configure FC=gfortran}} のように手で指定する.
\end{quote}

\sphinxAtStartPar
\sphinxcode{\sphinxupquote{\sphinxhyphen{}\sphinxhyphen{}help}}
\begin{quote}

\sphinxAtStartPar
このオプションを指定した時には, ビルドの環境設定は行われず,
上記を含めたすべてのオプションを表示する.
\end{quote}

\sphinxstepscope


\chapter{ライブラリのリンク方法}
\label{\detokenize{link:id1}}\label{\detokenize{link::doc}}
\sphinxAtStartPar
\sphinxstylestrong{例/ intel fortran の場合}
\begin{quote}

\begin{sphinxVerbatim}[commandchars=\\\{\}]
\PYGZdl{}\PYG{+w}{ }ifort\PYG{+w}{ }program.f90\PYG{+w}{ }\PYGZhy{}L\PYG{+w}{ }install\PYGZus{}dir/lib\PYG{+w}{ }\PYGZhy{}I\PYG{+w}{ }install\PYGZus{}dir/include\PYG{+w}{ }\PYGZhy{}ltetrabz\PYG{+w}{ }\PYGZhy{}fopenmp
\PYGZdl{}\PYG{+w}{ }mpiifort\PYG{+w}{ }program.f90\PYG{+w}{ }\PYGZhy{}L\PYG{+w}{ }install\PYGZus{}dir/lib\PYG{+w}{ }\PYGZhy{}I\PYG{+w}{ }install\PYGZus{}dir/include\PYG{+w}{ }\PYGZhy{}ltetrabz\PYG{+w}{ }\PYGZhy{}fopenmp
\end{sphinxVerbatim}
\end{quote}

\sphinxAtStartPar
\sphinxstylestrong{例/ intel C の場合}
\begin{quote}

\begin{sphinxVerbatim}[commandchars=\\\{\}]
\PYGZdl{}\PYG{+w}{ }icc\PYG{+w}{ }\PYGZhy{}lifcore\PYG{+w}{ }program.f90\PYG{+w}{ }\PYGZhy{}L\PYG{+w}{ }install\PYGZus{}dir/lib\PYG{+w}{ }\PYGZhy{}I\PYG{+w}{ }install\PYGZus{}dir/include\PYG{+w}{ }\PYGZhy{}ltetrabz\PYG{+w}{ }\PYGZhy{}fopenmp
\PYGZdl{}\PYG{+w}{ }mpiicc\PYG{+w}{ }\PYGZhy{}lifcore\PYG{+w}{ }program.f90\PYG{+w}{ }\PYGZhy{}L\PYG{+w}{ }install\PYGZus{}dir/lib\PYG{+w}{ }\PYGZhy{}I\PYG{+w}{ }install\PYGZus{}dir/include\PYG{+w}{ }\PYGZhy{}ltetrabz\PYGZus{}mpi\PYG{+w}{ }\PYGZhy{}fopenmp
\end{sphinxVerbatim}
\end{quote}

\sphinxstepscope


\chapter{各サブルーチンの説明}
\label{\detokenize{routine:id1}}\label{\detokenize{routine::doc}}
\sphinxAtStartPar
以下のサブルーチンを任意のプログラム内で

\begin{sphinxVerbatim}[commandchars=\\\{\}]
\PYG{k}{USE }\PYG{n}{libtetrabz}\PYG{p}{,}\PYG{+w}{ }\PYG{k}{ONLY}\PYG{+w}{ }\PYG{p}{:}\PYG{+w}{ }\PYG{n}{libtetrabz\PYGZus{}occ}

\PYG{k}{CALL }\PYG{n}{libtetrabz\PYGZus{}occ}\PYG{p}{(}\PYG{n}{ltetra}\PYG{p}{,}\PYG{n}{bvec}\PYG{p}{,}\PYG{n}{nb}\PYG{p}{,}\PYG{n}{nge}\PYG{p}{,}\PYG{n}{eig}\PYG{p}{,}\PYG{n}{ngw}\PYG{p}{,}\PYG{n}{wght}\PYG{p}{)}
\end{sphinxVerbatim}

\sphinxAtStartPar
のように呼び出して使用できる.
サブルーチン名はすべて \sphinxcode{\sphinxupquote{libtetrabz\_}} からはじまる.

\sphinxAtStartPar
C言語で書かれたプログラムから呼び出す場合には次のようにする.

\begin{sphinxVerbatim}[commandchars=\\\{\}]
\PYG{c+cp}{\PYGZsh{}}\PYG{c+cp}{include}\PYG{+w}{ }\PYG{c+cpf}{\PYGZdq{}libtetrabz.h\PYGZdq{}}

\PYG{n}{libtetrabz\PYGZus{}occ}\PYG{p}{(}\PYG{o}{\PYGZam{}}\PYG{n}{ltetra}\PYG{p}{,}\PYG{n}{bvec}\PYG{p}{,}\PYG{o}{\PYGZam{}}\PYG{n}{nb}\PYG{p}{,}\PYG{n}{nge}\PYG{p}{,}\PYG{n}{eig}\PYG{p}{,}\PYG{n}{ngw}\PYG{p}{,}\PYG{n}{wght}\PYG{p}{)}
\end{sphinxVerbatim}

\sphinxAtStartPar
変数はすべてポインタとして渡す.
配列はすべて1次元配列として定義し一番左の添字が内側のループとなるようにする.
またMPI/ハイブリッド並列のときにライブラリに渡すコミュニケーター変数を,
次のようにC/C++のものからfortranのものに変換する。

\begin{sphinxVerbatim}[commandchars=\\\{\}]
\PYG{n}{comm\PYGZus{}f}\PYG{+w}{ }\PYG{o}{=}\PYG{+w}{ }\PYG{n}{MPI\PYGZus{}Comm\PYGZus{}c2f}\PYG{p}{(}\PYG{n}{comm\PYGZus{}c}\PYG{p}{)}\PYG{p}{;}
\end{sphinxVerbatim}


\section{全エネルギー, 電荷密度等(占有率の計算)}
\label{\detokenize{routine:id2}}\begin{equation*}
\begin{split}\begin{align}
\sum_{n}
\int_{\rm BZ} \frac{d^3 k}{V_{\rm BZ}}
\theta(\varepsilon_{\rm F} -
\varepsilon_{n k}) X_{n k}
\end{align}\end{split}
\end{equation*}
\begin{sphinxVerbatim}[commandchars=\\\{\}]
\PYG{k}{CALL }\PYG{n}{libtetrabz\PYGZus{}occ}\PYG{p}{(}\PYG{n}{ltetra}\PYG{p}{,}\PYG{n}{bvec}\PYG{p}{,}\PYG{n}{nb}\PYG{p}{,}\PYG{n}{nge}\PYG{p}{,}\PYG{n}{eig}\PYG{p}{,}\PYG{n}{ngw}\PYG{p}{,}\PYG{n}{wght}\PYG{p}{,}\PYG{n}{comm}\PYG{p}{)}
\end{sphinxVerbatim}

\sphinxAtStartPar
パラメーター
\begin{quote}

\begin{sphinxVerbatim}[commandchars=\\\{\}]
\PYG{k+kt}{INTEGER}\PYG{p}{,}\PYG{k}{INTENT}\PYG{p}{(}\PYG{n}{IN}\PYG{p}{)}\PYG{+w}{ }\PYG{k+kd}{::}\PYG{+w}{ }\PYG{n}{ltetra}
\end{sphinxVerbatim}
\begin{quote}

\sphinxAtStartPar
テトラへドロン法の種類を決める.
1 \(\cdots\) 線形テトラへドロン法,
2 \(\cdots\) 最適化線形テトラへドロン法 {\hyperref[\detokenize{ref:ref}]{\sphinxcrossref{\DUrole{std,std-ref}{{[}1{]}}}}}
\end{quote}

\begin{sphinxVerbatim}[commandchars=\\\{\}]
\PYG{k+kt}{REAL}\PYG{p}{(}\PYG{l+m+mi}{8}\PYG{p}{)}\PYG{p}{,}\PYG{k}{INTENT}\PYG{p}{(}\PYG{n}{IN}\PYG{p}{)}\PYG{+w}{ }\PYG{k+kd}{::}\PYG{+w}{ }\PYG{n}{bvec}\PYG{p}{(}\PYG{l+m+mi}{3}\PYG{p}{,}\PYG{l+m+mi}{3}\PYG{p}{)}
\end{sphinxVerbatim}
\begin{quote}

\sphinxAtStartPar
逆格子ベクトル. 単位は任意で良い.
逆格子の形によって四面体の切り方を決めるため,
それらの長さの比のみが必要であるため.
\end{quote}

\begin{sphinxVerbatim}[commandchars=\\\{\}]
\PYG{k+kt}{INTEGER}\PYG{p}{,}\PYG{k}{INTENT}\PYG{p}{(}\PYG{n}{IN}\PYG{p}{)}\PYG{+w}{ }\PYG{k+kd}{::}\PYG{+w}{ }\PYG{n}{nb}
\end{sphinxVerbatim}
\begin{quote}

\sphinxAtStartPar
バンド本数
\end{quote}

\begin{sphinxVerbatim}[commandchars=\\\{\}]
\PYG{k+kt}{INTEGER}\PYG{p}{,}\PYG{k}{INTENT}\PYG{p}{(}\PYG{n}{IN}\PYG{p}{)}\PYG{+w}{ }\PYG{k+kd}{::}\PYG{+w}{ }\PYG{n}{nge}\PYG{p}{(}\PYG{l+m+mi}{3}\PYG{p}{)}
\end{sphinxVerbatim}
\begin{quote}

\sphinxAtStartPar
軌道エネルギーのメッシュ数.
\end{quote}

\begin{sphinxVerbatim}[commandchars=\\\{\}]
\PYG{k+kt}{REAL}\PYG{p}{(}\PYG{l+m+mi}{8}\PYG{p}{)}\PYG{p}{,}\PYG{k}{INTENT}\PYG{p}{(}\PYG{n}{IN}\PYG{p}{)}\PYG{+w}{ }\PYG{k+kd}{::}\PYG{+w}{ }\PYG{n}{eig}\PYG{p}{(}\PYG{n}{nb}\PYG{p}{,}\PYG{n}{nge}\PYG{p}{(}\PYG{l+m+mi}{1}\PYG{p}{)}\PYG{p}{,}\PYG{n}{nge}\PYG{p}{(}\PYG{l+m+mi}{2}\PYG{p}{)}\PYG{p}{,}\PYG{n}{nge}\PYG{p}{(}\PYG{l+m+mi}{3}\PYG{p}{)}\PYG{p}{)}
\end{sphinxVerbatim}
\begin{quote}

\sphinxAtStartPar
軌道エネルギー.
Fermiエネルギーを基準とすること( \(\varepsilon_{\rm F} = 0\) ).
\end{quote}

\begin{sphinxVerbatim}[commandchars=\\\{\}]
\PYG{k+kt}{INTEGER}\PYG{p}{,}\PYG{k}{INTENT}\PYG{p}{(}\PYG{n}{IN}\PYG{p}{)}\PYG{+w}{ }\PYG{k+kd}{::}\PYG{+w}{ }\PYG{n}{ngw}\PYG{p}{(}\PYG{l+m+mi}{3}\PYG{p}{)}
\end{sphinxVerbatim}
\begin{quote}

\sphinxAtStartPar
\sphinxcode{\sphinxupquote{ngw(3)}} : (入力, 整数配列) 積分重みの \(k\) メッシュ.
\sphinxcode{\sphinxupquote{nge}} と違っていても構わない({\hyperref[\detokenize{app:app}]{\sphinxcrossref{\DUrole{std,std-ref}{補遺}}}} 参照).
\end{quote}

\begin{sphinxVerbatim}[commandchars=\\\{\}]
\PYG{k+kt}{REAL}\PYG{p}{(}\PYG{l+m+mi}{8}\PYG{p}{)}\PYG{p}{,}\PYG{k}{INTENT}\PYG{p}{(}\PYG{n}{OUT}\PYG{p}{)}\PYG{+w}{ }\PYG{k+kd}{::}\PYG{+w}{ }\PYG{n}{wght}\PYG{p}{(}\PYG{n}{nb}\PYG{p}{,}\PYG{n}{ngw}\PYG{p}{(}\PYG{l+m+mi}{1}\PYG{p}{)}\PYG{p}{,}\PYG{n}{ngw}\PYG{p}{(}\PYG{l+m+mi}{2}\PYG{p}{)}\PYG{p}{,}\PYG{n}{ngw}\PYG{p}{(}\PYG{l+m+mi}{3}\PYG{p}{)}\PYG{p}{)}
\end{sphinxVerbatim}
\begin{quote}

\sphinxAtStartPar
\sphinxcode{\sphinxupquote{wght(nb,ngw(1),ngw(2),ngw(3))}} : (出力, 実数配列) 積分重み
\end{quote}

\begin{sphinxVerbatim}[commandchars=\\\{\}]
\PYG{k+kt}{INTEGER}\PYG{p}{,}\PYG{k}{INTENT}\PYG{p}{(}\PYG{n}{IN}\PYG{p}{)}\PYG{p}{,}\PYG{k}{OPTIONAL}\PYG{+w}{ }\PYG{k+kd}{::}\PYG{+w}{ }\PYG{n}{comm}
\end{sphinxVerbatim}
\begin{quote}

\sphinxAtStartPar
オプショナル引数.
MPIのコミニュケーター( \sphinxcode{\sphinxupquote{MPI\_COMM\_WORLD}} など)を入れる.
libtetrabz を内部でMPI/Hybrid並列するときのみ入力する.
C言語では使用しないときには \sphinxcode{\sphinxupquote{NULL}} を入れる.
\end{quote}
\end{quote}


\section{Fermi エネルギー(占有率も同時に計算する)}
\label{\detokenize{routine:fermi}}\begin{equation*}
\begin{split}\begin{align}
\sum_{n}
\int_{\rm BZ} \frac{d^3 k}{V_{\rm BZ}}
\theta(\varepsilon_{\rm F} -
\varepsilon_{n k}) X_{n k}
\end{align}\end{split}
\end{equation*}
\begin{sphinxVerbatim}[commandchars=\\\{\}]
\PYG{k}{CALL }\PYG{n}{libtetrabz\PYGZus{}fermieng}\PYG{p}{(}\PYG{n}{ltetra}\PYG{p}{,}\PYG{n}{bvec}\PYG{p}{,}\PYG{n}{nb}\PYG{p}{,}\PYG{n}{nge}\PYG{p}{,}\PYG{n}{eig}\PYG{p}{,}\PYG{n}{ngw}\PYG{p}{,}\PYG{n}{wght}\PYG{p}{,}\PYG{n}{ef}\PYG{p}{,}\PYG{n}{nelec}\PYG{p}{,}\PYG{n}{comm}\PYG{p}{)}
\end{sphinxVerbatim}

\sphinxAtStartPar
パラメーター
\begin{quote}

\begin{sphinxVerbatim}[commandchars=\\\{\}]
\PYG{k+kt}{INTEGER}\PYG{p}{,}\PYG{k}{INTENT}\PYG{p}{(}\PYG{n}{IN}\PYG{p}{)}\PYG{+w}{ }\PYG{k+kd}{::}\PYG{+w}{ }\PYG{n}{ltetra}
\end{sphinxVerbatim}
\begin{quote}

\sphinxAtStartPar
テトラへドロン法の種類を決める.
1 \(\cdots\) 線形テトラへドロン法,
2 \(\cdots\) 最適化線形テトラへドロン法 {\hyperref[\detokenize{ref:ref}]{\sphinxcrossref{\DUrole{std,std-ref}{{[}1{]}}}}}
\end{quote}

\begin{sphinxVerbatim}[commandchars=\\\{\}]
\PYG{k+kt}{REAL}\PYG{p}{(}\PYG{l+m+mi}{8}\PYG{p}{)}\PYG{p}{,}\PYG{k}{INTENT}\PYG{p}{(}\PYG{n}{IN}\PYG{p}{)}\PYG{+w}{ }\PYG{k+kd}{::}\PYG{+w}{ }\PYG{n}{bvec}\PYG{p}{(}\PYG{l+m+mi}{3}\PYG{p}{,}\PYG{l+m+mi}{3}\PYG{p}{)}
\end{sphinxVerbatim}
\begin{quote}

\sphinxAtStartPar
逆格子ベクトル. 単位は任意で良い.
逆格子の形によって四面体の切り方を決めるため,
それらの長さの比のみが必要であるため.
\end{quote}

\begin{sphinxVerbatim}[commandchars=\\\{\}]
\PYG{k+kt}{INTEGER}\PYG{p}{,}\PYG{k}{INTENT}\PYG{p}{(}\PYG{n}{IN}\PYG{p}{)}\PYG{+w}{ }\PYG{k+kd}{::}\PYG{+w}{ }\PYG{n}{nb}
\end{sphinxVerbatim}
\begin{quote}

\sphinxAtStartPar
バンド本数
\end{quote}

\begin{sphinxVerbatim}[commandchars=\\\{\}]
\PYG{k+kt}{INTEGER}\PYG{p}{,}\PYG{k}{INTENT}\PYG{p}{(}\PYG{n}{IN}\PYG{p}{)}\PYG{+w}{ }\PYG{k+kd}{::}\PYG{+w}{ }\PYG{n}{nge}\PYG{p}{(}\PYG{l+m+mi}{3}\PYG{p}{)}
\end{sphinxVerbatim}
\begin{quote}

\sphinxAtStartPar
軌道エネルギーのメッシュ数.
\end{quote}

\begin{sphinxVerbatim}[commandchars=\\\{\}]
\PYG{k+kt}{REAL}\PYG{p}{(}\PYG{l+m+mi}{8}\PYG{p}{)}\PYG{p}{,}\PYG{k}{INTENT}\PYG{p}{(}\PYG{n}{IN}\PYG{p}{)}\PYG{+w}{ }\PYG{k+kd}{::}\PYG{+w}{ }\PYG{n}{eig}\PYG{p}{(}\PYG{n}{nb}\PYG{p}{,}\PYG{n}{nge}\PYG{p}{(}\PYG{l+m+mi}{1}\PYG{p}{)}\PYG{p}{,}\PYG{n}{nge}\PYG{p}{(}\PYG{l+m+mi}{2}\PYG{p}{)}\PYG{p}{,}\PYG{n}{nge}\PYG{p}{(}\PYG{l+m+mi}{3}\PYG{p}{)}\PYG{p}{)}
\end{sphinxVerbatim}
\begin{quote}

\sphinxAtStartPar
軌道エネルギー.
\end{quote}

\begin{sphinxVerbatim}[commandchars=\\\{\}]
\PYG{k+kt}{INTEGER}\PYG{p}{,}\PYG{k}{INTENT}\PYG{p}{(}\PYG{n}{IN}\PYG{p}{)}\PYG{+w}{ }\PYG{k+kd}{::}\PYG{+w}{ }\PYG{n}{nge}\PYG{p}{(}\PYG{l+m+mi}{3}\PYG{p}{)}
\end{sphinxVerbatim}
\begin{quote}

\sphinxAtStartPar
軌道エネルギーのメッシュ数.
\end{quote}

\begin{sphinxVerbatim}[commandchars=\\\{\}]
\PYG{k+kt}{INTEGER}\PYG{p}{,}\PYG{k}{INTENT}\PYG{p}{(}\PYG{n}{IN}\PYG{p}{)}\PYG{+w}{ }\PYG{k+kd}{::}\PYG{+w}{ }\PYG{n}{ngw}\PYG{p}{(}\PYG{l+m+mi}{3}\PYG{p}{)}
\end{sphinxVerbatim}
\begin{quote}

\sphinxAtStartPar
積分重みの \(k\) メッシュ.
\sphinxcode{\sphinxupquote{nge}} と違っていても構わない({\hyperref[\detokenize{app:app}]{\sphinxcrossref{\DUrole{std,std-ref}{補遺}}}} 参照).
\end{quote}

\begin{sphinxVerbatim}[commandchars=\\\{\}]
\PYG{k+kt}{REAL}\PYG{p}{(}\PYG{l+m+mi}{8}\PYG{p}{)}\PYG{p}{,}\PYG{k}{INTENT}\PYG{p}{(}\PYG{n}{OUT}\PYG{p}{)}\PYG{+w}{ }\PYG{k+kd}{::}\PYG{+w}{ }\PYG{n}{wght}\PYG{p}{(}\PYG{n}{nb}\PYG{p}{,}\PYG{n}{ngw}\PYG{p}{(}\PYG{l+m+mi}{1}\PYG{p}{)}\PYG{p}{,}\PYG{n}{ngw}\PYG{p}{(}\PYG{l+m+mi}{2}\PYG{p}{)}\PYG{p}{,}\PYG{n}{ngw}\PYG{p}{(}\PYG{l+m+mi}{3}\PYG{p}{)}\PYG{p}{)}
\end{sphinxVerbatim}
\begin{quote}

\sphinxAtStartPar
積分重み
\end{quote}

\begin{sphinxVerbatim}[commandchars=\\\{\}]
\PYG{k+kt}{REAL}\PYG{p}{(}\PYG{l+m+mi}{8}\PYG{p}{)}\PYG{p}{,}\PYG{k}{INTENT}\PYG{p}{(}\PYG{n}{OUT}\PYG{p}{)}\PYG{+w}{ }\PYG{k+kd}{::}\PYG{+w}{ }\PYG{n}{ef}
\end{sphinxVerbatim}
\begin{quote}

\sphinxAtStartPar
Fermi エネルギー
\end{quote}

\begin{sphinxVerbatim}[commandchars=\\\{\}]
\PYG{k+kt}{REAL}\PYG{p}{(}\PYG{l+m+mi}{8}\PYG{p}{)}\PYG{p}{,}\PYG{k}{INTENT}\PYG{p}{(}\PYG{n}{IN}\PYG{p}{)}\PYG{+w}{ }\PYG{k+kd}{::}\PYG{+w}{ }\PYG{n}{nelec}
\end{sphinxVerbatim}
\begin{quote}

\sphinxAtStartPar
スピンあたりの(荷)電子数
\end{quote}

\begin{sphinxVerbatim}[commandchars=\\\{\}]
\PYG{k+kt}{INTEGER}\PYG{p}{,}\PYG{k}{INTENT}\PYG{p}{(}\PYG{n}{IN}\PYG{p}{)}\PYG{p}{,}\PYG{k}{OPTIONAL}\PYG{+w}{ }\PYG{k+kd}{::}\PYG{+w}{ }\PYG{n}{comm}
\end{sphinxVerbatim}
\begin{quote}

\sphinxAtStartPar
オプショナル引数.
MPIのコミニュケーター( \sphinxcode{\sphinxupquote{MPI\_COMM\_WORLD}} など)を入れる.
libtetrabz を内部でMPI/Hybrid並列するときのみ入力する.
C言語では使用しないときには \sphinxcode{\sphinxupquote{NULL}} を入れる.
\end{quote}
\end{quote}


\section{(部分)状態密度}
\label{\detokenize{routine:id3}}\begin{equation*}
\begin{split}\begin{align}
\sum_{n}
\int_{\rm BZ} \frac{d^3 k}{V_{\rm BZ}}
\delta(\omega - \varepsilon_{n k})
X_{n k}(\omega)
\end{align}\end{split}
\end{equation*}
\begin{sphinxVerbatim}[commandchars=\\\{\}]
\PYG{k}{CALL }\PYG{n}{libtetrabz\PYGZus{}dos}\PYG{p}{(}\PYG{n}{ltetra}\PYG{p}{,}\PYG{n}{bvec}\PYG{p}{,}\PYG{n}{nb}\PYG{p}{,}\PYG{n}{nge}\PYG{p}{,}\PYG{n}{eig}\PYG{p}{,}\PYG{n}{ngw}\PYG{p}{,}\PYG{n}{wght}\PYG{p}{,}\PYG{n}{ne}\PYG{p}{,}\PYG{n}{e0}\PYG{p}{,}\PYG{n}{comm}\PYG{p}{)}
\end{sphinxVerbatim}

\sphinxAtStartPar
パラメーター
\begin{quote}

\begin{sphinxVerbatim}[commandchars=\\\{\}]
\PYG{k+kt}{INTEGER}\PYG{p}{,}\PYG{k}{INTENT}\PYG{p}{(}\PYG{n}{IN}\PYG{p}{)}\PYG{+w}{ }\PYG{k+kd}{::}\PYG{+w}{ }\PYG{n}{ltetra}
\end{sphinxVerbatim}
\begin{quote}

\sphinxAtStartPar
テトラへドロン法の種類を決める.
1 \(\cdots\) 線形テトラへドロン法,
2 \(\cdots\) 最適化線形テトラへドロン法 {\hyperref[\detokenize{ref:ref}]{\sphinxcrossref{\DUrole{std,std-ref}{{[}1{]}}}}}
\end{quote}

\begin{sphinxVerbatim}[commandchars=\\\{\}]
\PYG{k+kt}{REAL}\PYG{p}{(}\PYG{l+m+mi}{8}\PYG{p}{)}\PYG{p}{,}\PYG{k}{INTENT}\PYG{p}{(}\PYG{n}{IN}\PYG{p}{)}\PYG{+w}{ }\PYG{k+kd}{::}\PYG{+w}{ }\PYG{n}{bvec}\PYG{p}{(}\PYG{l+m+mi}{3}\PYG{p}{,}\PYG{l+m+mi}{3}\PYG{p}{)}
\end{sphinxVerbatim}
\begin{quote}

\sphinxAtStartPar
逆格子ベクトル. 単位は任意で良い.
逆格子の形によって四面体の切り方を決めるため,
それらの長さの比のみが必要であるため.
\end{quote}

\begin{sphinxVerbatim}[commandchars=\\\{\}]
\PYG{k+kt}{INTEGER}\PYG{p}{,}\PYG{k}{INTENT}\PYG{p}{(}\PYG{n}{IN}\PYG{p}{)}\PYG{+w}{ }\PYG{k+kd}{::}\PYG{+w}{ }\PYG{n}{nb}
\end{sphinxVerbatim}
\begin{quote}

\sphinxAtStartPar
バンド本数
\end{quote}

\begin{sphinxVerbatim}[commandchars=\\\{\}]
\PYG{k+kt}{INTEGER}\PYG{p}{,}\PYG{k}{INTENT}\PYG{p}{(}\PYG{n}{IN}\PYG{p}{)}\PYG{+w}{ }\PYG{k+kd}{::}\PYG{+w}{ }\PYG{n}{nge}\PYG{p}{(}\PYG{l+m+mi}{3}\PYG{p}{)}
\end{sphinxVerbatim}
\begin{quote}

\sphinxAtStartPar
軌道エネルギーの \(k\) メッシュ数.
\end{quote}

\begin{sphinxVerbatim}[commandchars=\\\{\}]
\PYG{k+kt}{REAL}\PYG{p}{(}\PYG{l+m+mi}{8}\PYG{p}{)}\PYG{p}{,}\PYG{k}{INTENT}\PYG{p}{(}\PYG{n}{IN}\PYG{p}{)}\PYG{+w}{ }\PYG{k+kd}{::}\PYG{+w}{ }\PYG{n}{eig}\PYG{p}{(}\PYG{n}{nb}\PYG{p}{,}\PYG{n}{nge}\PYG{p}{(}\PYG{l+m+mi}{1}\PYG{p}{)}\PYG{p}{,}\PYG{n}{nge}\PYG{p}{(}\PYG{l+m+mi}{2}\PYG{p}{)}\PYG{p}{,}\PYG{n}{nge}\PYG{p}{(}\PYG{l+m+mi}{3}\PYG{p}{)}\PYG{p}{)}
\end{sphinxVerbatim}
\begin{quote}

\sphinxAtStartPar
軌道エネルギー.
\end{quote}

\begin{sphinxVerbatim}[commandchars=\\\{\}]
\PYG{k+kt}{INTEGER}\PYG{p}{,}\PYG{k}{INTENT}\PYG{p}{(}\PYG{n}{IN}\PYG{p}{)}\PYG{+w}{ }\PYG{k+kd}{::}\PYG{+w}{ }\PYG{n}{ngw}\PYG{p}{(}\PYG{l+m+mi}{3}\PYG{p}{)}
\end{sphinxVerbatim}
\begin{quote}

\sphinxAtStartPar
積分重みの \(k\) メッシュ.
\sphinxcode{\sphinxupquote{nge}} と違っていても構わない({\hyperref[\detokenize{app:app}]{\sphinxcrossref{\DUrole{std,std-ref}{補遺}}}} 参照).
\end{quote}

\begin{sphinxVerbatim}[commandchars=\\\{\}]
\PYG{k+kt}{REAL}\PYG{p}{(}\PYG{l+m+mi}{8}\PYG{p}{)}\PYG{p}{,}\PYG{k}{INTENT}\PYG{p}{(}\PYG{n}{OUT}\PYG{p}{)}\PYG{+w}{ }\PYG{k+kd}{::}\PYG{+w}{ }\PYG{n}{wght}\PYG{p}{(}\PYG{n}{ne}\PYG{p}{,}\PYG{n}{nb}\PYG{p}{,}\PYG{n}{ngw}\PYG{p}{(}\PYG{l+m+mi}{1}\PYG{p}{)}\PYG{p}{,}\PYG{n}{ngw}\PYG{p}{(}\PYG{l+m+mi}{2}\PYG{p}{)}\PYG{p}{,}\PYG{n}{ngw}\PYG{p}{(}\PYG{l+m+mi}{3}\PYG{p}{)}\PYG{p}{)}
\end{sphinxVerbatim}
\begin{quote}

\sphinxAtStartPar
積分重み
\end{quote}

\begin{sphinxVerbatim}[commandchars=\\\{\}]
\PYG{k+kt}{INTEGER}\PYG{p}{,}\PYG{k}{INTENT}\PYG{p}{(}\PYG{n}{IN}\PYG{p}{)}\PYG{+w}{ }\PYG{k+kd}{::}\PYG{+w}{ }\PYG{n}{ne}
\end{sphinxVerbatim}
\begin{quote}

\sphinxAtStartPar
状態密度を計算するエネルギー点数
\end{quote}

\begin{sphinxVerbatim}[commandchars=\\\{\}]
\PYG{k+kt}{REAL}\PYG{p}{(}\PYG{l+m+mi}{8}\PYG{p}{)}\PYG{p}{,}\PYG{k}{INTENT}\PYG{p}{(}\PYG{n}{IN}\PYG{p}{)}\PYG{+w}{ }\PYG{k+kd}{::}\PYG{+w}{ }\PYG{n}{e0}\PYG{p}{(}\PYG{n}{ne}\PYG{p}{)}
\end{sphinxVerbatim}
\begin{quote}

\sphinxAtStartPar
状態密度を計算するエネルギー
\end{quote}

\begin{sphinxVerbatim}[commandchars=\\\{\}]
\PYG{k+kt}{INTEGER}\PYG{p}{,}\PYG{k}{INTENT}\PYG{p}{(}\PYG{n}{IN}\PYG{p}{)}\PYG{p}{,}\PYG{k}{OPTIONAL}\PYG{+w}{ }\PYG{k+kd}{::}\PYG{+w}{ }\PYG{n}{comm}
\end{sphinxVerbatim}
\begin{quote}

\sphinxAtStartPar
オプショナル引数.
MPIのコミニュケーター( \sphinxcode{\sphinxupquote{MPI\_COMM\_WORLD}} など)を入れる.
libtetrabz を内部でMPI/Hybrid並列するときのみ入力する.
C言語では使用しないときには \sphinxcode{\sphinxupquote{NULL}} を入れる.
\end{quote}
\end{quote}


\section{積分状態密度}
\label{\detokenize{routine:id4}}\begin{equation*}
\begin{split}\begin{align}
\sum_{n}
\int_{\rm BZ} \frac{d^3 k}{V_{\rm BZ}}
\theta(\omega - \varepsilon_{n k})
X_{n k}(\omega)
\end{align}\end{split}
\end{equation*}
\begin{sphinxVerbatim}[commandchars=\\\{\}]
\PYG{k}{CALL }\PYG{n}{libtetrabz\PYGZus{}intdos}\PYG{p}{(}\PYG{n}{ltetra}\PYG{p}{,}\PYG{n}{bvec}\PYG{p}{,}\PYG{n}{nb}\PYG{p}{,}\PYG{n}{nge}\PYG{p}{,}\PYG{n}{eig}\PYG{p}{,}\PYG{n}{ngw}\PYG{p}{,}\PYG{n}{wght}\PYG{p}{,}\PYG{n}{ne}\PYG{p}{,}\PYG{n}{e0}\PYG{p}{,}\PYG{n}{comm}\PYG{p}{)}
\end{sphinxVerbatim}

\sphinxAtStartPar
パラメーター
\begin{quote}

\begin{sphinxVerbatim}[commandchars=\\\{\}]
\PYG{k+kt}{INTEGER}\PYG{p}{,}\PYG{k}{INTENT}\PYG{p}{(}\PYG{n}{IN}\PYG{p}{)}\PYG{+w}{ }\PYG{k+kd}{::}\PYG{+w}{ }\PYG{n}{ltetra}
\end{sphinxVerbatim}
\begin{quote}

\sphinxAtStartPar
テトラへドロン法の種類を決める.
1 \(\cdots\) 線形テトラへドロン法,
2 \(\cdots\) 最適化線形テトラへドロン法 {\hyperref[\detokenize{ref:ref}]{\sphinxcrossref{\DUrole{std,std-ref}{{[}1{]}}}}}
\end{quote}

\begin{sphinxVerbatim}[commandchars=\\\{\}]
\PYG{k+kt}{REAL}\PYG{p}{(}\PYG{l+m+mi}{8}\PYG{p}{)}\PYG{p}{,}\PYG{k}{INTENT}\PYG{p}{(}\PYG{n}{IN}\PYG{p}{)}\PYG{+w}{ }\PYG{k+kd}{::}\PYG{+w}{ }\PYG{n}{bvec}\PYG{p}{(}\PYG{l+m+mi}{3}\PYG{p}{,}\PYG{l+m+mi}{3}\PYG{p}{)}
\end{sphinxVerbatim}
\begin{quote}

\sphinxAtStartPar
逆格子ベクトル. 単位は任意で良い.
逆格子の形によって四面体の切り方を決めるため,
それらの長さの比のみが必要であるため.
\end{quote}

\begin{sphinxVerbatim}[commandchars=\\\{\}]
\PYG{k+kt}{INTEGER}\PYG{p}{,}\PYG{k}{INTENT}\PYG{p}{(}\PYG{n}{IN}\PYG{p}{)}\PYG{+w}{ }\PYG{k+kd}{::}\PYG{+w}{ }\PYG{n}{nb}
\end{sphinxVerbatim}
\begin{quote}

\sphinxAtStartPar
バンド本数
\end{quote}

\begin{sphinxVerbatim}[commandchars=\\\{\}]
\PYG{k+kt}{INTEGER}\PYG{p}{,}\PYG{k}{INTENT}\PYG{p}{(}\PYG{n}{IN}\PYG{p}{)}\PYG{+w}{ }\PYG{k+kd}{::}\PYG{+w}{ }\PYG{n}{nge}\PYG{p}{(}\PYG{l+m+mi}{3}\PYG{p}{)}
\end{sphinxVerbatim}
\begin{quote}

\sphinxAtStartPar
軌道エネルギーの \(k\) メッシュ数.
\end{quote}

\begin{sphinxVerbatim}[commandchars=\\\{\}]
\PYG{k+kt}{REAL}\PYG{p}{(}\PYG{l+m+mi}{8}\PYG{p}{)}\PYG{p}{,}\PYG{k}{INTENT}\PYG{p}{(}\PYG{n}{IN}\PYG{p}{)}\PYG{+w}{ }\PYG{k+kd}{::}\PYG{+w}{ }\PYG{n}{eig}\PYG{p}{(}\PYG{n}{nb}\PYG{p}{,}\PYG{n}{nge}\PYG{p}{(}\PYG{l+m+mi}{1}\PYG{p}{)}\PYG{p}{,}\PYG{n}{nge}\PYG{p}{(}\PYG{l+m+mi}{2}\PYG{p}{)}\PYG{p}{,}\PYG{n}{nge}\PYG{p}{(}\PYG{l+m+mi}{3}\PYG{p}{)}\PYG{p}{)}
\end{sphinxVerbatim}
\begin{quote}

\sphinxAtStartPar
軌道エネルギー.
\end{quote}

\begin{sphinxVerbatim}[commandchars=\\\{\}]
\PYG{k+kt}{INTEGER}\PYG{p}{,}\PYG{k}{INTENT}\PYG{p}{(}\PYG{n}{IN}\PYG{p}{)}\PYG{+w}{ }\PYG{k+kd}{::}\PYG{+w}{ }\PYG{n}{ngw}\PYG{p}{(}\PYG{l+m+mi}{3}\PYG{p}{)}
\end{sphinxVerbatim}
\begin{quote}

\sphinxAtStartPar
積分重みの \(k\) メッシュ.
\sphinxcode{\sphinxupquote{nge}} と違っていても構わない({\hyperref[\detokenize{app:app}]{\sphinxcrossref{\DUrole{std,std-ref}{補遺}}}} 参照).
\end{quote}

\begin{sphinxVerbatim}[commandchars=\\\{\}]
\PYG{k+kt}{REAL}\PYG{p}{(}\PYG{l+m+mi}{8}\PYG{p}{)}\PYG{p}{,}\PYG{k}{INTENT}\PYG{p}{(}\PYG{n}{OUT}\PYG{p}{)}\PYG{+w}{ }\PYG{k+kd}{::}\PYG{+w}{ }\PYG{n}{wght}\PYG{p}{(}\PYG{n}{ne}\PYG{p}{,}\PYG{n}{nb}\PYG{p}{,}\PYG{n}{ngw}\PYG{p}{(}\PYG{l+m+mi}{1}\PYG{p}{)}\PYG{p}{,}\PYG{n}{ngw}\PYG{p}{(}\PYG{l+m+mi}{2}\PYG{p}{)}\PYG{p}{,}\PYG{n}{ngw}\PYG{p}{(}\PYG{l+m+mi}{3}\PYG{p}{)}\PYG{p}{)}
\end{sphinxVerbatim}
\begin{quote}

\sphinxAtStartPar
積分重み
\end{quote}

\begin{sphinxVerbatim}[commandchars=\\\{\}]
\PYG{k+kt}{INTEGER}\PYG{p}{,}\PYG{k}{INTENT}\PYG{p}{(}\PYG{n}{IN}\PYG{p}{)}\PYG{+w}{ }\PYG{k+kd}{::}\PYG{+w}{ }\PYG{n}{ne}
\end{sphinxVerbatim}
\begin{quote}

\sphinxAtStartPar
状態密度を計算するエネルギー点数
\end{quote}

\begin{sphinxVerbatim}[commandchars=\\\{\}]
\PYG{k+kt}{REAL}\PYG{p}{(}\PYG{l+m+mi}{8}\PYG{p}{)}\PYG{p}{,}\PYG{k}{INTENT}\PYG{p}{(}\PYG{n}{IN}\PYG{p}{)}\PYG{+w}{ }\PYG{k+kd}{::}\PYG{+w}{ }\PYG{n}{e0}\PYG{p}{(}\PYG{n}{ne}\PYG{p}{)}
\end{sphinxVerbatim}
\begin{quote}

\sphinxAtStartPar
状態密度を計算するエネルギー
\end{quote}

\begin{sphinxVerbatim}[commandchars=\\\{\}]
\PYG{k+kt}{INTEGER}\PYG{p}{,}\PYG{k}{INTENT}\PYG{p}{(}\PYG{n}{IN}\PYG{p}{)}\PYG{p}{,}\PYG{k}{OPTIONAL}\PYG{+w}{ }\PYG{k+kd}{::}\PYG{+w}{ }\PYG{n}{comm}
\end{sphinxVerbatim}
\begin{quote}

\sphinxAtStartPar
オプショナル引数.
MPIのコミニュケーター( \sphinxcode{\sphinxupquote{MPI\_COMM\_WORLD}} など)を入れる.
libtetrabz を内部でMPI/Hybrid並列するときのみ入力する.
C言語では使用しないときには \sphinxcode{\sphinxupquote{NULL}} を入れる.
\end{quote}
\end{quote}


\section{ネスティング関数, Fröhlich パラメーター}
\label{\detokenize{routine:frohlich}}\begin{equation*}
\begin{split}\begin{align}
\sum_{n n'}
\int_{\rm BZ} \frac{d^3 k}{V_{\rm BZ}}
\delta(\varepsilon_{\rm F} -
\varepsilon_{n k}) \delta(\varepsilon_{\rm F} - \varepsilon'_{n' k})
X_{n n' k}
\end{align}\end{split}
\end{equation*}
\begin{sphinxVerbatim}[commandchars=\\\{\}]
\PYG{k}{CALL }\PYG{n}{libtetrabz\PYGZus{}dbldelta}\PYG{p}{(}\PYG{n}{ltetra}\PYG{p}{,}\PYG{n}{bvec}\PYG{p}{,}\PYG{n}{nb}\PYG{p}{,}\PYG{n}{nge}\PYG{p}{,}\PYG{n}{eig1}\PYG{p}{,}\PYG{n}{eig2}\PYG{p}{,}\PYG{n}{ngw}\PYG{p}{,}\PYG{n}{wght}\PYG{p}{,}\PYG{n}{comm}\PYG{p}{)}
\end{sphinxVerbatim}

\sphinxAtStartPar
パラメーター
\begin{quote}

\begin{sphinxVerbatim}[commandchars=\\\{\}]
\PYG{k+kt}{INTEGER}\PYG{p}{,}\PYG{k}{INTENT}\PYG{p}{(}\PYG{n}{IN}\PYG{p}{)}\PYG{+w}{ }\PYG{k+kd}{::}\PYG{+w}{ }\PYG{n}{ltetra}
\end{sphinxVerbatim}
\begin{quote}

\sphinxAtStartPar
テトラへドロン法の種類を決める.
1 \(\cdots\) 線形テトラへドロン法,
2 \(\cdots\) 最適化線形テトラへドロン法 {\hyperref[\detokenize{ref:ref}]{\sphinxcrossref{\DUrole{std,std-ref}{{[}1{]}}}}}
\end{quote}

\begin{sphinxVerbatim}[commandchars=\\\{\}]
\PYG{k+kt}{REAL}\PYG{p}{(}\PYG{l+m+mi}{8}\PYG{p}{)}\PYG{p}{,}\PYG{k}{INTENT}\PYG{p}{(}\PYG{n}{IN}\PYG{p}{)}\PYG{+w}{ }\PYG{k+kd}{::}\PYG{+w}{ }\PYG{n}{bvec}\PYG{p}{(}\PYG{l+m+mi}{3}\PYG{p}{,}\PYG{l+m+mi}{3}\PYG{p}{)}
\end{sphinxVerbatim}
\begin{quote}

\sphinxAtStartPar
逆格子ベクトル. 単位は任意で良い.
逆格子の形によって四面体の切り方を決めるため,
それらの長さの比のみが必要であるため.
\end{quote}

\begin{sphinxVerbatim}[commandchars=\\\{\}]
\PYG{k+kt}{INTEGER}\PYG{p}{,}\PYG{k}{INTENT}\PYG{p}{(}\PYG{n}{IN}\PYG{p}{)}\PYG{+w}{ }\PYG{k+kd}{::}\PYG{+w}{ }\PYG{n}{nb}
\end{sphinxVerbatim}
\begin{quote}

\sphinxAtStartPar
バンド本数
\end{quote}

\begin{sphinxVerbatim}[commandchars=\\\{\}]
\PYG{k+kt}{INTEGER}\PYG{p}{,}\PYG{k}{INTENT}\PYG{p}{(}\PYG{n}{IN}\PYG{p}{)}\PYG{+w}{ }\PYG{k+kd}{::}\PYG{+w}{ }\PYG{n}{nge}\PYG{p}{(}\PYG{l+m+mi}{3}\PYG{p}{)}
\end{sphinxVerbatim}
\begin{quote}

\sphinxAtStartPar
軌道エネルギーの \(k\) メッシュ数.
\end{quote}

\begin{sphinxVerbatim}[commandchars=\\\{\}]
\PYG{k+kt}{REAL}\PYG{p}{(}\PYG{l+m+mi}{8}\PYG{p}{)}\PYG{p}{,}\PYG{k}{INTENT}\PYG{p}{(}\PYG{n}{IN}\PYG{p}{)}\PYG{+w}{ }\PYG{k+kd}{::}\PYG{+w}{ }\PYG{n}{eig1}\PYG{p}{(}\PYG{n}{nb}\PYG{p}{,}\PYG{n}{nge}\PYG{p}{(}\PYG{l+m+mi}{1}\PYG{p}{)}\PYG{p}{,}\PYG{n}{nge}\PYG{p}{(}\PYG{l+m+mi}{2}\PYG{p}{)}\PYG{p}{,}\PYG{n}{nge}\PYG{p}{(}\PYG{l+m+mi}{3}\PYG{p}{)}\PYG{p}{)}
\end{sphinxVerbatim}
\begin{quote}

\sphinxAtStartPar
軌道エネルギー.
Fermi エネルギーを基準とすること( \(\varepsilon_{\rm F}=0\) ).
\sphinxcode{\sphinxupquote{eig2}} も同様.
\end{quote}

\begin{sphinxVerbatim}[commandchars=\\\{\}]
\PYG{k+kt}{REAL}\PYG{p}{(}\PYG{l+m+mi}{8}\PYG{p}{)}\PYG{p}{,}\PYG{k}{INTENT}\PYG{p}{(}\PYG{n}{IN}\PYG{p}{)}\PYG{+w}{ }\PYG{k+kd}{::}\PYG{+w}{ }\PYG{n}{eig2}\PYG{p}{(}\PYG{n}{nb}\PYG{p}{,}\PYG{n}{nge}\PYG{p}{(}\PYG{l+m+mi}{1}\PYG{p}{)}\PYG{p}{,}\PYG{n}{nge}\PYG{p}{(}\PYG{l+m+mi}{2}\PYG{p}{)}\PYG{p}{,}\PYG{n}{nge}\PYG{p}{(}\PYG{l+m+mi}{3}\PYG{p}{)}\PYG{p}{)}
\end{sphinxVerbatim}
\begin{quote}

\sphinxAtStartPar
軌道エネルギー.
移行運動量の分だけグリッドをずらしたものなど.
\end{quote}

\begin{sphinxVerbatim}[commandchars=\\\{\}]
\PYG{k+kt}{INTEGER}\PYG{p}{,}\PYG{k}{INTENT}\PYG{p}{(}\PYG{n}{IN}\PYG{p}{)}\PYG{+w}{ }\PYG{k+kd}{::}\PYG{+w}{ }\PYG{n}{ngw}\PYG{p}{(}\PYG{l+m+mi}{3}\PYG{p}{)}
\end{sphinxVerbatim}
\begin{quote}

\sphinxAtStartPar
積分重みの \(k\) メッシュ.
\sphinxcode{\sphinxupquote{nge}} と違っていても構わない({\hyperref[\detokenize{app:app}]{\sphinxcrossref{\DUrole{std,std-ref}{補遺}}}} 参照).
\end{quote}

\begin{sphinxVerbatim}[commandchars=\\\{\}]
\PYG{k+kt}{REAL}\PYG{p}{(}\PYG{l+m+mi}{8}\PYG{p}{)}\PYG{p}{,}\PYG{k}{INTENT}\PYG{p}{(}\PYG{n}{OUT}\PYG{p}{)}\PYG{+w}{ }\PYG{k+kd}{::}\PYG{+w}{ }\PYG{n}{wght}\PYG{p}{(}\PYG{n}{nb}\PYG{p}{,}\PYG{n}{nb}\PYG{p}{,}\PYG{n}{ngw}\PYG{p}{(}\PYG{l+m+mi}{1}\PYG{p}{)}\PYG{p}{,}\PYG{n}{ngw}\PYG{p}{(}\PYG{l+m+mi}{2}\PYG{p}{)}\PYG{p}{,}\PYG{n}{ngw}\PYG{p}{(}\PYG{l+m+mi}{3}\PYG{p}{)}\PYG{p}{)}
\end{sphinxVerbatim}
\begin{quote}

\sphinxAtStartPar
積分重み
\end{quote}

\begin{sphinxVerbatim}[commandchars=\\\{\}]
\PYG{k+kt}{INTEGER}\PYG{p}{,}\PYG{k}{INTENT}\PYG{p}{(}\PYG{n}{IN}\PYG{p}{)}\PYG{p}{,}\PYG{k}{OPTIONAL}\PYG{+w}{ }\PYG{k+kd}{::}\PYG{+w}{ }\PYG{n}{comm}
\end{sphinxVerbatim}
\begin{quote}

\sphinxAtStartPar
オプショナル引数.
MPIのコミニュケーター( \sphinxcode{\sphinxupquote{MPI\_COMM\_WORLD}} など)を入れる.
libtetrabz を内部でMPI/Hybrid並列するときのみ入力する.
C言語では使用しないときには \sphinxcode{\sphinxupquote{NULL}} を入れる.
\end{quote}
\end{quote}


\section{DFPT 計算の一部}
\label{\detokenize{routine:dfpt}}\begin{equation*}
\begin{split}\begin{align}
\sum_{n n'}
\int_{\rm BZ} \frac{d^3 k}{V_{\rm BZ}}
\theta(\varepsilon_{\rm F} -
\varepsilon_{n k}) \theta(\varepsilon_{n k} - \varepsilon'_{n' k})
X_{n n' k}
\end{align}\end{split}
\end{equation*}
\begin{sphinxVerbatim}[commandchars=\\\{\}]
\PYG{k}{CALL }\PYG{n}{libtetrabz\PYGZus{}dblstep}\PYG{p}{(}\PYG{n}{ltetra}\PYG{p}{,}\PYG{n}{bvec}\PYG{p}{,}\PYG{n}{nb}\PYG{p}{,}\PYG{n}{nge}\PYG{p}{,}\PYG{n}{eig1}\PYG{p}{,}\PYG{n}{eig2}\PYG{p}{,}\PYG{n}{ngw}\PYG{p}{,}\PYG{n}{wght}\PYG{p}{,}\PYG{n}{comm}\PYG{p}{)}
\end{sphinxVerbatim}

\sphinxAtStartPar
パラメーター
\begin{quote}

\begin{sphinxVerbatim}[commandchars=\\\{\}]
\PYG{k+kt}{INTEGER}\PYG{p}{,}\PYG{k}{INTENT}\PYG{p}{(}\PYG{n}{IN}\PYG{p}{)}\PYG{+w}{ }\PYG{k+kd}{::}\PYG{+w}{ }\PYG{n}{ltetra}
\end{sphinxVerbatim}
\begin{quote}

\sphinxAtStartPar
テトラへドロン法の種類を決める.
1 \(\cdots\) 線形テトラへドロン法,
2 \(\cdots\) 最適化線形テトラへドロン法 {\hyperref[\detokenize{ref:ref}]{\sphinxcrossref{\DUrole{std,std-ref}{{[}1{]}}}}}
\end{quote}

\begin{sphinxVerbatim}[commandchars=\\\{\}]
\PYG{k+kt}{REAL}\PYG{p}{(}\PYG{l+m+mi}{8}\PYG{p}{)}\PYG{p}{,}\PYG{k}{INTENT}\PYG{p}{(}\PYG{n}{IN}\PYG{p}{)}\PYG{+w}{ }\PYG{k+kd}{::}\PYG{+w}{ }\PYG{n}{bvec}\PYG{p}{(}\PYG{l+m+mi}{3}\PYG{p}{,}\PYG{l+m+mi}{3}\PYG{p}{)}
\end{sphinxVerbatim}
\begin{quote}

\sphinxAtStartPar
逆格子ベクトル. 単位は任意で良い.
逆格子の形によって四面体の切り方を決めるため,
それらの長さの比のみが必要であるため.
\end{quote}

\begin{sphinxVerbatim}[commandchars=\\\{\}]
\PYG{k+kt}{INTEGER}\PYG{p}{,}\PYG{k}{INTENT}\PYG{p}{(}\PYG{n}{IN}\PYG{p}{)}\PYG{+w}{ }\PYG{k+kd}{::}\PYG{+w}{ }\PYG{n}{nb}
\end{sphinxVerbatim}
\begin{quote}

\sphinxAtStartPar
バンド本数
\end{quote}

\begin{sphinxVerbatim}[commandchars=\\\{\}]
\PYG{k+kt}{INTEGER}\PYG{p}{,}\PYG{k}{INTENT}\PYG{p}{(}\PYG{n}{IN}\PYG{p}{)}\PYG{+w}{ }\PYG{k+kd}{::}\PYG{+w}{ }\PYG{n}{nge}\PYG{p}{(}\PYG{l+m+mi}{3}\PYG{p}{)}
\end{sphinxVerbatim}
\begin{quote}

\sphinxAtStartPar
軌道エネルギーのメッシュ数.
\end{quote}

\begin{sphinxVerbatim}[commandchars=\\\{\}]
\PYG{k+kt}{REAL}\PYG{p}{(}\PYG{l+m+mi}{8}\PYG{p}{)}\PYG{p}{,}\PYG{k}{INTENT}\PYG{p}{(}\PYG{n}{IN}\PYG{p}{)}\PYG{+w}{ }\PYG{k+kd}{::}\PYG{+w}{ }\PYG{n}{eig1}\PYG{p}{(}\PYG{n}{nb}\PYG{p}{,}\PYG{n}{nge}\PYG{p}{(}\PYG{l+m+mi}{1}\PYG{p}{)}\PYG{p}{,}\PYG{n}{nge}\PYG{p}{(}\PYG{l+m+mi}{2}\PYG{p}{)}\PYG{p}{,}\PYG{n}{nge}\PYG{p}{(}\PYG{l+m+mi}{3}\PYG{p}{)}\PYG{p}{)}
\end{sphinxVerbatim}
\begin{quote}

\sphinxAtStartPar
軌道エネルギー.
Fermi エネルギーを基準とすること
( \(\varepsilon_{\rm F}=0\) ). \sphinxcode{\sphinxupquote{eig2}} も同様.
\end{quote}

\begin{sphinxVerbatim}[commandchars=\\\{\}]
\PYG{k+kt}{REAL}\PYG{p}{(}\PYG{l+m+mi}{8}\PYG{p}{)}\PYG{p}{,}\PYG{k}{INTENT}\PYG{p}{(}\PYG{n}{IN}\PYG{p}{)}\PYG{+w}{ }\PYG{k+kd}{::}\PYG{+w}{ }\PYG{n}{eig2}\PYG{p}{(}\PYG{n}{nb}\PYG{p}{,}\PYG{n}{nge}\PYG{p}{(}\PYG{l+m+mi}{1}\PYG{p}{)}\PYG{p}{,}\PYG{n}{nge}\PYG{p}{(}\PYG{l+m+mi}{2}\PYG{p}{)}\PYG{p}{,}\PYG{n}{nge}\PYG{p}{(}\PYG{l+m+mi}{3}\PYG{p}{)}\PYG{p}{)}
\end{sphinxVerbatim}
\begin{quote}

\sphinxAtStartPar
軌道エネルギー.
移行運動量の分だけグリッドをずらしたものなど.
\end{quote}

\begin{sphinxVerbatim}[commandchars=\\\{\}]
\PYG{k+kt}{INTEGER}\PYG{p}{,}\PYG{k}{INTENT}\PYG{p}{(}\PYG{n}{IN}\PYG{p}{)}\PYG{+w}{ }\PYG{k+kd}{::}\PYG{+w}{ }\PYG{n}{ngw}\PYG{p}{(}\PYG{l+m+mi}{3}\PYG{p}{)}
\end{sphinxVerbatim}
\begin{quote}

\sphinxAtStartPar
積分重みの \(k\) メッシュ. \sphinxcode{\sphinxupquote{nge}}
と違っていても構わない({\hyperref[\detokenize{app:app}]{\sphinxcrossref{\DUrole{std,std-ref}{補遺}}}} 参照).
\end{quote}

\begin{sphinxVerbatim}[commandchars=\\\{\}]
\PYG{k+kt}{REAL}\PYG{p}{(}\PYG{l+m+mi}{8}\PYG{p}{)}\PYG{p}{,}\PYG{k}{INTENT}\PYG{p}{(}\PYG{n}{OUT}\PYG{p}{)}\PYG{+w}{ }\PYG{k+kd}{::}\PYG{+w}{ }\PYG{n}{wght}\PYG{p}{(}\PYG{n}{nb}\PYG{p}{,}\PYG{n}{nb}\PYG{p}{,}\PYG{n}{ngw}\PYG{p}{(}\PYG{l+m+mi}{1}\PYG{p}{)}\PYG{p}{,}\PYG{n}{ngw}\PYG{p}{(}\PYG{l+m+mi}{2}\PYG{p}{)}\PYG{p}{,}\PYG{n}{ngw}\PYG{p}{(}\PYG{l+m+mi}{3}\PYG{p}{)}\PYG{p}{)}
\end{sphinxVerbatim}
\begin{quote}

\sphinxAtStartPar
積分重み
\end{quote}

\begin{sphinxVerbatim}[commandchars=\\\{\}]
\PYG{k+kt}{INTEGER}\PYG{p}{,}\PYG{k}{INTENT}\PYG{p}{(}\PYG{n}{IN}\PYG{p}{)}\PYG{p}{,}\PYG{k}{OPTIONAL}\PYG{+w}{ }\PYG{k+kd}{::}\PYG{+w}{ }\PYG{n}{comm}
\end{sphinxVerbatim}
\begin{quote}

\sphinxAtStartPar
オプショナル引数.
MPIのコミニュケーター( \sphinxcode{\sphinxupquote{MPI\_COMM\_WORLD}} など)を入れる.
libtetrabz を内部でMPI/Hybrid並列するときのみ入力する.
C言語では使用しないときには \sphinxcode{\sphinxupquote{NULL}} を入れる.
\end{quote}
\end{quote}


\section{独立分極関数(静的)}
\label{\detokenize{routine:id5}}\begin{equation*}
\begin{split}\begin{align}
\sum_{n n'}
\int_{\rm BZ} \frac{d^3 k}{V_{\rm BZ}}
\frac{\theta(\varepsilon_{\rm F} - \varepsilon_{n k})
\theta(\varepsilon'_{n' k} - \varepsilon_{\rm F})}
{\varepsilon'_{n' k} - \varepsilon_{n k}}
X_{n n' k}
\end{align}\end{split}
\end{equation*}
\begin{sphinxVerbatim}[commandchars=\\\{\}]
\PYG{k}{CALL }\PYG{n}{libtetrabz\PYGZus{}polstat}\PYG{p}{(}\PYG{n}{ltetra}\PYG{p}{,}\PYG{n}{bvec}\PYG{p}{,}\PYG{n}{nb}\PYG{p}{,}\PYG{n}{nge}\PYG{p}{,}\PYG{n}{eig1}\PYG{p}{,}\PYG{n}{eig2}\PYG{p}{,}\PYG{n}{ngw}\PYG{p}{,}\PYG{n}{wght}\PYG{p}{,}\PYG{n}{comm}\PYG{p}{)}
\end{sphinxVerbatim}

\sphinxAtStartPar
パラメーター
\begin{quote}

\begin{sphinxVerbatim}[commandchars=\\\{\}]
\PYG{k+kt}{INTEGER}\PYG{p}{,}\PYG{k}{INTENT}\PYG{p}{(}\PYG{n}{IN}\PYG{p}{)}\PYG{+w}{ }\PYG{k+kd}{::}\PYG{+w}{ }\PYG{n}{ltetra}
\end{sphinxVerbatim}
\begin{quote}

\sphinxAtStartPar
テトラへドロン法の種類を決める.
1 \(\cdots\) 線形テトラへドロン法,
2 \(\cdots\) 最適化線形テトラへドロン法 {\hyperref[\detokenize{ref:ref}]{\sphinxcrossref{\DUrole{std,std-ref}{{[}1{]}}}}}
\end{quote}

\begin{sphinxVerbatim}[commandchars=\\\{\}]
\PYG{k+kt}{REAL}\PYG{p}{(}\PYG{l+m+mi}{8}\PYG{p}{)}\PYG{p}{,}\PYG{k}{INTENT}\PYG{p}{(}\PYG{n}{IN}\PYG{p}{)}\PYG{+w}{ }\PYG{k+kd}{::}\PYG{+w}{ }\PYG{n}{bvec}\PYG{p}{(}\PYG{l+m+mi}{3}\PYG{p}{,}\PYG{l+m+mi}{3}\PYG{p}{)}
\end{sphinxVerbatim}
\begin{quote}

\sphinxAtStartPar
逆格子ベクトル. 単位は任意で良い.
逆格子の形によって四面体の切り方を決めるため,
それらの長さの比のみが必要であるため.
\end{quote}

\begin{sphinxVerbatim}[commandchars=\\\{\}]
\PYG{k+kt}{INTEGER}\PYG{p}{,}\PYG{k}{INTENT}\PYG{p}{(}\PYG{n}{IN}\PYG{p}{)}\PYG{+w}{ }\PYG{k+kd}{::}\PYG{+w}{ }\PYG{n}{nb}
\end{sphinxVerbatim}
\begin{quote}

\sphinxAtStartPar
バンド本数
\end{quote}

\begin{sphinxVerbatim}[commandchars=\\\{\}]
\PYG{k+kt}{INTEGER}\PYG{p}{,}\PYG{k}{INTENT}\PYG{p}{(}\PYG{n}{IN}\PYG{p}{)}\PYG{+w}{ }\PYG{k+kd}{::}\PYG{+w}{ }\PYG{n}{nge}\PYG{p}{(}\PYG{l+m+mi}{3}\PYG{p}{)}
\end{sphinxVerbatim}
\begin{quote}

\sphinxAtStartPar
軌道エネルギーのメッシュ数.
\end{quote}

\begin{sphinxVerbatim}[commandchars=\\\{\}]
\PYG{k+kt}{REAL}\PYG{p}{(}\PYG{l+m+mi}{8}\PYG{p}{)}\PYG{p}{,}\PYG{k}{INTENT}\PYG{p}{(}\PYG{n}{IN}\PYG{p}{)}\PYG{+w}{ }\PYG{k+kd}{::}\PYG{+w}{ }\PYG{n}{eig1}\PYG{p}{(}\PYG{n}{nb}\PYG{p}{,}\PYG{n}{nge}\PYG{p}{(}\PYG{l+m+mi}{1}\PYG{p}{)}\PYG{p}{,}\PYG{n}{nge}\PYG{p}{(}\PYG{l+m+mi}{2}\PYG{p}{)}\PYG{p}{,}\PYG{n}{nge}\PYG{p}{(}\PYG{l+m+mi}{3}\PYG{p}{)}\PYG{p}{)}
\end{sphinxVerbatim}
\begin{quote}

\sphinxAtStartPar
軌道エネルギー.
Fermi エネルギーを基準とすること
( \(\varepsilon_{\rm F}=0\) ). \sphinxcode{\sphinxupquote{eig2}} も同様.
\end{quote}

\begin{sphinxVerbatim}[commandchars=\\\{\}]
\PYG{k+kt}{REAL}\PYG{p}{(}\PYG{l+m+mi}{8}\PYG{p}{)}\PYG{p}{,}\PYG{k}{INTENT}\PYG{p}{(}\PYG{n}{IN}\PYG{p}{)}\PYG{+w}{ }\PYG{k+kd}{::}\PYG{+w}{ }\PYG{n}{eig2}\PYG{p}{(}\PYG{n}{nb}\PYG{p}{,}\PYG{n}{nge}\PYG{p}{(}\PYG{l+m+mi}{1}\PYG{p}{)}\PYG{p}{,}\PYG{n}{nge}\PYG{p}{(}\PYG{l+m+mi}{2}\PYG{p}{)}\PYG{p}{,}\PYG{n}{nge}\PYG{p}{(}\PYG{l+m+mi}{3}\PYG{p}{)}\PYG{p}{)}
\end{sphinxVerbatim}
\begin{quote}

\sphinxAtStartPar
軌道エネルギー.
移行運動量の分だけグリッドをずらしたものなど.
\end{quote}

\begin{sphinxVerbatim}[commandchars=\\\{\}]
\PYG{k+kt}{INTEGER}\PYG{p}{,}\PYG{k}{INTENT}\PYG{p}{(}\PYG{n}{IN}\PYG{p}{)}\PYG{+w}{ }\PYG{k+kd}{::}\PYG{+w}{ }\PYG{n}{ngw}\PYG{p}{(}\PYG{l+m+mi}{3}\PYG{p}{)}
\end{sphinxVerbatim}
\begin{quote}

\sphinxAtStartPar
積分重みの \(k\) メッシュ.
\sphinxcode{\sphinxupquote{nge}} と違っていても構わない({\hyperref[\detokenize{app:app}]{\sphinxcrossref{\DUrole{std,std-ref}{補遺}}}} 参照).
\end{quote}

\begin{sphinxVerbatim}[commandchars=\\\{\}]
\PYG{k+kt}{REAL}\PYG{p}{(}\PYG{l+m+mi}{8}\PYG{p}{)}\PYG{p}{,}\PYG{k}{INTENT}\PYG{p}{(}\PYG{n}{OUT}\PYG{p}{)}\PYG{+w}{ }\PYG{k+kd}{::}\PYG{+w}{ }\PYG{n}{wght}\PYG{p}{(}\PYG{n}{nb}\PYG{p}{,}\PYG{n}{nb}\PYG{p}{,}\PYG{n}{ngw}\PYG{p}{(}\PYG{l+m+mi}{1}\PYG{p}{)}\PYG{p}{,}\PYG{n}{ngw}\PYG{p}{(}\PYG{l+m+mi}{2}\PYG{p}{)}\PYG{p}{,}\PYG{n}{ngw}\PYG{p}{(}\PYG{l+m+mi}{3}\PYG{p}{)}\PYG{p}{)}
\end{sphinxVerbatim}
\begin{quote}

\sphinxAtStartPar
積分重み
\end{quote}

\begin{sphinxVerbatim}[commandchars=\\\{\}]
\PYG{k+kt}{INTEGER}\PYG{p}{,}\PYG{k}{INTENT}\PYG{p}{(}\PYG{n}{IN}\PYG{p}{)}\PYG{p}{,}\PYG{k}{OPTIONAL}\PYG{+w}{ }\PYG{k+kd}{::}\PYG{+w}{ }\PYG{n}{comm}
\end{sphinxVerbatim}
\begin{quote}

\sphinxAtStartPar
オプショナル引数.
MPIのコミニュケーター( \sphinxcode{\sphinxupquote{MPI\_COMM\_WORLD}} など)を入れる.
libtetrabz を内部でMPI/Hybrid並列するときのみ入力する.
C言語では使用しないときには \sphinxcode{\sphinxupquote{NULL}} を入れる.
\end{quote}
\end{quote}


\section{フォノン線幅等}
\label{\detokenize{routine:id6}}\begin{equation*}
\begin{split}\begin{align}
\sum_{n n'}
\int_{\rm BZ} \frac{d^3 k}{V_{\rm BZ}}
\theta(\varepsilon_{\rm F} -
\varepsilon_{n k}) \theta(\varepsilon'_{n' k} - \varepsilon_{\rm F})
\delta(\varepsilon'_{n' k} - \varepsilon_{n k} - \omega)
X_{n n' k}(\omega)
\end{align}\end{split}
\end{equation*}
\begin{sphinxVerbatim}[commandchars=\\\{\}]
\PYG{k}{CALL }\PYG{n}{libtetrabz\PYGZus{}fermigr}\PYG{p}{(}\PYG{n}{ltetra}\PYG{p}{,}\PYG{n}{bvec}\PYG{p}{,}\PYG{n}{nb}\PYG{p}{,}\PYG{n}{nge}\PYG{p}{,}\PYG{n}{eig1}\PYG{p}{,}\PYG{n}{eig2}\PYG{p}{,}\PYG{n}{ngw}\PYG{p}{,}\PYG{n}{wght}\PYG{p}{,}\PYG{n}{ne}\PYG{p}{,}\PYG{n}{e0}\PYG{p}{,}\PYG{n}{comm}\PYG{p}{)}
\end{sphinxVerbatim}

\sphinxAtStartPar
パラメーター
\begin{quote}

\begin{sphinxVerbatim}[commandchars=\\\{\}]
\PYG{k+kt}{INTEGER}\PYG{p}{,}\PYG{k}{INTENT}\PYG{p}{(}\PYG{n}{IN}\PYG{p}{)}\PYG{+w}{ }\PYG{k+kd}{::}\PYG{+w}{ }\PYG{n}{ltetra}
\end{sphinxVerbatim}
\begin{quote}

\sphinxAtStartPar
テトラへドロン法の種類を決める.
1 \(\cdots\) 線形テトラへドロン法,
2 \(\cdots\) 最適化線形テトラへドロン法 {\hyperref[\detokenize{ref:ref}]{\sphinxcrossref{\DUrole{std,std-ref}{{[}1{]}}}}}
\end{quote}

\begin{sphinxVerbatim}[commandchars=\\\{\}]
\PYG{k+kt}{REAL}\PYG{p}{(}\PYG{l+m+mi}{8}\PYG{p}{)}\PYG{p}{,}\PYG{k}{INTENT}\PYG{p}{(}\PYG{n}{IN}\PYG{p}{)}\PYG{+w}{ }\PYG{k+kd}{::}\PYG{+w}{ }\PYG{n}{bvec}\PYG{p}{(}\PYG{l+m+mi}{3}\PYG{p}{,}\PYG{l+m+mi}{3}\PYG{p}{)}
\end{sphinxVerbatim}
\begin{quote}

\sphinxAtStartPar
逆格子ベクトル. 単位は任意で良い.
逆格子の形によって四面体の切り方を決めるため,
それらの長さの比のみが必要であるため.
\end{quote}

\begin{sphinxVerbatim}[commandchars=\\\{\}]
\PYG{k+kt}{INTEGER}\PYG{p}{,}\PYG{k}{INTENT}\PYG{p}{(}\PYG{n}{IN}\PYG{p}{)}\PYG{+w}{ }\PYG{k+kd}{::}\PYG{+w}{ }\PYG{n}{nb}
\end{sphinxVerbatim}
\begin{quote}

\sphinxAtStartPar
バンド本数
\end{quote}

\begin{sphinxVerbatim}[commandchars=\\\{\}]
\PYG{k+kt}{INTEGER}\PYG{p}{,}\PYG{k}{INTENT}\PYG{p}{(}\PYG{n}{IN}\PYG{p}{)}\PYG{+w}{ }\PYG{k+kd}{::}\PYG{+w}{ }\PYG{n}{nge}\PYG{p}{(}\PYG{l+m+mi}{3}\PYG{p}{)}
\end{sphinxVerbatim}
\begin{quote}

\sphinxAtStartPar
軌道エネルギーのメッシュ数.
\end{quote}

\begin{sphinxVerbatim}[commandchars=\\\{\}]
\PYG{k+kt}{REAL}\PYG{p}{(}\PYG{l+m+mi}{8}\PYG{p}{)}\PYG{p}{,}\PYG{k}{INTENT}\PYG{p}{(}\PYG{n}{IN}\PYG{p}{)}\PYG{+w}{ }\PYG{k+kd}{::}\PYG{+w}{ }\PYG{n}{eig1}\PYG{p}{(}\PYG{n}{nb}\PYG{p}{,}\PYG{n}{nge}\PYG{p}{(}\PYG{l+m+mi}{1}\PYG{p}{)}\PYG{p}{,}\PYG{n}{nge}\PYG{p}{(}\PYG{l+m+mi}{2}\PYG{p}{)}\PYG{p}{,}\PYG{n}{nge}\PYG{p}{(}\PYG{l+m+mi}{3}\PYG{p}{)}\PYG{p}{)}
\end{sphinxVerbatim}
\begin{quote}

\sphinxAtStartPar
軌道エネルギー.
Fermi エネルギーを基準とすること
( \(\varepsilon_{\rm F}=0\) ). \sphinxcode{\sphinxupquote{eig2}} も同様.
\end{quote}

\begin{sphinxVerbatim}[commandchars=\\\{\}]
\PYG{k+kt}{REAL}\PYG{p}{(}\PYG{l+m+mi}{8}\PYG{p}{)}\PYG{p}{,}\PYG{k}{INTENT}\PYG{p}{(}\PYG{n}{IN}\PYG{p}{)}\PYG{+w}{ }\PYG{k+kd}{::}\PYG{+w}{ }\PYG{n}{eig2}\PYG{p}{(}\PYG{n}{nb}\PYG{p}{,}\PYG{n}{nge}\PYG{p}{(}\PYG{l+m+mi}{1}\PYG{p}{)}\PYG{p}{,}\PYG{n}{nge}\PYG{p}{(}\PYG{l+m+mi}{2}\PYG{p}{)}\PYG{p}{,}\PYG{n}{nge}\PYG{p}{(}\PYG{l+m+mi}{3}\PYG{p}{)}\PYG{p}{)}
\end{sphinxVerbatim}
\begin{quote}

\sphinxAtStartPar
軌道エネルギー.
移行運動量の分だけグリッドをずらしたものなど.
\end{quote}

\begin{sphinxVerbatim}[commandchars=\\\{\}]
\PYG{k+kt}{INTEGER}\PYG{p}{,}\PYG{k}{INTENT}\PYG{p}{(}\PYG{n}{IN}\PYG{p}{)}\PYG{+w}{ }\PYG{k+kd}{::}\PYG{+w}{ }\PYG{n}{ngw}\PYG{p}{(}\PYG{l+m+mi}{3}\PYG{p}{)}
\end{sphinxVerbatim}
\begin{quote}

\sphinxAtStartPar
積分重みの \(k\) メッシュ.
\sphinxcode{\sphinxupquote{nge}} と違っていても構わない({\hyperref[\detokenize{app:app}]{\sphinxcrossref{\DUrole{std,std-ref}{補遺}}}} 参照).
\end{quote}

\begin{sphinxVerbatim}[commandchars=\\\{\}]
\PYG{k+kt}{REAL}\PYG{p}{(}\PYG{l+m+mi}{8}\PYG{p}{)}\PYG{p}{,}\PYG{k}{INTENT}\PYG{p}{(}\PYG{n}{OUT}\PYG{p}{)}\PYG{+w}{ }\PYG{k+kd}{::}\PYG{+w}{ }\PYG{n}{wght}\PYG{p}{(}\PYG{n}{ne}\PYG{p}{,}\PYG{n}{nb}\PYG{p}{,}\PYG{n}{nb}\PYG{p}{,}\PYG{n}{ngw}\PYG{p}{(}\PYG{l+m+mi}{1}\PYG{p}{)}\PYG{p}{,}\PYG{n}{ngw}\PYG{p}{(}\PYG{l+m+mi}{2}\PYG{p}{)}\PYG{p}{,}\PYG{n}{ngw}\PYG{p}{(}\PYG{l+m+mi}{3}\PYG{p}{)}\PYG{p}{)}
\end{sphinxVerbatim}
\begin{quote}

\sphinxAtStartPar
積分重み
\end{quote}

\begin{sphinxVerbatim}[commandchars=\\\{\}]
\PYG{k+kt}{INTEGER}\PYG{p}{,}\PYG{k}{INTENT}\PYG{p}{(}\PYG{n}{IN}\PYG{p}{)}\PYG{+w}{ }\PYG{k+kd}{::}\PYG{+w}{ }\PYG{n}{ne}
\end{sphinxVerbatim}
\begin{quote}

\sphinxAtStartPar
フォノンモード数
\end{quote}

\begin{sphinxVerbatim}[commandchars=\\\{\}]
\PYG{k+kt}{REAL}\PYG{p}{(}\PYG{l+m+mi}{8}\PYG{p}{)}\PYG{p}{,}\PYG{k}{INTENT}\PYG{p}{(}\PYG{n}{IN}\PYG{p}{)}\PYG{+w}{ }\PYG{k+kd}{::}\PYG{+w}{ }\PYG{n}{e0}\PYG{p}{(}\PYG{n}{ne}\PYG{p}{)}
\end{sphinxVerbatim}
\begin{quote}

\sphinxAtStartPar
フォノン振動数
\end{quote}

\begin{sphinxVerbatim}[commandchars=\\\{\}]
\PYG{k+kt}{INTEGER}\PYG{p}{,}\PYG{k}{INTENT}\PYG{p}{(}\PYG{n}{IN}\PYG{p}{)}\PYG{p}{,}\PYG{k}{OPTIONAL}\PYG{+w}{ }\PYG{k+kd}{::}\PYG{+w}{ }\PYG{n}{comm}
\end{sphinxVerbatim}
\begin{quote}

\sphinxAtStartPar
オプショナル引数.
MPIのコミニュケーター( \sphinxcode{\sphinxupquote{MPI\_COMM\_WORLD}} など)を入れる.
libtetrabz を内部でMPI/Hybrid並列するときのみ入力する.
C言語では使用しないときには \sphinxcode{\sphinxupquote{NULL}} を入れる.
\end{quote}
\end{quote}


\section{分極関数(複素振動数)}
\label{\detokenize{routine:id7}}\begin{equation*}
\begin{split}\begin{align}
\sum_{n n'}
\int_{\rm BZ} \frac{d^3 k}{V_{\rm BZ}}
\frac{\theta(\varepsilon_{\rm F} - \varepsilon_{n k})
\theta(\varepsilon'_{n' k} - \varepsilon_{\rm F})}
{\varepsilon'_{n' k} - \varepsilon_{n k} + i \omega}
X_{n n' k}(\omega)
\end{align}\end{split}
\end{equation*}
\begin{sphinxVerbatim}[commandchars=\\\{\}]
\PYG{k}{CALL }\PYG{n}{libtetrabz\PYGZus{}polcmplx}\PYG{p}{(}\PYG{n}{ltetra}\PYG{p}{,}\PYG{n}{bvec}\PYG{p}{,}\PYG{n}{nb}\PYG{p}{,}\PYG{n}{nge}\PYG{p}{,}\PYG{n}{eig1}\PYG{p}{,}\PYG{n}{eig2}\PYG{p}{,}\PYG{n}{ngw}\PYG{p}{,}\PYG{n}{wght}\PYG{p}{,}\PYG{n}{ne}\PYG{p}{,}\PYG{n}{e0}\PYG{p}{,}\PYG{n}{comm}\PYG{p}{)}
\end{sphinxVerbatim}

\sphinxAtStartPar
パラメーター
\begin{quote}

\begin{sphinxVerbatim}[commandchars=\\\{\}]
\PYG{k+kt}{INTEGER}\PYG{p}{,}\PYG{k}{INTENT}\PYG{p}{(}\PYG{n}{IN}\PYG{p}{)}\PYG{+w}{ }\PYG{k+kd}{::}\PYG{+w}{ }\PYG{n}{ltetra}
\end{sphinxVerbatim}
\begin{quote}

\sphinxAtStartPar
テトラへドロン法の種類を決める.
1 \(\cdots\) 線形テトラへドロン法,
2 \(\cdots\) 最適化線形テトラへドロン法 {\hyperref[\detokenize{ref:ref}]{\sphinxcrossref{\DUrole{std,std-ref}{{[}1{]}}}}}
\end{quote}

\begin{sphinxVerbatim}[commandchars=\\\{\}]
\PYG{k+kt}{REAL}\PYG{p}{(}\PYG{l+m+mi}{8}\PYG{p}{)}\PYG{p}{,}\PYG{k}{INTENT}\PYG{p}{(}\PYG{n}{IN}\PYG{p}{)}\PYG{+w}{ }\PYG{k+kd}{::}\PYG{+w}{ }\PYG{n}{bvec}\PYG{p}{(}\PYG{l+m+mi}{3}\PYG{p}{,}\PYG{l+m+mi}{3}\PYG{p}{)}
\end{sphinxVerbatim}
\begin{quote}

\sphinxAtStartPar
逆格子ベクトル. 単位は任意で良い.
逆格子の形によって四面体の切り方を決めるため,
それらの長さの比のみが必要であるため.
\end{quote}

\begin{sphinxVerbatim}[commandchars=\\\{\}]
\PYG{k+kt}{INTEGER}\PYG{p}{,}\PYG{k}{INTENT}\PYG{p}{(}\PYG{n}{IN}\PYG{p}{)}\PYG{+w}{ }\PYG{k+kd}{::}\PYG{+w}{ }\PYG{n}{nb}
\end{sphinxVerbatim}
\begin{quote}

\sphinxAtStartPar
バンド本数
\end{quote}

\begin{sphinxVerbatim}[commandchars=\\\{\}]
\PYG{k+kt}{INTEGER}\PYG{p}{,}\PYG{k}{INTENT}\PYG{p}{(}\PYG{n}{IN}\PYG{p}{)}\PYG{+w}{ }\PYG{k+kd}{::}\PYG{+w}{ }\PYG{n}{nge}\PYG{p}{(}\PYG{l+m+mi}{3}\PYG{p}{)}
\end{sphinxVerbatim}
\begin{quote}

\sphinxAtStartPar
軌道エネルギーのメッシュ数.
\end{quote}

\begin{sphinxVerbatim}[commandchars=\\\{\}]
\PYG{k+kt}{REAL}\PYG{p}{(}\PYG{l+m+mi}{8}\PYG{p}{)}\PYG{p}{,}\PYG{k}{INTENT}\PYG{p}{(}\PYG{n}{IN}\PYG{p}{)}\PYG{+w}{ }\PYG{k+kd}{::}\PYG{+w}{ }\PYG{n}{eig1}\PYG{p}{(}\PYG{n}{nb}\PYG{p}{,}\PYG{n}{nge}\PYG{p}{(}\PYG{l+m+mi}{1}\PYG{p}{)}\PYG{p}{,}\PYG{n}{nge}\PYG{p}{(}\PYG{l+m+mi}{2}\PYG{p}{)}\PYG{p}{,}\PYG{n}{nge}\PYG{p}{(}\PYG{l+m+mi}{3}\PYG{p}{)}\PYG{p}{)}
\end{sphinxVerbatim}
\begin{quote}

\sphinxAtStartPar
軌道エネルギー.
Fermi エネルギーを基準とすること
( \(\varepsilon_{\rm F}=0\) ). \sphinxcode{\sphinxupquote{eig2}} も同様.
\end{quote}

\begin{sphinxVerbatim}[commandchars=\\\{\}]
\PYG{k+kt}{REAL}\PYG{p}{(}\PYG{l+m+mi}{8}\PYG{p}{)}\PYG{p}{,}\PYG{k}{INTENT}\PYG{p}{(}\PYG{n}{IN}\PYG{p}{)}\PYG{+w}{ }\PYG{k+kd}{::}\PYG{+w}{ }\PYG{n}{eig2}\PYG{p}{(}\PYG{n}{nb}\PYG{p}{,}\PYG{n}{nge}\PYG{p}{(}\PYG{l+m+mi}{1}\PYG{p}{)}\PYG{p}{,}\PYG{n}{nge}\PYG{p}{(}\PYG{l+m+mi}{2}\PYG{p}{)}\PYG{p}{,}\PYG{n}{nge}\PYG{p}{(}\PYG{l+m+mi}{3}\PYG{p}{)}\PYG{p}{)}
\end{sphinxVerbatim}
\begin{quote}

\sphinxAtStartPar
軌道エネルギー.
移行運動量の分だけグリッドをずらしたものなど.
\end{quote}

\begin{sphinxVerbatim}[commandchars=\\\{\}]
\PYG{k+kt}{INTEGER}\PYG{p}{,}\PYG{k}{INTENT}\PYG{p}{(}\PYG{n}{IN}\PYG{p}{)}\PYG{+w}{ }\PYG{k+kd}{::}\PYG{+w}{ }\PYG{n}{ngw}\PYG{p}{(}\PYG{l+m+mi}{3}\PYG{p}{)}
\end{sphinxVerbatim}
\begin{quote}

\sphinxAtStartPar
積分重みの \(k\) メッシュ.
\sphinxcode{\sphinxupquote{nge}} と違っていても構わない({\hyperref[\detokenize{app:app}]{\sphinxcrossref{\DUrole{std,std-ref}{補遺}}}} 参照).
\end{quote}

\begin{sphinxVerbatim}[commandchars=\\\{\}]
\PYG{k+kt}{COMPLEX}\PYG{p}{(}\PYG{l+m+mi}{8}\PYG{p}{)}\PYG{p}{,}\PYG{k}{INTENT}\PYG{p}{(}\PYG{n}{OUT}\PYG{p}{)}\PYG{+w}{ }\PYG{k+kd}{::}\PYG{+w}{ }\PYG{n}{wght}\PYG{p}{(}\PYG{n}{ne}\PYG{p}{,}\PYG{n}{nb}\PYG{p}{,}\PYG{n}{nb}\PYG{p}{,}\PYG{n}{ngw}\PYG{p}{(}\PYG{l+m+mi}{1}\PYG{p}{)}\PYG{p}{,}\PYG{n}{ngw}\PYG{p}{(}\PYG{l+m+mi}{2}\PYG{p}{)}\PYG{p}{,}\PYG{n}{ngw}\PYG{p}{(}\PYG{l+m+mi}{3}\PYG{p}{)}\PYG{p}{)}
\end{sphinxVerbatim}
\begin{quote}

\sphinxAtStartPar
積分重み .
\end{quote}

\begin{sphinxVerbatim}[commandchars=\\\{\}]
\PYG{k+kt}{INTEGER}\PYG{p}{,}\PYG{k}{INTENT}\PYG{p}{(}\PYG{n}{IN}\PYG{p}{)}\PYG{+w}{ }\PYG{k+kd}{::}\PYG{+w}{ }\PYG{n}{ne}
\end{sphinxVerbatim}
\begin{quote}

\sphinxAtStartPar
計算を行う虚振動数の点数
\end{quote}

\begin{sphinxVerbatim}[commandchars=\\\{\}]
\PYG{k+kt}{COMPLEX}\PYG{p}{(}\PYG{l+m+mi}{8}\PYG{p}{)}\PYG{p}{,}\PYG{k}{INTENT}\PYG{p}{(}\PYG{n}{IN}\PYG{p}{)}\PYG{+w}{ }\PYG{k+kd}{::}\PYG{+w}{ }\PYG{n}{e0}\PYG{p}{(}\PYG{n}{ne}\PYG{p}{)}
\end{sphinxVerbatim}
\begin{quote}

\sphinxAtStartPar
計算を行う複素振動数
\end{quote}

\begin{sphinxVerbatim}[commandchars=\\\{\}]
\PYG{k+kt}{INTEGER}\PYG{p}{,}\PYG{k}{INTENT}\PYG{p}{(}\PYG{n}{IN}\PYG{p}{)}\PYG{p}{,}\PYG{k}{OPTIONAL}\PYG{+w}{ }\PYG{k+kd}{::}\PYG{+w}{ }\PYG{n}{comm}
\end{sphinxVerbatim}
\begin{quote}

\sphinxAtStartPar
オプショナル引数.
MPIのコミニュケーター( \sphinxcode{\sphinxupquote{MPI\_COMM\_WORLD}} など)を入れる.
libtetrabz を内部でMPI/Hybrid並列するときのみ入力する.
C言語では使用しないときには \sphinxcode{\sphinxupquote{NULL}} を入れる.
\end{quote}
\end{quote}

\sphinxstepscope


\chapter{サンプルコード(抜粋)}
\label{\detokenize{sample:id1}}\label{\detokenize{sample::doc}}
\sphinxAtStartPar
以下では電荷密度
\begin{equation*}
\begin{split}\begin{align}
\rho(r) = 2 \sum_{n k} \theta(\varepsilon_{\rm F} - \varepsilon_{n k})
|\varphi_{n k}(r)|^2
\end{align}\end{split}
\end{equation*}
\sphinxAtStartPar
を計算するサブルーチンを示す.

\begin{sphinxVerbatim}[commandchars=\\\{\}]
\PYG{k}{SUBROUTINE }\PYG{n}{calc\PYGZus{}rho}\PYG{p}{(}\PYG{n}{nr}\PYG{p}{,}\PYG{n}{nb}\PYG{p}{,}\PYG{n}{ng}\PYG{p}{,}\PYG{n}{nelec}\PYG{p}{,}\PYG{n}{bvec}\PYG{p}{,}\PYG{n}{eig}\PYG{p}{,}\PYG{n}{ef}\PYG{p}{,}\PYG{n}{phi}\PYG{p}{,}\PYG{n}{rho}\PYG{p}{)}
\PYG{+w}{  }\PYG{c}{!}
\PYG{+w}{  }\PYG{k}{USE }\PYG{n}{libtetrabz}\PYG{p}{,}\PYG{+w}{ }\PYG{k}{ONLY}\PYG{+w}{ }\PYG{p}{:}\PYG{+w}{ }\PYG{n}{libtetrabz\PYGZus{}fermieng}
\PYG{+w}{  }\PYG{k}{IMPLICIT }\PYG{k}{NONE}
\PYG{+w}{  }\PYG{c}{!}
\PYG{+w}{  }\PYG{k+kt}{INTEGER}\PYG{p}{,}\PYG{k}{INTENT}\PYG{p}{(}\PYG{n}{IN}\PYG{p}{)}\PYG{+w}{ }\PYG{k+kd}{::}\PYG{+w}{ }\PYG{n}{nr}\PYG{+w}{ }\PYG{c}{! number of r}
\PYG{+w}{  }\PYG{k+kt}{INTEGER}\PYG{p}{,}\PYG{k}{INTENT}\PYG{p}{(}\PYG{n}{IN}\PYG{p}{)}\PYG{+w}{ }\PYG{k+kd}{::}\PYG{+w}{ }\PYG{n}{nb}\PYG{+w}{ }\PYG{c}{! number of bands}
\PYG{+w}{  }\PYG{k+kt}{INTEGER}\PYG{p}{,}\PYG{k}{INTENT}\PYG{p}{(}\PYG{n}{IN}\PYG{p}{)}\PYG{+w}{ }\PYG{k+kd}{::}\PYG{+w}{ }\PYG{n}{ng}\PYG{p}{(}\PYG{l+m+mi}{3}\PYG{p}{)}
\PYG{+w}{  }\PYG{c}{! k\PYGZhy{}point mesh}
\PYG{+w}{  }\PYG{k+kt}{REAL}\PYG{p}{(}\PYG{l+m+mi}{8}\PYG{p}{)}\PYG{p}{,}\PYG{k}{INTENT}\PYG{p}{(}\PYG{n}{IN}\PYG{p}{)}\PYG{+w}{ }\PYG{k+kd}{::}\PYG{+w}{ }\PYG{n}{nelec}\PYG{+w}{ }\PYG{c}{! number of electrons per spin}
\PYG{+w}{  }\PYG{k+kt}{REAL}\PYG{p}{(}\PYG{l+m+mi}{8}\PYG{p}{)}\PYG{p}{,}\PYG{k}{INTENT}\PYG{p}{(}\PYG{n}{IN}\PYG{p}{)}\PYG{+w}{ }\PYG{k+kd}{::}\PYG{+w}{ }\PYG{n}{bvec}\PYG{p}{(}\PYG{l+m+mi}{3}\PYG{p}{,}\PYG{l+m+mi}{3}\PYG{p}{)}\PYG{+w}{ }\PYG{c}{! reciplocal lattice vector}
\PYG{+w}{  }\PYG{k+kt}{REAL}\PYG{p}{(}\PYG{l+m+mi}{8}\PYG{p}{)}\PYG{p}{,}\PYG{k}{INTENT}\PYG{p}{(}\PYG{n}{IN}\PYG{p}{)}\PYG{+w}{ }\PYG{k+kd}{::}\PYG{+w}{ }\PYG{n}{eig}\PYG{p}{(}\PYG{n}{nb}\PYG{p}{,}\PYG{n}{ng}\PYG{p}{(}\PYG{l+m+mi}{1}\PYG{p}{)}\PYG{p}{,}\PYG{n}{ng}\PYG{p}{(}\PYG{l+m+mi}{2}\PYG{p}{)}\PYG{p}{,}\PYG{n}{ng}\PYG{p}{(}\PYG{l+m+mi}{3}\PYG{p}{)}\PYG{p}{)}\PYG{+w}{ }\PYG{c}{! Kohn\PYGZhy{}Sham eigenvalues}
\PYG{+w}{  }\PYG{k+kt}{REAL}\PYG{p}{(}\PYG{l+m+mi}{8}\PYG{p}{)}\PYG{p}{,}\PYG{k}{INTENT}\PYG{p}{(}\PYG{n}{OUT}\PYG{p}{)}\PYG{+w}{ }\PYG{k+kd}{::}\PYG{+w}{ }\PYG{n}{ef}\PYG{+w}{ }\PYG{c}{! Fermi energy}
\PYG{+w}{  }\PYG{k+kt}{COMPLEX}\PYG{p}{(}\PYG{l+m+mi}{8}\PYG{p}{)}\PYG{p}{,}\PYG{k}{INTENT}\PYG{p}{(}\PYG{n}{IN}\PYG{p}{)}\PYG{+w}{ }\PYG{k+kd}{::}\PYG{+w}{ }\PYG{n}{phi}\PYG{p}{(}\PYG{n}{nr}\PYG{p}{,}\PYG{n}{nb}\PYG{p}{,}\PYG{n}{ng}\PYG{p}{(}\PYG{l+m+mi}{1}\PYG{p}{)}\PYG{p}{,}\PYG{n}{ng}\PYG{p}{(}\PYG{l+m+mi}{2}\PYG{p}{)}\PYG{p}{,}\PYG{n}{ng}\PYG{p}{(}\PYG{l+m+mi}{3}\PYG{p}{)}\PYG{p}{)}\PYG{+w}{ }\PYG{c}{! Kohn\PYGZhy{}Sham orbitals}
\PYG{+w}{  }\PYG{k+kt}{REAL}\PYG{p}{(}\PYG{l+m+mi}{8}\PYG{p}{)}\PYG{p}{,}\PYG{k}{INTENT}\PYG{p}{(}\PYG{n}{OUT}\PYG{p}{)}\PYG{+w}{ }\PYG{k+kd}{::}\PYG{+w}{ }\PYG{n}{rho}\PYG{p}{(}\PYG{n}{nr}\PYG{p}{)}\PYG{+w}{ }\PYG{c}{! Charge density}
\PYG{+w}{  }\PYG{c}{!}
\PYG{+w}{  }\PYG{k+kt}{INTEGER}\PYG{+w}{ }\PYG{k+kd}{::}\PYG{+w}{ }\PYG{n}{ib}\PYG{p}{,}\PYG{+w}{ }\PYG{n}{i1}\PYG{p}{,}\PYG{+w}{ }\PYG{n}{i2}\PYG{p}{,}\PYG{+w}{ }\PYG{n}{i3}\PYG{p}{,}\PYG{+w}{ }\PYG{n}{ltetra}
\PYG{+w}{  }\PYG{k+kt}{REAL}\PYG{p}{(}\PYG{l+m+mi}{8}\PYG{p}{)}\PYG{+w}{ }\PYG{k+kd}{::}\PYG{+w}{ }\PYG{n}{wght}\PYG{p}{(}\PYG{n}{nb}\PYG{p}{,}\PYG{n}{ng}\PYG{p}{(}\PYG{l+m+mi}{1}\PYG{p}{)}\PYG{p}{,}\PYG{n}{ng}\PYG{p}{(}\PYG{l+m+mi}{2}\PYG{p}{)}\PYG{p}{,}\PYG{n}{ng}\PYG{p}{(}\PYG{l+m+mi}{3}\PYG{p}{)}\PYG{p}{)}
\PYG{+w}{  }\PYG{c}{!}
\PYG{+w}{  }\PYG{n}{ltetra}\PYG{+w}{ }\PYG{o}{=}\PYG{+w}{ }\PYG{l+m+mi}{2}
\PYG{+w}{  }\PYG{c}{!}
\PYG{+w}{  }\PYG{k}{CALL }\PYG{n}{libtetrabz\PYGZus{}fermieng}\PYG{p}{(}\PYG{n}{ltetra}\PYG{p}{,}\PYG{n}{bvec}\PYG{p}{,}\PYG{n}{nb}\PYG{p}{,}\PYG{n}{ng}\PYG{p}{,}\PYG{n}{eig}\PYG{p}{,}\PYG{n}{ng}\PYG{p}{,}\PYG{n}{wght}\PYG{p}{,}\PYG{n}{ef}\PYG{p}{,}\PYG{n}{nelec}\PYG{p}{)}
\PYG{+w}{  }\PYG{c}{!}
\PYG{+w}{  }\PYG{n}{rho}\PYG{p}{(}\PYG{l+m+mi}{1}\PYG{p}{:}\PYG{n}{nr}\PYG{p}{)}\PYG{+w}{ }\PYG{o}{=}\PYG{+w}{ }\PYG{l+m+mi}{0}\PYG{n}{d0}
\PYG{+w}{  }\PYG{k}{DO }\PYG{n}{i1}\PYG{+w}{ }\PYG{o}{=}\PYG{+w}{ }\PYG{l+m+mi}{1}\PYG{p}{,}\PYG{+w}{ }\PYG{n}{ng}\PYG{p}{(}\PYG{l+m+mi}{3}\PYG{p}{)}
\PYG{+w}{     }\PYG{k}{DO }\PYG{n}{i2}\PYG{+w}{ }\PYG{o}{=}\PYG{+w}{ }\PYG{l+m+mi}{1}\PYG{p}{,}\PYG{+w}{ }\PYG{n}{ng}\PYG{p}{(}\PYG{l+m+mi}{2}\PYG{p}{)}
\PYG{+w}{        }\PYG{k}{DO }\PYG{n}{i1}\PYG{+w}{ }\PYG{o}{=}\PYG{+w}{ }\PYG{l+m+mi}{1}\PYG{p}{,}\PYG{+w}{ }\PYG{n}{ng}\PYG{p}{(}\PYG{l+m+mi}{1}\PYG{p}{)}
\PYG{+w}{           }\PYG{k}{DO }\PYG{n}{ib}\PYG{+w}{ }\PYG{o}{=}\PYG{+w}{ }\PYG{l+m+mi}{1}\PYG{p}{,}\PYG{+w}{ }\PYG{n}{nb}
\PYG{+w}{              }\PYG{n}{rho}\PYG{p}{(}\PYG{l+m+mi}{1}\PYG{p}{:}\PYG{n}{nr}\PYG{p}{)}\PYG{+w}{ }\PYG{o}{=}\PYG{+w}{ }\PYG{n}{rho}\PYG{p}{(}\PYG{l+m+mi}{1}\PYG{p}{:}\PYG{n}{nr}\PYG{p}{)}\PYG{+w}{ }\PYG{o}{+}\PYG{+w}{ }\PYG{l+m+mi}{2}\PYG{n}{d0}\PYG{+w}{ }\PYG{o}{*}\PYG{+w}{ }\PYG{n}{wght}\PYG{p}{(}\PYG{n}{ib}\PYG{p}{,}\PYG{n}{i1}\PYG{p}{,}\PYG{n}{i2}\PYG{p}{,}\PYG{n}{i3}\PYG{p}{)}\PYG{+w}{ }\PYG{p}{\PYGZam{}}
\PYG{+w}{              }\PYG{p}{\PYGZam{}}\PYG{+w}{     }\PYG{o}{*}\PYG{+w}{ }\PYG{n+nb}{DBLE}\PYG{p}{(}\PYG{n+nb}{CONJG}\PYG{p}{(}\PYG{n}{phi}\PYG{p}{(}\PYG{l+m+mi}{1}\PYG{p}{:}\PYG{n}{nr}\PYG{p}{,}\PYG{n}{ib}\PYG{p}{,}\PYG{n}{i1}\PYG{p}{,}\PYG{n}{i2}\PYG{p}{,}\PYG{n}{i3}\PYG{p}{)}\PYG{p}{)}\PYG{+w}{ }\PYG{o}{*}\PYG{+w}{ }\PYG{n}{phi}\PYG{p}{(}\PYG{l+m+mi}{1}\PYG{p}{:}\PYG{n}{nr}\PYG{p}{,}\PYG{n}{ib}\PYG{p}{,}\PYG{n}{i1}\PYG{p}{,}\PYG{n}{i2}\PYG{p}{,}\PYG{n}{i3}\PYG{p}{)}\PYG{p}{)}
\PYG{+w}{           }\PYG{k}{END }\PYG{k}{DO}
\PYG{k}{        }\PYG{k}{END }\PYG{k}{DO}
\PYG{k}{     }\PYG{k}{END }\PYG{k}{DO}
\PYG{k}{  }\PYG{k}{END }\PYG{k}{DO}
\PYG{+w}{  }\PYG{c}{!}
\PYG{k}{END }\PYG{k}{SUBROUTINE }\PYG{n}{calc\PYGZus{}rho}
\end{sphinxVerbatim}

\sphinxstepscope


\chapter{プログラムの再配布}
\label{\detokenize{copy:id1}}\label{\detokenize{copy::doc}}

\section{自分のプログラムにlibtetrabzを含める}
\label{\detokenize{copy:libtetrabz}}
\sphinxAtStartPar
libtetrabzは下記の {\hyperref[\detokenize{copy:mitlicense}]{\sphinxcrossref{\DUrole{std,std-ref}{MIT ライセンス}}}} に基づいて配布されている.
これはかいつまんで言うと,
個人的(研究室や共同研究者等のグループ)なプログラムであろうとも,
公開したり売ったりするプログラムであろうとも
自由にコピペしたり改変して良いし,
どのようなライセンスで配布しても構わない, と言うことである.


\section{Autoconfを使わずにlibtetrabzをビルドする}
\label{\detokenize{copy:autoconflibtetrabz}}
\sphinxAtStartPar
このパッケージではAutotools (Autoconf, Aitomake, Libtool)を使ってlibtetrabzをビルドしている.
もし再配布するソースコードにlibtetrabzを含めるときに,
Autoconfの使用に支障がある場合には, 以下の簡易版のMakefileを使うと良い (タブに注意).

\begin{sphinxVerbatim}[commandchars=\\\{\}]
\PYG{n+nv}{F90}\PYG{+w}{ }\PYG{o}{=}\PYG{+w}{ }gfortran
\PYG{n+nv}{FFLAGS}\PYG{+w}{ }\PYG{o}{=}\PYG{+w}{ }\PYGZhy{}fopenmp\PYG{+w}{ }\PYGZhy{}O2\PYG{+w}{ }\PYGZhy{}g

\PYG{n+nv}{OBJS}\PYG{+w}{ }\PYG{o}{=}\PYG{+w}{ }\PYG{l+s+se}{\PYGZbs{}}
libtetrabz.o\PYG{+w}{ }\PYG{l+s+se}{\PYGZbs{}}
libtetrabz\PYGZus{}dbldelta\PYGZus{}mod.o\PYG{+w}{ }\PYG{l+s+se}{\PYGZbs{}}
libtetrabz\PYGZus{}dblstep\PYGZus{}mod.o\PYG{+w}{ }\PYG{l+s+se}{\PYGZbs{}}
libtetrabz\PYGZus{}dos\PYGZus{}mod.o\PYG{+w}{ }\PYG{l+s+se}{\PYGZbs{}}
libtetrabz\PYGZus{}fermigr\PYGZus{}mod.o\PYG{+w}{ }\PYG{l+s+se}{\PYGZbs{}}
libtetrabz\PYGZus{}occ\PYGZus{}mod.o\PYG{+w}{ }\PYG{l+s+se}{\PYGZbs{}}
libtetrabz\PYGZus{}polcmplx\PYGZus{}mod.o\PYG{+w}{ }\PYG{l+s+se}{\PYGZbs{}}
libtetrabz\PYGZus{}polstat\PYGZus{}mod.o\PYG{+w}{ }\PYG{l+s+se}{\PYGZbs{}}
libtetrabz\PYGZus{}common.o\PYG{+w}{ }\PYG{l+s+se}{\PYGZbs{}}

\PYG{n+nf}{.SUFFIXES }\PYG{o}{:}
\PYG{n+nf}{.SUFFIXES }\PYG{o}{:}\PYG{+w}{ }.\PYG{n}{o} .\PYG{n}{F}90

\PYG{n+nf}{libtetrabz.a}\PYG{o}{:}\PYG{k}{\PYGZdl{}(}\PYG{n+nv}{OBJS}\PYG{k}{)}
\PYG{+w}{     }ar\PYG{+w}{ }cr\PYG{+w}{ }\PYG{n+nv}{\PYGZdl{}@}\PYG{+w}{ }\PYG{k}{\PYGZdl{}(}OBJS\PYG{k}{)}

\PYG{n+nf}{.F90.o}\PYG{o}{:}
\PYG{+w}{      }\PYG{k}{\PYGZdl{}(}F90\PYG{k}{)}\PYG{+w}{ }\PYG{k}{\PYGZdl{}(}FFLAGS\PYG{k}{)}\PYG{+w}{ }\PYGZhy{}c\PYG{+w}{ }\PYGZdl{}\PYGZlt{}

\PYG{n+nf}{clean}\PYG{o}{:}
\PYG{+w}{      }rm\PYG{+w}{ }\PYGZhy{}f\PYG{+w}{ }*.a\PYG{+w}{ }*.o\PYG{+w}{ }*.mod

\PYG{n+nf}{libtetrabz.o}\PYG{o}{:}\PYG{n}{libtetrabz\PYGZus{}polcmplx\PYGZus{}mod}.\PYG{n}{o}
\PYG{n+nf}{libtetrabz.o}\PYG{o}{:}\PYG{n}{libtetrabz\PYGZus{}fermigr\PYGZus{}mod}.\PYG{n}{o}
\PYG{n+nf}{libtetrabz.o}\PYG{o}{:}\PYG{n}{libtetrabz\PYGZus{}polstat\PYGZus{}mod}.\PYG{n}{o}
\PYG{n+nf}{libtetrabz.o}\PYG{o}{:}\PYG{n}{libtetrabz\PYGZus{}dbldelta\PYGZus{}mod}.\PYG{n}{o}
\PYG{n+nf}{libtetrabz.o}\PYG{o}{:}\PYG{n}{libtetrabz\PYGZus{}dblstep\PYGZus{}mod}.\PYG{n}{o}
\PYG{n+nf}{libtetrabz.o}\PYG{o}{:}\PYG{n}{libtetrabz\PYGZus{}dos\PYGZus{}mod}.\PYG{n}{o}
\PYG{n+nf}{libtetrabz.o}\PYG{o}{:}\PYG{n}{libtetrabz\PYGZus{}occ\PYGZus{}mod}.\PYG{n}{o}
\PYG{n+nf}{libtetrabz\PYGZus{}dbldelta\PYGZus{}mod.o}\PYG{o}{:}\PYG{n}{libtetrabz\PYGZus{}common}.\PYG{n}{o}
\PYG{n+nf}{libtetrabz\PYGZus{}dblstep\PYGZus{}mod.o}\PYG{o}{:}\PYG{n}{libtetrabz\PYGZus{}common}.\PYG{n}{o}
\PYG{n+nf}{libtetrabz\PYGZus{}dos\PYGZus{}mod.o}\PYG{o}{:}\PYG{n}{libtetrabz\PYGZus{}common}.\PYG{n}{o}
\PYG{n+nf}{libtetrabz\PYGZus{}fermigr\PYGZus{}mod.o}\PYG{o}{:}\PYG{n}{libtetrabz\PYGZus{}common}.\PYG{n}{o}
\PYG{n+nf}{libtetrabz\PYGZus{}occ\PYGZus{}mod.o}\PYG{o}{:}\PYG{n}{libtetrabz\PYGZus{}common}.\PYG{n}{o}
\PYG{n+nf}{libtetrabz\PYGZus{}polcmplx\PYGZus{}mod.o}\PYG{o}{:}\PYG{n}{libtetrabz\PYGZus{}common}.\PYG{n}{o}
\PYG{n+nf}{libtetrabz\PYGZus{}polstat\PYGZus{}mod.o}\PYG{o}{:}\PYG{n}{libtetrabz\PYGZus{}common}.\PYG{n}{o}
\end{sphinxVerbatim}


\section{MIT ライセンス}
\label{\detokenize{copy:mit}}\label{\detokenize{copy:mitlicense}}
\begin{DUlineblock}{0em}
\item[] Copyright (c) 2014 Mitsuaki Kawamura
\item[] 
\item[] 以下に定める条件に従い,
\item[] 本ソフトウェアおよび関連文書のファイル(以下「ソフトウェア」)
\item[] の複製を取得するすべての人に対し,
\item[] ソフトウェアを無制限に扱うことを無償で許可します. これには,
\item[] ソフトウェアの複製を使用, 複写, 変更, 結合, 掲載, 頒布, サブライセンス,
\item[] および/または販売する権利,
\item[] およびソフトウェアを提供する相手に同じことを許可する権利も無制限に含まれます.
\item[] 
\item[] 上記の著作権表示および本許諾表示を,
\item[] ソフトウェアのすべての複製または重要な部分に記載するものとします.
\item[] 
\item[] ソフトウェアは「現状のまま」で, 明示であるか暗黙であるかを問わず,
\item[] 何らの保証もなく提供されます. ここでいう保証とは, 商品性,
\item[] 特定の目的への適合性, および権利非侵害についての保証も含みますが,
\item[] それに限定されるものではありません. 作者または著作権者は, 契約行為,
\item[] 不法行為, またはそれ以外であろうと, ソフトウェアに起因または関連し,
\item[] あるいはソフトウェアの使用またはその他の扱いによって生じる一切の請求,
\item[] 損害, その他の義務について何らの責任も負わないものとします.
\end{DUlineblock}

\sphinxstepscope


\chapter{問い合わせ先}
\label{\detokenize{contact:id1}}\label{\detokenize{contact::doc}}
\sphinxAtStartPar
プログラムのバグや質問は以下のフォーラムへご投稿ください.
\begin{quote}

\sphinxAtStartPar
\sphinxurl{http://sourceforge.jp/projects/libtetrabz/forums/}
\end{quote}

\sphinxAtStartPar
開発に参加したい方は以下の連絡先にて受け付けております.

\sphinxAtStartPar
東京大学物性研究所

\sphinxAtStartPar
河村光晶

\sphinxAtStartPar
\sphinxcode{\sphinxupquote{mkawamura\_\_at\_\_issp.u\sphinxhyphen{}tokyo.ac.jp}}

\sphinxstepscope


\chapter{補遺}
\label{\detokenize{app:app}}\label{\detokenize{app:id1}}\label{\detokenize{app::doc}}

\section{逆補間}
\label{\detokenize{app:id2}}
\sphinxAtStartPar
積分
\begin{equation*}
\begin{split}\begin{align}
\langle X \rangle = \sum_{k} X_k w(\varepsilon_k)
\end{align}\end{split}
\end{equation*}
\sphinxAtStartPar
を計算するとする. このとき,
\begin{itemize}
\item {} 
\sphinxAtStartPar
\(w\) は \(\varepsilon_k\) に敏感な関数(階段関数 \(\cdot\) デルタ関数等)であり,
なるべく細かいグリッド上の \(\varepsilon_k\) が必要である.

\item {} 
\sphinxAtStartPar
\(X_k\) を求めるための計算コストが \(\varepsilon_k\) の計算コストよりかなり大きい.

\end{itemize}

\sphinxAtStartPar
という場合には \(X_k\) のグリッドを補間により増やす方法が有効である.
それは,
\begin{enumerate}
\sphinxsetlistlabels{\arabic}{enumi}{enumii}{}{.}%
\item {} 
\sphinxAtStartPar
\(\varepsilon_k\) を細かい \(k\) グリッド上で計算する.

\item {} 
\sphinxAtStartPar
\(X_k\) を粗いグリッド上で計算し, それを補間(線形補間, 多項式補間,
スプライン補間など)して細かいグリッド上での値を得る.

\end{enumerate}
\begin{equation*}
\begin{split}\begin{align}
X_k^{\rm dense} = \sum_{k'}^{\rm coarse}
F_{k k'} X_{k'}^{\rm coarse}
\end{align}\end{split}
\end{equation*}\begin{enumerate}
\sphinxsetlistlabels{\arabic}{enumi}{enumii}{}{.}%
\item {} 
\sphinxAtStartPar
細かい \(k\) グリッドで上記の積分を行う.

\end{enumerate}
\begin{equation*}
\begin{split}\begin{align}
\langle X \rangle = \sum_{k}^{\rm dense}
X_k^{\rm dense} w_k^{\rm dense}
\end{align}\end{split}
\end{equation*}
\sphinxAtStartPar
という流れで行われる.

\sphinxAtStartPar
さらに,
この計算と同じ結果を得るように粗いグリッド上での積分重み
\(w_k^{\rm coarse}\) を  \(w_k{\rm dense}\) から求める
\sphinxstylestrong{逆補間} ( {\hyperref[\detokenize{ref:ref}]{\sphinxcrossref{\DUrole{std,std-ref}{{[}2{]}}}}} のAppendix)も可能である.
すなわち,
\begin{equation*}
\begin{split}\begin{align}
\sum_k^{\rm dense} X_k^{\rm dense} w_k^{\rm dense}
= \sum_k^{\rm coarse} X_k^{\rm coarse} w_k^{\rm coarse}
\end{align}\end{split}
\end{equation*}
\sphinxAtStartPar
が満たされる事を要請すると
\begin{equation*}
\begin{split}\begin{align}
w_k^{\rm coarse} = \sum_k^{\rm dense} F_{k' k}
w_{k'}^{\rm dense}
\end{align}\end{split}
\end{equation*}
\sphinxAtStartPar
となる. この場合の計算手順は,
\begin{enumerate}
\sphinxsetlistlabels{\arabic}{enumi}{enumii}{}{.}%
\item {} 
\sphinxAtStartPar
細かい \(k\) グリッド上の  \(\varepsilon_k\) から
\(w_k^{\rm dense}\) を計算する.

\item {} 
\sphinxAtStartPar
逆補間により \(w_k^{\rm coarse}\) を求める.

\item {} 
\sphinxAtStartPar
粗いグリッド上での \(X_k\) との積和を行う.

\end{enumerate}

\sphinxAtStartPar
となる. このライブラリ内の全ルーチンはこの逆補間の機能を備えており,
軌道エネルギーの \(k\) グリッドと重み関数の \(k\) グリッドを
異なる値にすると逆補間された \(w_k^{\rm coarse}\) が出力される.


\section{2重デルタ関数の積分}
\label{\detokenize{app:id3}}
\sphinxAtStartPar
For the integration
\begin{equation*}
\begin{split}\begin{align}
\sum_{n n' k} \delta(\varepsilon_{\rm F} -
\varepsilon_{n k}) \delta(\varepsilon_{\rm F} - \varepsilon'_{n' k})
X_{n n' k}
\end{align}\end{split}
\end{equation*}
\sphinxAtStartPar
first, we cut out one or two triangles where
\(\varepsilon_{n k} = \varepsilon_{\rm F}\) from a tetrahedron
and evaluate \(\varepsilon_{n' k+q}\) at the corners of each triangles as
\begin{equation*}
\begin{split}\begin{align}
\varepsilon'^{k+q}_{i} = \sum_{j=1}^4 F_{i j}(
\varepsilon_1^{k}, \cdots, \varepsilon_{4}^{k}, \varepsilon_{\rm F})
\epsilon_{j}^{k+q}.
\end{align}\end{split}
\end{equation*}
\sphinxAtStartPar
Then we calculate \(\delta(\varepsilon_{n' k+q} - \varepsilon{\rm F})\)
in each triangles and obtain weights of corners.
This weights of corners are mapped into those of corners of the original tetrahedron as
\begin{equation*}
\begin{split}\begin{align}
W_{i} = \sum_{j=1}^3 \frac{S}{\nabla_k \varepsilon_k}F_{j i}(
\varepsilon_{1}^k, \cdots, \varepsilon_{4}^k, \varepsilon_{\rm F})
W'_{j}.
\end{align}\end{split}
\end{equation*}
\sphinxAtStartPar
\(F_{i j}\) and \(\frac{S}{\nabla_k \varepsilon_k}\) are calculated as follows
(\(a_{i j} \equiv (\varepsilon_i - \varepsilon_j)/(\varepsilon_{\rm F} - \varepsilon_j)\)):

\begin{figure}[htbp]
\centering
\capstart

\noindent\sphinxincludegraphics[scale=1.0]{{dbldelta}.png}
\caption{How to divide a tetrahedron
in the case of \(\epsilon_1 \leq \varepsilon_{\rm F} \leq \varepsilon_2\) (a),
\(\varepsilon_2 \leq \varepsilon_{\rm F} \leq \varepsilon_3\) (b), and
\(\varepsilon_3 \leq \varepsilon_{\rm F} \leq \varepsilon_4\) (c).}\label{\detokenize{app:id4}}\label{\detokenize{app:dbldeltapng}}\end{figure}
\begin{itemize}
\item {} 
\sphinxAtStartPar
When \(\varepsilon_1 \leq \varepsilon_{\rm F} \leq \varepsilon_2 \leq \varepsilon_3 \leq\varepsilon_4\)
{[}Fig. \ref{app:dbldeltapng} (a){]},
\begin{quote}
\begin{equation*}
\begin{split}\begin{align}
F &=
\begin{pmatrix}
a_{1 2} & a_{2 1} &       0 & 0 \\
a_{1 3} &       0 & a_{3 1} & 0 \\
a_{1 4} &       0 &       0 & a_{4 1}
\end{pmatrix},
\qquad
\frac{S}{\nabla_k \varepsilon_k} = \frac{3 a_{2 1} a_{3 1} a_{4 1}}{\varepsilon_{\rm F} - \varepsilon_1}
\end{align}\end{split}
\end{equation*}\end{quote}

\item {} 
\sphinxAtStartPar
When \(\varepsilon_1 \leq \varepsilon_2 \leq \varepsilon_{\rm F} \leq \varepsilon_3 \leq\varepsilon_4\)
{[}Fig. \ref{app:dbldeltapng} (b){]},
\begin{quote}
\begin{equation*}
\begin{split}\begin{align}
F &=
\begin{pmatrix}
a_{1 3} &       0 & a_{3 1} & 0 \\
a_{1 4} &       0 &       0 & a_{4 1} \\
0 & a_{2 4} &       0 & a_{4 2}
\end{pmatrix},
\qquad
\frac{S}{\nabla_k \varepsilon_k} = \frac{3 a_{3 1} a_{4 1} a_{2 4}}{\varepsilon_{\rm F} - \varepsilon_1}
\end{align}\end{split}
\end{equation*}\begin{equation*}
\begin{split}\begin{align}
F &=
\begin{pmatrix}
a_{1 3} &       0 & a_{3 1} & 0 \\
0 & a_{2 3} & a_{3 2} & 0 \\
0 & a_{2 4} &       0 & a_{4 2}
\end{pmatrix},
\qquad
\frac{S}{\nabla_k \varepsilon_k} = \frac{3 a_{2 3} a_{3 1} a_{4 2}}{\varepsilon_{\rm F} - \varepsilon_1}
\end{align}\end{split}
\end{equation*}\end{quote}

\item {} 
\sphinxAtStartPar
When \(\varepsilon_1 \leq \varepsilon_2 \leq \varepsilon_3 \leq \varepsilon_{\rm F} \leq \varepsilon_4\)
{[}Fig. \ref{app:dbldeltapng} (c){]},
\begin{quote}
\begin{equation*}
\begin{split}\begin{align}
F &=
\begin{pmatrix}
a_{1 4} &       0 &       0 & a_{4 1} \\
a_{1 3} & a_{2 4} &       0 & a_{4 2} \\
a_{1 2} &       0 & a_{3 4} & a_{4 3}
\end{pmatrix},
\qquad
\frac{S}{\nabla_k \varepsilon_k} = \frac{3 a_{1 4} a_{2 4} a_{3 4}}{\varepsilon_1 - \varepsilon_{\rm F}}
\end{align}\end{split}
\end{equation*}\end{quote}

\end{itemize}

\sphinxAtStartPar
Weights on each corners of the triangle are computed as follows
{[}(\(a'_{i j} \equiv (\varepsilon'_i - \varepsilon'_j)/(\varepsilon_{\rm F} - \varepsilon'_j)\)){]}:
\begin{itemize}
\item {} 
\sphinxAtStartPar
When \(\varepsilon'_1 \leq \varepsilon_{\rm F} \leq \varepsilon'_2 \leq \varepsilon'_3\) {[}Fig. \ref{app:dbldeltapng} (d){]},
\begin{quote}
\begin{equation*}
\begin{split}\begin{align}
W'_1 = L (a'_{1 2} + a'_{1 3}), \qquad
W'_2 = L a'_{2 1}, \qquad
W'_3 = L a'_{3 1}, \qquad
L \equiv \frac{a'_{2 1} a'_{3 1}}{\varepsilon_{\rm F} - \varepsilon'_{1}}
\end{align}\end{split}
\end{equation*}\end{quote}

\item {} 
\sphinxAtStartPar
When \(\varepsilon'_1 \leq \varepsilon'_2 \leq \varepsilon_{\rm F} \leq \varepsilon'_3\) {[}Fig. \ref{app:dbldeltapng} (e){]},
\begin{quote}
\begin{equation*}
\begin{split}\begin{align}
W'_1 = L a'_{1 3}, \qquad
W'_2 = L a'_{2 3}, \qquad
W'_3 = L (a'_{3 1} + a'_{3 2}), \qquad
L \equiv \frac{a'_{1 3} a'_{2 3}}{\varepsilon'_{3} - \varepsilon_{\rm F}}
\end{align}\end{split}
\end{equation*}\end{quote}

\end{itemize}

\sphinxstepscope


\chapter{参考文献}
\label{\detokenize{ref:ref}}\label{\detokenize{ref:id1}}\label{\detokenize{ref::doc}}
\sphinxAtStartPar
{[}1{]} \sphinxhref{https://journals.aps.org/prb/abstract/10.1103/PhysRevB.89.094515}{M. Kawamura, Y. Gohda, and S. Tsuneyuki, Phys. Rev. B 89, 094515 (2014).}

\sphinxAtStartPar
{[}2{]} \sphinxhref{https://journals.aps.org/prb/abstract/10.1103/PhysRevB.95.054506}{M. Kawamura, R. Akashi, and S. Tsuneyuki, Phys. Rev. B 95, 054506 (2017).}



\renewcommand{\indexname}{索引}
\printindex
\end{document}